\parskip = \baselineskip
\parindent = 0in

\footline = {\tenrm \hfil Page \folio \hfil 3/30/99 \hfil}

\font\nfont = cmr12
\font\bfont = cmr17
\font\ifont = cmti10 at 12pt
\font\tfont = cmtt10 at 12pt
\font\rfont = cmitt10 at 12pt
\nfont

\def\begcode{\begingroup\parskip = 0in\bigskip\leftskip = .75in\tfont\obeywhitespace}
\def\endcode{\par\endgroup}
\def\res#1{{\rfont{#1}}}
\def\pb{\#$\backslash$}
\def\var#1{{\ifont #1\/}}


\centerline{\bfont \underbar{Scheme Tutorial}}

\bigskip

This tutorial is designed to give a little hands on exposure to scheme
for a person with programming experience.

Courier text is typed by you, and italic text is that
produced by the scheme interpreter.

To evoke the scheme interpreter for command line interaction (as
opposed to GUI/Windows programming) use the following:

\begcode
mzscheme
\endcode

Scheme will respond with a banner and then a ``$>$'' (command) prompt.  At
the command prompt you can type in your command.  Once the command is
completed scheme will respond with the result.  For example:

\begcode
Welcome to MzScheme version 53
>
\endcode

Scheme can be exited by typing the following command:

\begcode
> (exit)
\endcode

Scheme has all the normal data types that other languages have.  When
data is typed in by itself scheme simply responds with itself as
follows:

\begcode
> 45
\res{45}
> 3.14
\res{3.14}
> "This is a string"
\res{"This is a string"}
\endcode

Note that strings are enclosed in double quotes.

Characters are represented in a strange way - they are preceded with a
``\pb'' for example the letter ``x'' is represented as follows:

\begcode
> \pb{x}
\res{\pb{x}}
\endcode

Other non-printable characters are represented as follows:

\begcode
> \pb{space}
\res{\pb{space}}
> \pb{newline}
\res{\pb{newline}}
\endcode

Scheme also has a boolean data type which is returned when performing
tests (such as whether or not two numbers are equal).  \#t represents
TRUE and \#f represents FALSE, for example:

\begcode
> \#t 
\res{\#t}
> \#f
\res{\#f}
\endcode


The boolean data types are used in conditional (\var{if}) statements.

Comments begin with a ``;'' and extend to the end of the line.  Anything
before the ``;'' is not part of the comment and gets executed normally,
for example:

\begcode
> ; A comment
> 45 ; comment
\res{45}
\endcode

In scheme, operations (expressions) are performed in a very consistent
manner. In fact scheme expressions are more consistent than in any
other language.  On one hand this makes the language very simple, but
on the other hand, scheme can look a little unusual.  Once you get
used to it, however, it will not be confusing.

All scheme expressions are enclosed in parentheses.  The first thing
in the parenthesis (or list) is always the operation to be
performed. The remaining things in the parentheses are the arguments
(or things the operation is to be performed on).  For example:

\begcode
> (+ 3 5)
\res{8}
> (+ 3 5 2)
\res{10}
> (- 8 4)
\res{4}
\endcode


Parenthesis (or lists) can be nested arbitrarily deeply.  As you would
expect, the arguments inside the list get performed first.  Scheme
always performs operation in the order specified by the parenthesis
(as opposed to algebraically).  For example:


\begcode
> (+ 4 (* 7 2))
\res{18}
> (* 3 (+ 2 4))
\res{18}
\endcode


Values can be stored in variables.  In the following example \var{define} is
the operation being performed and {\it var} and 33 are its arguments.  The
operation (\var{define}) creates a new variable (its first argument) and
assigns it the value specified in the second argument.  For example:


\begcode
> (define var 33)
\endcode

\var{var} now has the value 33.  It can be used anywhere you might need the value.

\begcode
> var
\res{33}
> (+ var 4)
\res{37}
\endcode

Once a variable had been defined you can use \var{set!} to change its value.

\begcode
> (set! var 55)
> var
\res{55}
\endcode

Functions can be defined to carry out a sequence of pre-defined
operations.  For example the following expression defines a function
named \var{try} which will take two arguments (named \var{a} and
\var{b}).  The function will add the two arguments passed and return
the result.  Scheme function always return the value of the last
expression in the function.


\begcode
> (define (try a b)
          (+ a b))
\endcode

Now that we have defined the function we can use it as defined.

\begcode
> (try 5 4)
\res{9}
> (try 2 5)
\res{7}
> (try 7 90)
\res{97}
\endcode


When test operations are performed boolean values are returned.


\begcode
> (= 3 4)
\res{\#f}
> (= 3 3)
\res{\#t}
\endcode


Conditional statements (\var{if}'s) take two or three arguments.  The first
argument is the test.  The second argument is what to execute if the
test succeeds.  The (optional) third argument is what gets executed if
the test fails.


\begcode
> (if (= 3 3)
      8
      9)
\res{8}
> (if (= 3 4)
      8
      9)
\res{9}
\endcode

Since only one expression gets executed if the test succeeds or fails,
if more than one expression is desired it must be made into a single
expression by grouping it with the \var{begin} operation.  \var{begin}
takes any number of arguments, executes each in succession, and
returns the value of the last expression.


\begcode
> (if (= 3 3)
      (begin
          (+ 5 6)
          (+ 5 5))
      (begin
          (+ 1 6)
          (+ 1 5)))
\res{10}
> (if (= 3 4)
      (begin
          (+ 5 6)
          (+ 5 5))
      (begin
          (+ 1 6)
          (+ 1 5)))
\res{6}
\endcode

Another function useful for multiple conditions is the \var{cond}.
The \var{cond} will execute any number of tests (as opposed to the \var{if}
which only performs a single test) and then execute any number of
expressions which are associated to the successful test.  Like most
scheme expressions the \var{cond} returns the value of the very last
expression evaluated.  For example:

\begcode
> (define var1 5)
> (cond ((= var1 4)
         1)
        ((= var1 5)
         2)
        ((= var1 6)
         3)
        (else 4))
\res{2}
\endcode


Within functions it is often convenient to be able to create temporary
variables.  These variables only exist within the context
(parenthesis) in which they are defined.  The operation \var{let*} is used
to create temporary variables.  They will only be usable within the
\var{let*} context.

The \var{let*} operation takes any number of arguments.  The first argument
is a list containing temporary variables and their initial values.
These initial values may also be expressions.  The remaining argument
are just expressions which get evaluated one after the other.  The
result of the \var{let*} is the value of the last expression evaluated.  For
example:


\begcode
> (let* ((x 4)
         (y (+ 3 2)))
        (+ x y))
\res{9}
> x
\res{reference to undefined identifier: x}
> y
\res{reference to undefined identifier: y}
\endcode


As you can see \var{x} and \var{y} were usable within the \var{let*}
context but don't exist outside of the \var{let*}.

The following example shows a \var{let*} expression inside a function.


\begcode
> (define (fun x)
          (let* ((y x)
                 (z 7))
                (set! y (+ y 4))
                (set! y (+ y z))
                y))
> (fun 8)
\res{19}
> x
\res{reference to undefined identifier: x}
> y
\res{reference to undefined identifier: y}
\endcode


Typically, scheme functions would be typed into a text file (with a
\var{.scm} extension) with a text editor.  You would then load the file to
be run in scheme with the following command:

\begcode
>  (load "Program.scm")
\endcode

The GNU EMACS editor is strongly recommended.  This editor is freely
available.  It automatically formats scheme programs and makes sure
all the parentheses are correct.

\bye
