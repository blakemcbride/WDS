\nopagenumbers

\font\bf=cmbx10
\font\header=cmbx12

\def\section#1{\bigskip  \leftline{\bf #1.} \medskip}

\centerline{\header Scheme Rational}
\bigskip

\section{Language power}

We required a robust, expressive, and full featured language to use as
an extension language.  Unlike many scripting languages, such as pearl
or bash, which have strong facilities in a few domains but fall short
in many other domains, scheme, a dialect of lisp, is a general purpose
language with facilities which arguably surpass languages such as C.

We required the ability to execute programs on the fly without having
to compile or link anything.  Additionally, when the program doesn't
change very often we'd like it to be compiled, without requiring any
development environment, to run fast.  We'd also like the ability to
distribute compiled modules when we want to protect the source code of
those modules.  Scheme provides all of these facilities.


\section{Ease of language}

The purpose of using an extension language is to allow application
extension without requiring all the tools, as well as knowledge to use
such tools, required to augment an application.  For this reason
providing customers with a C development environment, our development
tools, as well as the application modules, would not be practical.
Therefore, C could not be used as the extension language.  If C, a
very well known language, could not be used then some other, less
popular, language must be chosen.  Almost any language chosen as the
extension language would require a learning curve so it would be best
to choose a language that is easy to learn.  Scheme (or lisp) has the
simplest syntax of any language.  The Scheme standard is about 50
pages long while the C standard is 220 pages, and BASIC is 360 pages.

It was never our intent to have end-users use whatever extension
language we chose.  Extensions would predominantly be done in-house on
a consulting basis or by programmers employed by our clients.  Any
competent programmer can learn Scheme in very short order.  We provide
other, simpler to use, facilities to perform functions, such as the
definition of calculations, which end-users may require access to.


\section{Source code availability}

There are many scripting languages choices available.  It is important
to us that we not use something which we cannot control.  Using a
language without the source code leaves us exposed to not being able
to fix bugs on our schedule and relying on the vendor.  We would also
be subject to radical changes made by the vendor or their sale or lack
of support of a future environment.  The Scheme we are using is
written in portable C, and comes with full source code which we are
fully able to support in-house.

\section{Restrictive licensing}

Many languages come with highly costly or restrictive licensing
requirements.  The Scheme we have chosen does not carry any of
that baggage.


\section{Linking with our development tools}

We required a language which would easily link with our existing
development environment.  The Scheme we have chosen was specifically
designed to link with C and act as an embedded extension language.  We
have been able to fully link this Scheme with our development
environment such that we can just as easily write our entire
application in Scheme as we could in C.  All the Windows
functionality, which is accessed through our development tools, is
fully available from within Scheme.  Additionally, this Scheme makes
it trivial to make C application functions directly available from
within Scheme or Scheme functions available from within C.

\bye
