@c -*-texinfo-*-

@c  Copyright (c) 1996 Blake McBride
@c  All rights reserved.
@c
@c  Redistribution and use in source and binary forms, with or without
@c  modification, are permitted provided that the following conditions are
@c  met:
@c
@c  1. Redistributions of source code must retain the above copyright
@c  notice, this list of conditions and the following disclaimer.
@c
@c  2. Redistributions in binary form must reproduce the above copyright
@c  notice, this list of conditions and the following disclaimer in the
@c  documentation and/or other materials provided with the distribution.
@c
@c  THIS SOFTWARE IS PROVIDED BY THE COPYRIGHT HOLDERS AND CONTRIBUTORS
@c  "AS IS" AND ANY EXPRESS OR IMPLIED WARRANTIES, INCLUDING, BUT NOT
@c  LIMITED TO, THE IMPLIED WARRANTIES OF MERCHANTABILITY AND FITNESS FOR
@c  A PARTICULAR PURPOSE ARE DISCLAIMED. IN NO EVENT SHALL THE COPYRIGHT
@c  HOLDER OR CONTRIBUTORS BE LIABLE FOR ANY DIRECT, INDIRECT, INCIDENTAL,
@c  SPECIAL, EXEMPLARY, OR CONSEQUENTIAL DAMAGES (INCLUDING, BUT NOT
@c  LIMITED TO, PROCUREMENT OF SUBSTITUTE GOODS OR SERVICES; LOSS OF USE,
@c  DATA, OR PROFITS; OR BUSINESS INTERRUPTION) HOWEVER CAUSED AND ON ANY
@c  THEORY OF LIABILITY, WHETHER IN CONTRACT, STRICT LIABILITY, OR TORT
@c  (INCLUDING NEGLIGENCE OR OTHERWISE) ARISING IN ANY WAY OUT OF THE USE
@c  OF THIS SOFTWARE, EVEN IF ADVISED OF THE POSSIBILITY OF SUCH DAMAGE.

@chapter Introduction

The Dynace Windows Development System (WDS) is a Dynace class library
which enables a C programmer with no knowledge of C++, Dynace, the
Windows API or message-driven architecture to write real Windows
applications with an absolute minimum learning curve and number of
lines of code.  In fact, it is possible to become familiar enough with
windows, menus, dialogs and controls using WDS to write a Windows
application after just one day!

The programmer is able to write the application using familiar C in a
familiar procedural fashion.  Instead of taking the usual fifty plus
lines of code necessary to implement a typical ``Hello World'' program,
with WDS it takes four lines of code!  Fully functional menus and
dialogs can be implemented with WDS in a handful of lines instead of the
hundreds it takes using the Windows API or other available tools.

WDS applications are portable across Win32 which is the Windows GUI API.
It is also portable to Linux and Mac via the Wine system.

Dynace (pronounced @emph{d@=i-ne-s@=e}), what WDS is based upon, stands
for a ``DYNAmic C language Extension''.  It is an object oriented
extension to the C or C++ languages.  Dynace is written in the C
language and designed to be as portable as possible.  It solves many
problems associated with C++ and adds features previously only available
in languages such as CLOS or Smalltalk without their overhead.  Dynace
is fully documented in another manual.

@section Overview

The C language is the most popular and well known among the language
choices available for the PC and Unix environments.  Therefore, there
are many more programmers who know and feel comfortable with the C
language than any other.  All the popular tools available for software
development under Windows use the C++
language. C++ is a complex superset of the C language with questionable
benefits.  There is a significant learning curve associated with going
from proficient C ability to C++.


Although the currently available tools for Windows development, which
work in association with C++, provide a very high degree of flexibility
and power, they are tremendously complex to learn and use effectively.
Given the tremendous complexity of both the Windows development tools
currently available on the market and the fact that these tools are all
based on C++, the time necessary for a normal C programmer to become
knowledgeable and proficient in Windows development, including both C++
and the Windows tools, is an absolute minimum of six months.  This time
may vary to a period exceeding a year.

The currently available tools require an enormous amount of code (lines
of program text) in order to achieve the most fundamental functionality.
Although the current tools actually generate most of this code, the
bottom line is that any real life application will end up having an
enormous amount of code.  There is a clear relationship between lines of
code and a) how maintainable a program is for finding bugs or making
enhancements, and b) how difficult it is for new programmers to get up
to speed with respect to the new tools and application.  Given the cost
of software development and the tremendous backlog, this issue is of
paramount importance.

The focus of WDS is to enable a normal C programmer to learn and be able
to write and understand a fundamental Windows program in one day.  Given
just a little more time with WDS (in terms of days), the programmer will
be able to write and understand real Windows applications in a minimum
amount of time and with a very minimum amount of code.  There's no need
to learn a new language, such as C++.  The programmer may use his
existing knowledge in C and just needs to learn a very high level and
simple to understand set of tools.  Instead of taking hundreds of lines
of code to add a new dialog, as required by the existing tools, WDS can
accomplish the same task in half a dozen lines.  Instead of taking up to
one hundred lines of code just to bring up the main application window,
WDS just requires four lines!

@section Benefits

There are three main benefits to using WDS over the other available
options.  The first is that the learning curve associated with Dynace
for Windows allows a programmer or programming team to get up-to-speed
with respect to Windows programming in an absolute minimum amount of
time.  The difference in time is days instead of six months or more.
This directly translates into saved dollars and increases the success of
projects.

The second main benefit is that since application features may be
implemented in tens of lines of program code, instead of the hundreds or
thousands of lines required by existing tools, applications may be
developed in a drastically reduced time frame.

The third main benefit is that since WDS is so easy to learn and
requires so few lines of code for application development, applications
developed with WDS are much easier to debug, maintain and enhance.
Programmer turnover is much less a problem due to the fact that new
programmers can get up-to-speed in a minimum amount of time.

With WDS there is never a need for a code generator or wizard since
so few lines are needed to create the application.  All your code
is application specific.

In short, WDS is an invaluable tool for Windows application development
during the learning, development, and maintenance phases of application
development.  WDS is also portable to Linux and Mac via the Wine package.

@section A Short Example

The following complete example illustrates how easy it is to get started
with WDS.  Implementing this exact functionality using the Windows API
would take more then fifty lines of code!

@example
@group
#include "generics.h"

int     start()
@{
        object  win;

        win = vNew(MainWindow, "My Test Application");

        vPrintf(win, "Hello, World!\n");

        return gProcessMessages(win);
@}
@end group
@end example



@section Features
This section enumerates some of the key features of The Dynace Windows
Development System (WDS).

@itemize @bullet
@item Drastically reduces the learning curve associated with Windows
development
@item Drastically reduced the lines of code necessary to build Windows apps
@item Applications are portable between Windows 3.1 (Win16), Win32s,
Windows 95 and Windows NT
@item Applications are easier to debug, enhance and pass off to new programmers
since it is straight C and so few lines
@item Full support for main, popup and child windows, menus,
modal and modeless dialogs, cursors, icons, fonts, brushes, and pens
@item Support for all standard Windows controls as well as several supplied controls
@item Easy to use routines for printing reports
@item Support for most of the Windows common dialogs
@item Compiles with a regular C or C++ compiler -- not an interpreter -- 
runs FAST!
@item Full support for the Windows help system at all levels of context
@item Support for external DLL access
@item Access of full source code to WDS
@item Applications are royalty free (with the appropriate license)
@end itemize

@section Reasons to use WDS
The following lists several reasons to use the Dynace Windows
Development System:

@enumerate

@item
Drastically reduce the learning curve associated with learning to
write for Windows -- be able to write real Windows applications in
a few days instead of the normal 6 months to a year learning curve.

@item
Be able to write Windows apps fast using only a few lines of code --
do ``Hello World'' in 4 lines, menus and dialog in a few lines -- write
completely functional dialogs in a handful of lines instead of the
hundreds it normally takes.

@item
Drastically reduce the time necessary to debug, maintain and enhance the
app since it's so few lines of code.

@item
Write apps which are portable to 16 and 32 bit environments including
Windows 3.1 (Win16), Win32s, other Windows (Win32).

@item
Be much less dependent on a few ``expert'' programmers since the code
is much easier to follow and learn.

@item
No need to learn a new language since WDS apps are normal C code.

@item
Don't be the victim of a vendor since WDS comes with full source code.

@item
Be able to take advantage of the very advanced object oriented
capabilities of the Dynace Object Oriented Extension to C, which
WDS is built on top of.

@item
Your apps will run very fast since they are compiled into optimized
machine code by your compiler.

@item
Integrates well with existing compiler vendor's IDEs.  No need to
learn a new set of tools.

@item
It's easier to port non-Windows apps to Windows with WDS since WDS
encapsulates much of the message-driven architecture associated with
Windows programming.

@item
WDS applications are royalty free (with the appropriate license).

@end enumerate


@section Installation
The diskette(s) distributed with Dynace are standard DOS formatted
diskette(s).  The files on the diskette(s) have been compressed in order
to reduce the quantity of diskettes. 

See the file @code{README}, located on the first disk, for
installation instructions.



@section Contents
Once the system has been installed there will exist a series of
directories under the @code{dynace} directory.  The system will contain
the following directories:


@table @asis
@item \DYNACE
This is the root of the Dynace system.

@item \DYNACE\LIB
This is the location of all the Dynace libraries.  You may wish to add
this directory to the list of paths your linker searches.

@item \DYNACE\INCLUDE
This is the location of the include files necessary to compile Dynace
applications.  You may need to add this directory to the path your
compiler uses to search include files.

@item \DYNACE\BIN
Executable files necessary for development with the Dynace system.
This directory should be added to your normal search path for executable
programs.

@item \DYNACE\EXAMPLES
Example programs used to learn & demonstrate the Dynace object oriented
extension to C.

@item \DYNACE\WINEXAM
Example programs used to learn & demonstrate WDS.

@item \DYNACE\DOCS
Misc. documentation files.

@item \DYNACE\MANUAL
The Dynace and WDS manuals.

@item \DYNACE\KERNAL
Complete source to the Dynace kernel.

@item \DYNACE\CLASS
Source to all of the Dynace base classes.

@item \DYNACE\THREADS
Complete source to the multi-threader, pipes and semaphores.

@item \DYNACE\GENERICS
Files necessary to build the system generics files from scratch.

@item \DYNACE\DPP
Complete source to the @code{dpp} utility.

@item \DYNACE\WINDOWS
Complete source to the Dynace WDS library.

@item \DYNACE\UTILS
Complete source to the utility programs.

@end table



The only files that are absolutely necessary for a developer are those
located in the @code{\DYNACE\LIB, \DYNACE\INCLUDE} and @code{\DYNACE\BIN}
directories and @code{\DYNACE\UTILS\STARTUP.MK}.





@section Learning The System
This manual contains a description of the WDS concepts, a detailed
description of the WDS system and complete reference to all WDS classes.
It is designed for a programmer who is proficient in the C language, but
has little or no knowledge of Windows programming or Dynace.


In addition to this manual, the accompanying example programs will be
needed in order to learn how to use the system.  The example programs
included with WDS provide a step-by-step tutorial to getting started.


A proficient C programmer with a pre-existing compiler environment
should be able to install this package and feel comfortable enough with
the system to write a simple Windows application in one day.


The best approach to learning WDS would be to start by reading chapters
1 (Introduction) and 2 (Concepts).  Then refer to chapter 3 (Mechanics)
while working through the WDS example programs.

If you wish to dive right in and get a feel for the system you may
perform the setup procedure described in chapter 3 and then go directly
to the example programs.

While working through the examples you may refer to the index (located
in the back of the manual) to locate information on all WDS functions.

Although it is not necessary, you may also learn and incorporate the
many powerful object oriented features of the Dynace object oriented
extension to C, which is what this library is based on.  Dynace is fully
documented in a separate manual.






@section Examples & Tutorial
The WDS system includes a series of examples which serve both as a Quick
Start and a Tutorial.  Theses examples are located under the
@code{\DYNACE\WINEXAM} directory.  The examples located under the
@code{\DYNACE\EXAMPLES} directory are strictly for the Dynace object
oriented extension to C and are not needed to use WDS.

Each example is contained in its entirety in an independent directory.
This is done in order to illustrate the exact files and steps necessary
to create a single application.

The example programs are contained in sub-directories under the
@emph{WINEXAM} directory and are named @emph{EXAMnn} (where the
@emph{nn} is a two digit number).  These numbers are significant in that
they describe the correct order that the examples should be followed.
Each example depends on knowledge built up in previous examples and is
not repeated.

Each example contains a @emph{readme} file which describes information
relating to the purpose of the example and build instructions.  These
files should be read first.  Each example also includes makefiles which
are associated with the compiler vendor being used.  Full source is
included.

Each example program contains source (.c) files.  There is also a mirror
of each source file with a @code{.txt} file extension.  These text files
contain duplicates of the source file with extensive documentation in
the form of comments.  These files should be viewed in order to
understand the code in the application.

Complete setup and build instructions are contained in Chapter 3.

The example programs are tutorial in nature and should be considered the
main source of information used to get started.  The manual augments the
examples with introductory material, concepts and a complete reference.

The following is a list of the enclosed examples:

@table @asis
@item 01
Illustrates the steps necessary to compile and link a bare bones
Windows application.

@item 02
Illustrates the use of menus and message boxes.

@item 03
Illustrates the minimum steps necessary for the creation of a modal dialog.

@item 04
Illustrates the addition of text controls to a dialog.

@item 05
Illustrates the addition of numeric and date controls to a dialog.

@item 06
Illustrates the addition of push button controls to a dialog.

@item 07
Illustrates the addition of radio buttons and check boxes to a dialog.

@item 08
Illustrates the addition of a combo box and list box to a dialog.

@item 09
Illustrates the addition of a scroll bar to a dialog.

@item 10
Illustrates a modal dialog with all control types initialized and used.

@item 11
Illustrates the process of printing to the default printer.

@item 12
Illustrates the process of printing to a user selected and configured printer.

@item 13
Illustrates various methods of outputting text to the printer and
changing pages.

@item 14
Illustrates printer output utilizing different fonts.

@item 15
Illustrates printer graphics output, output scaling, brushes and pens.

@item 16
Illustrates the use of modeless dialogs.

@item 17
Illustrates the use of a modeless dialog which isn't associated with a parent
window.

@item 18
Illustrates the use of the context sensitive help system.

@item 19
Illustrates the creation of Dynace classes in conjunction with a WDS
application.

@c @item 99
@c Illustrates the use of 3rd party custom controls through DLLs.

@end table

See the file @code{\DYNACE\WINEXAM\LIST} for the most up-to-date list
of example programs.


@section Manual Organization
This manual serves as both a user manual and a complete reference manual
to the WDS  system.  The Dynace object oriented extension to C is documented
in a separate manual. 

Chapter 1 (Introduction) covers background material needed to orient
a new user.

Chapter 2 (Concepts) covers fundamental concepts associated with the
Windows environment.  It does this without introducing very much syntax
or other mechanics.

Chapter 3 (Mechanics) documents the exact procedures necessary to setup
and use the WDS system.  The example programs should be viewed subsequent
to setting the system up.

Chapter 4 (Class Reference)  provides a detailed reference to all
classes and methods associated with the class library included
with the WDS system.

The index (located in the back) provides a complete alphabetical listing
of all classes, methods, and macros described in chapter 4.








@section Support, Contact & Upgrades
We will respond to all questions or comments submitted by registered
users (see @code{REGISTER.DOC}).  We will also notify all registered
users of bug fixes and enhancements via e--mail.

In addition, it is our hope to hold a public forum concerning Dynace on
the internet news group comp.lang.misc.  If traffic becomes significant
we'd also like to create a comp.lang.dynace news group.

@table @asis

@c @item Voice
@c 615--791--1636

@item Internet
blake@@mcbride.name

@end table




@section Use, Copyrights & Trademarks
Copyright  @copyright{} 1996 Blake McBride
All rights reserved.

Redistribution and use in source and binary forms, with or without
modification, are permitted provided that the following conditions are
met:

1. Redistributions of source code must retain the above copyright
notice, this list of conditions and the following disclaimer.

2. Redistributions in binary form must reproduce the above copyright
notice, this list of conditions and the following disclaimer in the
documentation and/or other materials provided with the distribution.

THIS SOFTWARE IS PROVIDED BY THE COPYRIGHT HOLDERS AND CONTRIBUTORS
``AS IS'' AND ANY EXPRESS OR IMPLIED WARRANTIES, INCLUDING, BUT NOT
LIMITED TO, THE IMPLIED WARRANTIES OF MERCHANTABILITY AND FITNESS FOR
A PARTICULAR PURPOSE ARE DISCLAIMED. IN NO EVENT SHALL THE COPYRIGHT
HOLDER OR CONTRIBUTORS BE LIABLE FOR ANY DIRECT, INDIRECT, INCIDENTAL,
SPECIAL, EXEMPLARY, OR CONSEQUENTIAL DAMAGES (INCLUDING, BUT NOT
LIMITED TO, PROCUREMENT OF SUBSTITUTE GOODS OR SERVICES; LOSS OF USE,
DATA, OR PROFITS; OR BUSINESS INTERRUPTION) HOWEVER CAUSED AND ON ANY
THEORY OF LIABILITY, WHETHER IN CONTRACT, STRICT LIABILITY, OR TORT
(INCLUDING NEGLIGENCE OR OTHERWISE) ARISING IN ANY WAY OUT OF THE USE
OF THIS SOFTWARE, EVEN IF ADVISED OF THE POSSIBILITY OF SUCH DAMAGE.


@section Credits

The Dynace Object Oriented Extension to C, Windows Development System
and their associated documentation were written by Blake McBride
(blake@@mcbride.name).


