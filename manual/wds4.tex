@c -*-texinfo-*-

@c  Copyright (c) 1996 Blake McBride
@c  All rights reserved.
@c
@c  Redistribution and use in source and binary forms, with or without
@c  modification, are permitted provided that the following conditions are
@c  met:
@c
@c  1. Redistributions of source code must retain the above copyright
@c  notice, this list of conditions and the following disclaimer.
@c
@c  2. Redistributions in binary form must reproduce the above copyright
@c  notice, this list of conditions and the following disclaimer in the
@c  documentation and/or other materials provided with the distribution.
@c
@c  THIS SOFTWARE IS PROVIDED BY THE COPYRIGHT HOLDERS AND CONTRIBUTORS
@c  "AS IS" AND ANY EXPRESS OR IMPLIED WARRANTIES, INCLUDING, BUT NOT
@c  LIMITED TO, THE IMPLIED WARRANTIES OF MERCHANTABILITY AND FITNESS FOR
@c  A PARTICULAR PURPOSE ARE DISCLAIMED. IN NO EVENT SHALL THE COPYRIGHT
@c  HOLDER OR CONTRIBUTORS BE LIABLE FOR ANY DIRECT, INDIRECT, INCIDENTAL,
@c  SPECIAL, EXEMPLARY, OR CONSEQUENTIAL DAMAGES (INCLUDING, BUT NOT
@c  LIMITED TO, PROCUREMENT OF SUBSTITUTE GOODS OR SERVICES; LOSS OF USE,
@c  DATA, OR PROFITS; OR BUSINESS INTERRUPTION) HOWEVER CAUSED AND ON ANY
@c  THEORY OF LIABILITY, WHETHER IN CONTRACT, STRICT LIABILITY, OR TORT
@c  (INCLUDING NEGLIGENCE OR OTHERWISE) ARISING IN ANY WAY OUT OF THE USE
@c  OF THIS SOFTWARE, EVEN IF ADVISED OF THE POSSIBILITY OF SUCH DAMAGE.

@chapter Library Reference

This chapter gives a detailed description of each WDS class.  It is
organized by facility so that in order to look something up, you would
first go to the section which deals with the facility you are interested
in.  And then the functions are in alphabetical order.

There is a naming convention used with Dynace generics.  All generics
start with either a lower case ``g'', ``v'' or ``m'', and are always
followed by an upper case letter.  The ones which start with ``g'' are
normal generics and may be treated like normal C functions.  The ones
which begin with ``v'' use the variable argument facilities of C and
you should, therefore, take a bit extra care when using them since
there is no compile time argument checking being done with these functions.

Since all generics start with either ``g'', ``v'' or ``m'' and in order to
avoid the difficulty associated with grouping all the generics under
three letters, the first letter is dropped for indexing or heading
purposes.  Therefore if you are looking up a generic, it will always
appear in the index or header with its first letter missing.  The syntax
description and example code, however, will show the entire name.

There is one thing about Dynace which you must be aware of, however.
Dynace generic functions (which is what most of these functions are)
allow the same function to be called in many circumstances.  These
generic functions dynamically dispatch to the appropriate procedures
based on the type of the first argument to the generic function (or
``generic'' for short).  So even though you are calling the same
function name, entirely different functionality may occur based on the
type of the first argument.  This fact is seldom, if ever, a problem
since you know whether you are dealing with an icon or a dialog, for
example, at any given point.  Just be sure to look up the generic in
the appropriate section.


@section A Note To Dynace Language Users
In an effort to hide unnecessary details associated with class vs.
instance methods from WDS users who don't care about the differences,
these differences are not discussed or grouped as they are in the
Dynace manual.  You can always distinguish between the two by what
the method takes as its first argument.  If it's a class, then you
have a class method, otherwise it's an instance method.

If you are a WDS user with interest in the Dynace object oriented
extension to C, it is fully documented in its associated manual.




@section Class Hierarchy
This Dynace Windows Development System contains the following class
hierarchy:

@c @example
@c @group
@iftex
@break

@input wdscls1.tex


@end iftex
@c @end group
@c @end example





@section Application
The @code{Application} class is the class used to control application
wide defaults such as fonts, cursors, brushes and so on.  It is also
the class which gets executed by Windows first and executes your
application specific @code{start} function.

There are no instance methods associated with the @code{Application} class.
All access to this class is through class methods via the global class
object @code{Application}.














@deffn {CmdLine} CmdLine::Application
@sp 2
@example
@group
cl = gCmdLine(Application);

char    *cl;    /*  command line   */
@end group
@end example
This method is used to gain access to command line which was used when
the application was launched.
@example
@group
@exdent Example:

int     cl;

cl = gCmdLine(Application);
@end group
@end example
@sp 1
See also:  @code{Instance, Show, PrevInstance}
@end deffn















@deffn {Error} Error::Application
@sp 2
@example
@group
gError(obj, msg);

object  obj;    /*  any object     */
char    *msg;   /*  error message  */
@end group
@end example
This method is used in a severe error condition to issue an error message
to the user and terminate the application.  This method should be avoided
if possible.

Note that this method is actually associated with the Dynace @code{Object}
class which is why the first argument may be any Dynace object.
@example
@group
@exdent Example:

gError(Application, "Error message");
@end group
@end example
@sp 1
See also:  @code{QuitApplication}
@end deffn












@deffn {GetBackBrush} GetBackBrush::Application
@sp 2
@example
@group
bo = gGetBackBrush(Application);

object  bo;     /*  brush object  */
@end group
@end example
This method is used to obtain a copy of the default background brush
object associated with the application.  The returned object will be an
instance of one of the subclasses of the @code{Brush} class.

The object returned must, either explicitly or implicitly, be
disposed when it is no longer needed.  This is normally done
automatically by WDS when it is associated with a window.

See the @code{Brush} class and its subclasses for further details.
@example
@group
@exdent Example:

object  bo;

bo = gGetBackBrush(Application);
@end group
@end example
@sp 1
See also:  @code{GetTextBrush, SetBackBrush}
@end deffn























@deffn {GetCursor} GetCursor::Application
@sp 2
@example
@group
co = gGetCursor(Application);

object  co;     /*  cursor object  */
@end group
@end example
This method is used to obtain a copy of the default cursor object
associated with the application.  The returned object will be an
instance of one of the subclasses of the @code{Cursor} class.

The object returned must, either explicitly or implicitly, be
disposed when it is no longer needed.  This is normally done
automatically by WDS when it is associated with a window.

See the @code{Cursor} class and its subclasses for further details.
@example
@group
@exdent Example:

object  co;

co = gGetCursor(Application);
@end group
@end example
@sp 1
See also:  @code{SetCursor}
@end deffn
















@deffn {GetFont} GetFont::Application
@sp 2
@example
@group
fo = gGetFont(Application);

object  fo;     /*  font object  */
@end group
@end example
This method is used to obtain a copy of the default font object
associated with the application.  The returned object will be an
instance of one of the subclasses of the @code{Font} class.

The object returned must, either explicitly or implicitly, be
disposed when it is no longer needed.  This is normally done
automatically by WDS when it is associated with a window.

See the @code{Font} class and its subclasses for further details.
@example
@group
@exdent Example:

object  fo;

fo = gGetFont(Application);
@end group
@end example
@sp 1
See also:  @code{SetFont}
@end deffn
















@deffn {GetIcon} GetIcon::Application
@sp 2
@example
@group
io = gGetIcon(Application);

object  io;     /*  icon object  */
@end group
@end example
This method is used to obtain a copy of the default icon object
associated with the application.  The returned object will be an
instance of one of the subclasses of the @code{Icon} class.

The object returned must, either explicitly or implicitly, be
disposed when it is no longer needed.  This is normally done
automatically by WDS when it is associated with a window.

See the @code{Icon} class and its subclasses for further details.
@example
@group
@exdent Example:

object  io;

io = gGetIcon(Application);
@end group
@end example
@sp 1
See also:  @code{SetIcon}
@end deffn















@deffn {GetName} GetName::Application
@sp 2
@example
@group
nm = gGetName(Application);

char    *nm;    /*  application name  */
@end group
@end example
This method is used to obtain the global name associated with the application.
@example
@group
@exdent Example:

char    *nm;

nm = gGetName(Application);
@end group
@end example
@sp 1
See also:  @code{SetName}
@end deffn












@deffn {GetTextBrush} GetTextBrush::Application
@sp 2
@example
@group
bo = gGetTextBrush(Application);

object  bo;     /*  brush object  */
@end group
@end example
This method is used to obtain a copy of the default text brush object
associated with the application.  The returned object will be an
instance of one of the subclasses of the @code{Brush} class.

The object returned must, either explicitly or implicitly, be
disposed when it is no longer needed.  This is normally done
automatically by WDS when it is associated with a window.

See the @code{Brush} class and its subclasses for further details.
@example
@group
@exdent Example:

object  bo;

bo = gGetTextBrush(Application);
@end group
@end example
@sp 1
See also:  @code{GetBackBrush, SetTextBrush}
@end deffn















@deffn {GetScalingMode} GetScalingMode::Application
@sp 2
@example
@group
m = gGetScalingMode(Application);

int     m;      /*  scaling mode   */
@end group
@end example
This method is used to get the current scaling mode in affect.
See @code{SetScalingMode} for further details.
@example
@group
@exdent Example:

int     mode;

mode = gGetScalingMode(Application);
@end group
@end example
@sp 1
See also:  @code{SetScalingMode, ScaleToPixels, ScaleToCurrentMode}
@end deffn
















@deffn {GetSize} GetSize::Application
@sp 2
@example
@group
r = gGetSize(Application, vert, horz);

int    *vert;   /*  vertical size    */
int    *horz;   /*  horizontal size  */
object  r;      /*  Application      */
@end group
@end example
This method is used to get the total size of the user's screen.
@code{vert} and @code{horz} are in increments dictated by the
mode selected by @code{SetScalingMode}.  
@example
@group
@exdent Example:

int     y, x;

gGetSize(Application, &y, &x);
@end group
@end example
@sp 1
See also:  @code{SetScalingMode, GetSize::Window}
@end deffn













@deffn {Instance} Instance::Application
@sp 2
@example
@group
ins = gInstance(Application);

HINSTANCE  ins;    /*  instance handle   */
@end group
@end example
This method is used to gain access to the Windows instance handle
associated with the application.  This handle is mainly used internally
by Windows and WDS, and should not normally be needed by a WDS
programmer.
@example
@group
@exdent Example:

HINSTANCE  h;

h = gInstance(Application);
@end group
@end example
@sp 1
See also:  @code{PrevInstance, CmdLine, Show}
@end deffn











@deffn {PrevInstance} PrevInstance::Application
@sp 2
@example
@group
ins = gPrevInstance(Application);

HINSTANCE  ins;    /*  instance handle   */
@end group
@end example
This method is used to gain access to the Windows instance handle
associated with the previous instance of this application, should more
than one be running.  This handle is mainly used internally by Windows
and WDS, and should not normally be needed by a WDS programmer.

This value will always be @code{NULL} under Windows NT.
@example
@group
@exdent Example:

HINSTANCE  h;

h = gPrevInstance(Application);
@end group
@end example
@sp 1
See also:  @code{Instance, CmdLine, Show}
@end deffn









@deffn {QuitApplication} QuitApplication::Application
@sp 2
@example
@group
r = gQuitApplication(Application, ret);

int     ret;    /*  app return value  */
object  r;      /*  Application       */
@end group
@end example
This method is used to terminate an application.  The value passed will
be used as the return value of the application.
@example
@group
@exdent Example:

gQuitApplication(Application, 0);
@end group
@end example
@sp 1
See also:  @code{Error}
@end deffn











@deffn {ScaleToCurrentMode} ScaleToCurrentMode::Application
@sp 2
@example
@group
r = gScaleToCurrentMode(Application, y, x, fnt);

int     *y;     /*  row position         */
int     *x;     /*  column position      */
object  fnt;    /*  current font object  */
object  r;      /*  Application          */
@end group
@end example
This method is used to convert coordinates from pixels to the current
scaling mode.  It is mainly used internally by WDS in order to
convert Window's standard pixel positions into the current scaling mode.

On entry @code{x} and @code{y} point to values which are coordinates in
terms of pixels.  After this method returns, their value will be changed
to be in terms of the current scaling mode (see @code{SetScalingMode}).

@code{fnt} is a font object which is used only if the current scaling
mode is relative to a font.  If so, the font it will be related to
will be the one passed.
@example
@group
@exdent Example:

int     y, x;
object  fnt;

y = some position;
x = some position;
fnt = some font object;
gScaleToCurrentMode(Application, &y, &x, fnt);
@end group
@end example
@sp 1
See also:  @code{SetScalingMode, GetScalingMode, ScaleToPixels}
@end deffn















@deffn {ScaleToPixels} ScaleToPixels::Application
@sp 2
@example
@group
r = gScaleToPixels(Application, y, x, fnt);

int     *y;     /*  row position         */
int     *x;     /*  column position      */
object  fnt;    /*  current font object  */
object  r;      /*  Application          */
@end group
@end example
This method is used to convert coordinates from the current scaling mode
to pixels.  It is mainly used internally by WDS in order to convert
your coordinates into standard pixel positions.

On entry @code{x} and @code{y} point to values which are coordinates
in terms of the current scaling mode (see @code{SetScalingMode}).
After this method returns, their value will be changed to be in terms
of pixels.

@code{fnt} is a font object which is used only if the current scaling
mode is relative to a font.  If so, the font it will be related to
will be the one passed.
@example
@group
@exdent Example:

int     y, x;
object  fnt;

y = some position;
x = some position;
fnt = some font object;
gScaleToPixels(Application, &y, &x, fnt);
@end group
@end example
@sp 1
See also:  @code{SetScalingMode, GetScalingMode, ScaleToCurrentMode}
@end deffn
















@deffn {SetBackBrush} SetBackBrush::Application
@sp 2
@example
@group
r = gSetBackBrush(Application, bo);

object  bo;     /*  brush object         */
object  r;      /*  brush object passed  */
@end group
@end example
This method is used to set the application wide default background
brush.  This is the brush used to color everything except the text that
appears on windows and dialogs.  @code{bo} must be an instance of one of
the subclasses of @code{Brush}.  All windows and dialogs will use the
application wide default brush which is in effect when they are created
unless a specific brush object is specified for a particular window.

When a new default brush object is set, any previous default object
will be disposed.  If no default is set, WDS uses the system brush
identified as @code{COLOR_WINDOW}.

See the @code{Brush} class and its subclasses for further details.
@example
@group
@exdent Example:

gSetBackBrush(Application, vNew(SystemBrush, COLOR_WINDOW));
@end group
@end example
@sp 1
See also:  @code{GetBackBrush, SetTextBrush}
@end deffn












@deffn {SetCursor} SetCursor::Application
@sp 2
@example
@group
r = gSetCursor(Application, co);

object  co;     /*  cursor object         */
object  r;      /*  cursor object passed  */
@end group
@end example
This method is used to set the application wide default cursor.
@code{co} must be an instance of one of the subclasses of @code{Cursor}.
All windows will use the application wide default cursor which is in
effect when they are created unless a specific cursor object is
specified for a particular window.

When a new default cursor object is set, any previous default object
will be disposed.  If no default is set, WDS uses the system cursor
identified as @code{IDC_ARROW}.

See the @code{Cursor} class and its subclasses for further details.
@example
@group
@exdent Example:

gSetCursor(Application, gLoadSys(SystemCursor, IDC_ARROW));
@end group
@end example
@sp 1
See also:  @code{GetCursor, LoadSys::SystemCursor, Load::ExternalCursor}
@end deffn










@deffn {SetFont} SetFont::Application
@sp 2
@example
@group
r = gSetFont(Application, fo);

object  fo;     /*  font object         */
object  r;      /*  font object passed  */
@end group
@end example
This method is used to set the application wide default font.
@code{fo} must be an instance of one of the subclasses of @code{Font}.
All windows will use the application wide default font which is in
effect when they are created unless a specific font object is
specified for a particular window.

When a new default font object is set, any previous default object
will be disposed.  If no default is set, WDS uses the system font
identified as @code{SYSTEM_FONT}.

See the @code{Font} class and its subclasses for further details.
@example
@group
@exdent Example:

gSetFont(Application, vNew(SystemFont, SYSTEM_FONT));
@end group
@end example
@sp 1
See also:  @code{GetFont, Load::SystemFont, New::ExternalFont}
@end deffn
















@deffn {SetIcon} SetIcon::Application
@sp 2
@example
@group
r = gSetIcon(Application, io);

object  io;     /*  icon object         */
object  r;      /*  icon object passed  */
@end group
@end example
This method is used to set the application wide default icon.  An icon
associated to a window is the one which is displayed when the window is
iconized.  @code{io} must be an instance of one of the subclasses of
@code{Icon}.  All windows will use the application wide default icon
which is in effect when they are created unless a specific icon object
is specified for a particular window.

When a new default icon object is set, any previous default object
will be disposed.  If no default is set, WDS uses the system icon
identified as @code{IDI_APPLICATION}.

See the @code{Icon} class and its subclasses for further details.
@example
@group
@exdent Example:

gSetIcon(Application, gLoadSys(SystemIcon, IDI_APPLICATION));
@end group
@end example
@sp 1
See also:  @code{GetIcon, LoadSys::SystemIcon, Load::ExternalIcon}
@end deffn












@deffn {SetName} SetName::Application
@sp 2
@example
@group
gSetName(Application, nm);

char    *nm;    /*  application name  */
@end group
@end example
This method is used to give the application a globally accessible name.
This name is accessible via @code{GetName}.  There is no other use
made of this information.
@example
@group
@exdent Example:

gSetName(Application, "My App");
@end group
@end example
@sp 1
See also:  @code{GetName}
@end deffn















@deffn {SetScalingMode} SetScalingMode::Application
@sp 2
@example
@group
r = gSetScalingMode(Application, m);

int     m;      /*  scaling mode   */
int     r;      /*  previous mode  */
@end group
@end example
This method is used to set the scaling mode used by all other WDS methods
which take coordinate positions.  Valid modes are as follows:
@table @code
@item SM_1_PER_CHAR
This mode causes each position to be in increments determined by the
size of the current font.  For example row 7 would mean 7 times the
height of the current font.  Similar to line positions.
@item SM_10_PER_SYSCHAR
This mode causes each position to be in increments determined by one
tenth the size of the system font declared globally to the user's
Windows environment.  For example row 70 would mean 7 times the height
of the Windows global system font.  This allows positioning relative
to a scaling factor determined by the user.
@item SM_PIXELS
This mode performs no conversion.  The application is able to use
pixel coordinated directly.
@end table

The default value is @code{SM_1_PER_CHAR}.

The value returned is the mode which was previously set.
@example
@group
@exdent Example:

gSetScalingMode(Application, SM_PIXELS);
@end group
@end example
@sp 1
See also:  @code{GetScalingMode, ScaleToPixels, ScaleToCurrentMode}
@end deffn














@deffn {SetTextBrush} SetTextBrush::Application
@sp 2
@example
@group
r = gSetTextBrush(Application, bo);

object  bo;     /*  brush object         */
object  r;      /*  brush object passed  */
@end group
@end example
This method is used to set the application wide default text brush.
This is the brush used by all text output (or foreground) to windows or
dialogs.  @code{bo} must be an instance of one of the subclasses of
@code{Brush}.  All windows and dialogs will use the application wide
default brush which is in effect when they are created unless a specific
brush object is specified for a particular window.

When a new default brush object is set, any previous default object
will be disposed.  If no default is set, WDS uses the system brush
identified as @code{COLOR_WINDOWTEXT}.

See the @code{Brush} class and its subclasses for further details.
@example
@group
@exdent Example:

gSetTextBrush(Application, vNew(SystemBrush, COLOR_WINDOWTEXT));
@end group
@end example
@sp 1
See also:  @code{GetTextBrush, SetBackBrush}
@end deffn










@deffn {Show} Show::Application
@sp 2
@example
@group
sv = gShow(Application);

int     sv;     /*  show value   */
@end group
@end example
This method is used to gain access to the show value supplied by Windows
when an application starts.  It is normally used to determine how an
initial application's main window should be shown.  The available
options are Windows macros which begin with @code{SW_} and are fully
documented by the Windows documentation under the @code{WinMain}
function.
@example
@group
@exdent Example:

int     sv;

sv = gShow(Application);
@end group
@end example
@sp 1
See also:  @code{Instance, CmdLine, PrevInstance}
@end deffn









@section Windows
This section documents the @code{Window} class which contains all the
functionality which is common to Main, Child, and Popup windows.  See
chapter 2 of this manual for a description of the different window
types.






@deffn {AddHandlerAfter} AddHandlerAfter::Window
@sp 2
@example
@group
r = gAddHandlerAfter(wind, msg, func);

object   wind;     /*  a window object  */
unsigned msg;      /*  message          */
long    (*func)(); /*  function pointer */
object  r;         /*  the window obj   */
@end group
@end example
This method is used to associate function @code{func} with Windows window
message @code{msg} for window @code{wind}.  Whenever window @code{wind}
receives message @code{msg}, @code{func} will be called.

@code{wind} is the window object who's messages you wish to process.
@code{msg} is the particular message you wish to trap.  These messages
are fully documented in the Windows documentation in the Messages
section.  They normally begin with @code{WM_}.

@code{func} is the function which gets called whenever the specified
message gets received and takes the following form:
@example
@group
long    func(object     wind,
             HWND       hwnd, 
             UINT       mMsg, 
             WPARAM     wParam, 
             LPARAM     lParam)
@{
        .
        .
        .
        return 0L;  /* or whatever is appropriate  */
@}
@end group
@end example
Where @code{wind} is the window being sent the message.  The remaining
arguments and return value is fully documented in the Windows documentation
under the @code{WindowProc} function and the Windows Messages documentation.

WDS keeps a list of functions associated with each message associated
with each window.  When a particular message is received the appropriate
list of handler functions gets executed sequentially.
@code{AddHandlerAfter} appends the new function to the end of this list,
and @code{AddHandlerBefore} adds the new function to the beginning of
the list.

WDS may also, and optionally, execute the Windows default procedure
associated with a given message either before or after the user added
list of functions.  This behavior may be controlled via
@code{DefaultProcessingMode}.

Windows will only see the return value of the last message handler executed
including, if applicable, the default.
@example
@group
@exdent Example:

int     hSize, vSize;

static  long    process_wm_size(object  wind, 
                                HWND    hwnd, 
                                UINT    mMsg, 
                                WPARAM  wParam, 
                                LPARAM  lParam)
@{
        hSize = LOWORD(lParam);
        vSize = HIWORD(lParam);
        return 0L;
@}

        .
        .
        gAddHandlerAfter(wind, (unsigned) WM_SIZE, process_wm_size);
        .
        .
@end group
@end example
@sp 1
See also:  @code{DefaultProcessingMode, AddHandlerBefore}
@end deffn






@deffn {AddHandlerBefore} AddHandlerBefore::Window
@sp 2
@example
@group
r = gAddHandlerBefore(wind, msg, func);

object   wind;     /*  a window object  */
unsigned msg;      /*  message          */
long    (*func)(); /*  function pointer */
object  r;         /*  the window obj   */
@end group
@end example
This function is fully documented under @code{AddHandlerAfter}.
@c @example
@c @group
@c @exdent Example:
@c @end group
@c @end example
@sp 1
See also:  @code{AddHandlerAfter}
@end deffn







@deffn {Associate} Associate::Window
@sp 2
@example
@group
r = mAssociate(wind, itm, fun);

object  wind;   /*  a window object     */
int     itm;    /*  menu item           */
long    (*fun)();  /*  function         */
object   r;        /*  the menu         */
@end group
@end example
This method is used to associate an application specific function
(@code{fun}) with a menu item (@code{itm}) associated with the current
menu attached to window @code{wind}.  Once this is done, if the user selects
the menu option identified by @code{itm}, then the function @code{fun}
will be executed.

@code{itm} is a programmer defined macro which identifies one particular
choice among those available within the menu which is currently attached
to window @code{wind}.  This macro is defined while the programmer defines
the entire menu using the resource editor.

@code{fun} is the function which will be executed when the user selects
menu option @code{itm} and has the following form:
@example
@group
long    fun(object wind, unsigned id)
@{
        .
        .
        .
        return 0L;
@}
@end group
@end example
The function executed (@code{fun}) is passed the window object and
the specific menu id which the user selected and
returns a long.  The return value is documented in the Windows documentation
under the message named @code{WM_COMMAND}.  It should normally be @code{0L}.
@example
@group
@exdent Example:

static  long    file_message(object wind)
@{
        gMessage(wind, "File_Message");
        return 0L;
@}

        .
        .
        mLoadMenu(win, IDR_MENU1);
        mAssociate(win, ID_FILE_MESSAGE, file_message);
@end group
@end example
@sp 1
See also:  @code{LoadMenu, MenuItemMode}
@end deffn











@deffn {AutoDispose} AutoDispose::Window
@sp 2
@example
@group
r = gAutoDispose(wind, mode);

object   wind;     /*  a window object    */
int      mode;     /*  auto dispose mode  */
int      r;        /*  previous value     */
@end group
@end example
This method is used to enable or disable the auto dispose mechanism
associated with a window.  This feature, when enabled, causes WDS to
automatically dispose of the WDS object associated with a window
whenever the user closes the window (@code{wind}).  Normally, when
this feature is disabled, if the user closes a window, WDS removes
the window from the display, but the WDS object remains intact until
the program manually disposes of the window object (via @code{Dispose}).
This feature is most commonly used with asynchronous, popup windows.

Note that it is invalid to attempt to use a WDS object subsequent to it
being disposed, and each unneeded WDS object must be deleted.

@code{mode} is set to @code{1} to enable the feature and @code{0}
to disable it.  The value returned is the prior mode.
@example
@group
@exdent Example:

gAutoDispose(wind, 1);
@end group
@end example
@sp 1
See also:  @code{Dispose}
@end deffn











@deffn {AutoShow} AutoShow::Window
@sp 2
@example
@group
r = gAutoShow(wind, mode);

object   wind;     /*  a window object    */
int      mode;     /*  auto show mode     */
int      r;        /*  previous value     */
@end group
@end example
This method is used to enable or disable the auto show mechanism
associated with a window.  This feature, when enabled, causes WDS to
automatically display (execute @code{Show}) on a window which is written
to.  This enables the programmer to create a window which will automatically
popup the first time the program displays text to it.

@code{mode} is set to @code{1} to enable the feature and @code{0}
to disable it.  The value returned is the prior mode.
@example
@group
@exdent Example:

gAutoShow(wind, 1);
@end group
@end example
@sp 1
See also:  @code{Show}
@end deffn











@deffn {BackBrush} BackBrush::Window
@sp 2
@example
@group
r = gBackBrush(wind, brsh);

object  wind;   /*  a window object   */
object  brsh;   /*  brush object      */
object  r;      /*  wind              */
@end group
@end example
This method is used to determine what brush object is used for the
background of window @code{wind} and performs the same function as the
@code{Use} method when used with a @code{Brush} object.  Any previously
associated brush object will be disposed.  This brush object will also
be automatically disposed when the window is disposed.

The window passed is returned.
@example
@group
@exdent Example:

object  myWind;

gBackBrush(myWind, vNew(SolidBrush, 0, 0, 255));
@end group
@end example
@sp 1
See also:  @code{TextBrush, Use, SetTextBrush::Application}
        and the @code{Brush} classes
@end deffn












@deffn {DefaultProcessingMode} DefaultProcessingMode::Window
@sp 2
@example
@group
r = gDefaultProcessingMode(wind, msg, mode);

object   wind;     /*  a window object          */
unsigned msg;      /*  message                  */
int      mode;     /*  default processing mode  */
object   r;        /*  the window obj           */
@end group
@end example
This method is used to determine when or if the Windows default message
procedure is processed for a given message (@code{msg}) associated with
a particular window (@code{wind}).

WDS allows a programmer to specify an arbitrary number of functions
to be executed whenever a window receives a specific message (via
@code{AddHandlerAfter} and @code{AddHandlerBefore}).  Windows has
default procedures associated with many window messages.  At times
it is necessary to replace or augment this default functionality.
@code{DefaultProcessingMode} gives the programmer control over when and
if this default Windows functionality.  @code{mode} is used to specify
the desired mode.  The following table indicates the valid modes:

@table @code
@item 0
Do not execute the Windows default processing
@item 1
Execute default processing @emph{after} programmer defined handlers
@item 2
Execute default processing @emph{before} programmer defined handlers
@end table

Note that the default mode is always @code{1}, and must be explicitly
changed, if desired, for each message associated with each window.

@code{msg} is the particular message you wish to affect.  These messages
are fully documented in the Windows documentation in the Messages
section.  They normally begin with @code{WM_}.
@example
@group
@exdent Example:

gDefaultProcessingMode(wind, (unsigned) WM_SIZE, 0);
@end group
@end example
@sp 1
See also:  @code{AddHandlerAfter}
@end deffn








@deffn {Dispose} Dispose::Window
@sp 2
@example
@group
r = gDispose(wind);

object  wind;   /*  a window object    */
object  r;      /*  NULL               */
@end group
@end example
This method is used to remove and dispose of a window object when it
is no longer needed.  This method should be called on all windows
when they are no longer needed.

The value returned is always @code{NULL} and may be used to null out
the variable which contained the object being disposed in order to
avoid future accidental use.
@example
@group
@exdent Example:

object  myWind;

myWind = gDispose(myWind);
@end group
@end example
@sp 1
See also:  @code{AutoDispose, New::ChildWindow, New::PopupWindow}
@end deffn










@deffn {Erase} Erase::Window
@sp 2
@example
@group
r = gErase(wind, brow, erow, bcol, ecol);

object  wind;   /*  a window object     */
int     brow;   /*  beginning row       */
int     erow;   /*  ending row          */
int     bcol;   /*  beginning column    */
int     ecol;   /*  ending column       */
object  r;      /*  the window object   */
@end group
@end example
This method is used to delete all text vectors which intersect the
rectangle defined by the parameters.  These parameters are adjusted
according to the current scaling mode (set with
@code{SetScalingMode::Application}.  All coordinates are measured from
the upper left hand corner of the window and start with @code{0,0}.

The object returned is the window object passed.
@example
@group
@exdent Example:

gErase(win, 3, 10, 0, 200);
@end group
@end example
@sp 1
See also:  @code{EraseLines, EraseAll}
@end deffn










@deffn {EraseAll} EraseAll::Window
@sp 2
@example
@group
r = gEraseAll(wind);

object  wind;   /*  a window object     */
object  r;      /*  the window object   */
@end group
@end example
This method is used to erase all text lines associated with a window.

The object returned is the window object passed.
@example
@group
@exdent Example:

gEraseAll(win);
@end group
@end example
@sp 1
See also:  @code{EraseLines, Erase}
@end deffn










@deffn {EraseLines} EraseLines::Window
@sp 2
@example
@group
r = gEraseLines(wind, blin, elin);

object  wind;   /*  a window object     */
int     blin;   /*  beginning line      */
int     elin;   /*  ending line         */
object  r;      /*  the window object   */
@end group
@end example
This method is used to delete all text lines from a starting line number
(@code{blin}) to an ending line number (@code{elin}). All lines are
measured from the upper left hand corner of the window and start with
@code{0,0}.  Lines positions are calculated based on the current font.

The object returned is the window object passed.
@example
@group
@exdent Example:

gEraseLines(win, 3, 10);
@end group
@end example
@sp 1
See also:  @code{Erase, EraseAll}
@end deffn









@deffn {Getch} Getch::Window
@sp 2
@example
@group
ch = gGetch(wind);

object  wind;   /*  a window object   */
int     ch;     /*  character read    */
@end group
@end example
This method is used to obtain the next input character struck by the
user.  The @code{SetBlock} method may be used to determine whether or not
@code{Getch} waits if a character is not available.  The value returned
is the character struck by the user or 0 if no character was available
and the input was non-blocking.
@example
@group
@exdent Example:

object  myWind;
int     ch;

ch = gGetch(myWind);
@end group
@end example
@sp 1
See also:  @code{Kbhit, SetBlock, Gets}
@end deffn
















@deffn {GetName} GetName::Window
@sp 2
@example
@group
r = gGetName(w);

object  w;      /*  a window object     */
char   *r;      /*  the name associated with the window   */
@end group
@end example
This method is used to get the name associated with window @code{w}.
The name associated with a window is what is displayed at the top of the
window, if it has a title bar. 
@example
@group
@exdent Example:

object  myWind;
char    *n;

myWind = vNew(MainWindow, "App Name");
n = gGetName(myWind);   /*  n = "App Name"  */
@end group
@end example
@sp 1
See also:  @code{New::MainWindow, SetName}
@end deffn














@deffn {GetParent} GetParent::Window
@sp 2
@example
@group
prnt = gGetParent(wind);

object  wind;   /*  child window object   */
object  prnt;   /*  parent window object  */
@end group
@end example
This method is used to obtain the parent window object associated
with window @code{wind}.
@example
@group
@exdent Example:

object  myWind, parentWind;

parentWind = gGetParent(myWind);
@end group
@end example
@sp 1
See also:  @code{SetParent, New::ChildWindow}
@end deffn















@deffn {GetPosition} GetPosition::Window
@sp 2
@example
@group
r = gGetPosition(wind, vert, horz);

object  wind;   /*  a window object     */
int    *vert;   /*  vertical position   */
int    *horz;   /*  horizontal position */
object  r;      /*  wind arg passed     */
@end group
@end example
This method is used to get the current position of a window.
@code{vert} and @code{horz} are in increments dictated by the
mode selected by @code{SetScalingMode::Application}.  The window
object passed is returned.
@example
@group
@exdent Example:

int     y, x;

gGetPosition(myWind, &y, &x);
@end group
@end example
@sp 1
See also:  @code{SetScalingMode::Application, GetSize, SetPosition}
@end deffn







@deffn {Gets} Gets::Window
@sp 2
@example
@group
r = gGets(wind, buf, len);

object   wind;  /*  a window object       */
char    *buf;   /*  input buffer          */
unsigned len;   /*  length of buffer      */
char    *r;     /*  buf                   */
@end group
@end example
This method is used to read (accept) a single line of text from a user
until a return is hit or @code{len}-1 key strokes have been entered
and place them in @code{buf}.

This method returns a pointer to the buffer passed unless there is an
error, in which case @code{NULL} is returned.
The input accepted will a return terminated line entered by the user
up to @code{len}-1 characters.  However, the number of characters
may be less if non-blocking
io is selected (via @code{SetBlock}) and there aren't enough characters
available.  @code{SetRaw} may also be used to control whether backspace
processing will be performed.
@example
@group
@exdent Example:

object  myWind;
char    buf[80];

gGets(myWind, buf, sizeof(buf));
@end group
@end example
@sp 1
See also:  @code{SetBlock, SetRaw, Gets, Getch}
@end deffn












@deffn {GetSize} GetSize::Window
@sp 2
@example
@group
r = gGetSize(wind, vert, horz);

object  wind;   /*  a window object     */
int    *vert;   /*  vertical size       */
int    *horz;   /*  horizontal size     */
object  r;      /*  wind arg passed     */
@end group
@end example
This method is used to get the current size of a window.
@code{vert} and @code{horz} are in increments dictated by the
mode selected by @code{SetScalingMode::Application}.  The window
object passed is returned.
@example
@group
@exdent Example:

int     y, x;

gGetSize(myWind, &y, &x);
@end group
@end example
@sp 1
See also:  @code{SetScalingMode::Application, GetPosition, SetSize,}
@iftex
@hfil @break @hglue .63in 
@end iftex
@code{GetSize::Application}
@end deffn










@deffn {GetTag} GetTag::Window
@sp 2
@example
@group
r = gGetTag(wind);

object  wind;   /*  a window object   */
object  r;      /*  tag               */
@end group
@end example
This method is used to obtain a Dynace object which has been associated
with a window via @code{SetTag}.  The value return is the object which has
been associated with the window object @code{wind}.  If there is no object
associated with the window, @code{NULL} will be returned.
@example
@group
@exdent Example:

object  myWind, someObj;

someObj = gGetTag(myWind);
@end group
@end example
@sp 1
See also:  @code{SetTag, SetTag::Dialog}
@end deffn












@deffn {Handle} Handle::Window
@sp 2
@example
@group
h = gHandle(wind);

object  wind;   /*  window object  */
HANDLE  h;      /*  Windows handle */
@end group
@end example
This method is used to obtain the Windows internal handle associated with
a window object.  Note that this will be @code{0} prior to executing
@code{Show} on the window because that is the point where Windows creates
the handle.

Note that this method may be used with most WDS objects in order to obtain
the internal handle that Windows normally associates with each type of object.
See the appropriate documentation.
@example
@group
@exdent Example:

object  myWind;
HANDLE  h;

h = gHandle(myWind);
@end group
@end example
@c @sp 1
@c See also:  @code{Message}
@end deffn






@deffn {Kbhit} Kbhit::Window
@sp 2
@example
@group
r = gKbhit(wind);

object  wind;   /*  a window object   */
int     r;      /*  characters ready  */
@end group
@end example
This method is used to determine if and how many characters are waiting
in the character input queue.  This method is used in an attempt to
provide a familiar character based user input mechanism.

The value returned is the number of characters ready for input or
zero if none are available.
@example
@group
@exdent Example:

object  myWind;
int     n;

n = gKbhit(myWind);
@end group
@end example
@sp 1
See also:  @code{Getch, SetBlock, Gets}
@end deffn



















@deffn {LoadCursor} LoadCursor::Window
@sp 2
@example
@group
r = mLoadCursor(wind, csr);

object   wind;  /*  a window object    */
unsigned csr;   /*  cursor identifier  */
object   r;     /*  cursor object      */
@end group
@end example
This method is used to load a programmer defined cursor and associate it
with window @code{wind}.  The cursor would then be displayed any time
the pointer was placed in the window.

@code{csr} is a programmer defined unsigned integer which identifies
the cursor.  This identifier is normally a macro and defined through the
resource editor.  The cursor object will be automatically destroyed
whenever the window is destroyed or if a new cursor is loaded.

The value returned is an object representing the cursor loaded, or
@code{NULL} if the cursor was not found.
@example
@group
@exdent Example:

object  myWind;

mLoadCursor(myWind, MY_CURSOR);
@end group
@end example
@sp 1
See also:  @code{LoadSystemCursor, Load::ExternalCursor, Use}
@end deffn











@deffn {LoadFont} LoadFont::Window
@sp 2
@example
@group
r = gLoadFont(wind, fname, sz);

object   wind;  /*  a window object  */
char    *fname; /*  font name        */
int      sz;    /*  point size       */
object   r;     /*  font object      */
@end group
@end example
This method is used to load an arbitrary font by name at any point size
and associate it as the default font for future text output to window
@code{wind}.  @code{fname} is the full name of the font as it appears
when you list the available fonts via the control-panel / fonts Windows
utility, minus the font type in parentheses.  @code{sz} indicates the
desired point size.

All font objects associated with a window will be automatically
destroyed whenever the window is destroyed.

The value returned is an object representing the font loaded, or
@code{NULL} if the font was not found.

Note that @code{fname} may also be a Dynace object cast as a (@code{char *}).
@example
@group
@exdent Example:

object  myWind;

gLoadFont(myWind, "Times New Roman", 12);
@end group
@end example
@sp 1
See also:  @code{LoadSystemFont, Indirect::ExternalFont, Use}
@end deffn









@deffn {LoadIcon} LoadIcon::Window
@sp 2
@example
@group
r = mLoadIcon(wind, icn);

object   wind;  /*  a window object  */
unsigned icn;   /*  icon identifier  */
object   r;     /*  icon object      */
@end group
@end example
This method is used to load a programmer defined icon and associate it
with window @code{wind}.  The icon would then be displayed if window
@code{wind} was iconized.

@code{icn} is a programmer defined unsigned integer which identifies
the icon.  This identifier is normally a macro and defined through the
resource editor.  The icon object will be automatically destroyed
whenever the window is destroyed or if a new icon is loaded.

The value returned is an object representing the icon loaded, or
@code{NULL} if the icon was not found.
@example
@group
@exdent Example:

object  myWind;

mLoadIcon(myWind, ALGOCORP_ICON);
@end group
@end example
@sp 1
See also:  @code{LoadSystemIcon, Load::ExternalIcon, Use}
@end deffn








@deffn {LoadMenu} LoadMenu::Window
@sp 2
@example
@group
r = mLoadMenu(wind, mnu);

object   wind;  /*  a window object  */
unsigned mnu;   /*  menu identifier  */
object   r;     /*  menu object      */
@end group
@end example
This method is used to load a programmer defined menu and associate it
with window @code{wind}.  The menu would be displayed at the top
of the window.

@code{mnu} is a programmer defined unsigned integer which identifies the
menu.  This identifier is normally a macro and defined through the
resource editor.  Any menu previously associated with the window is
pushed on a stack so that previous menus may be easily returned to in a
last-in-first-out basis using the @code{PopMenu} method.  All menus
associated with a window will be destroyed when the window is destroyed.

The value returned is an object representing the menu loaded, or
@code{NULL} if the menu was not found.
@example
@group
@exdent Example:

object  myWind;

mLoadMenu(myWind, MY_MENU);
@end group
@end example
@sp 1
See also:  @code{Associate, PopMenu, LoadMenuStr, Load::ExternalMenu, Use}
@end deffn














@deffn {LoadMenuStr} LoadMenuStr::Window
@sp 2
@example
@group
r = gLoadMenuStr(wind, mnu);

object   wind;  /*  a window object  */
char    *mnu;   /*  menu identifier  */
object   r;     /*  menu object      */
@end group
@end example
This method is used to load a programmer defined menu and associate it
with window @code{wind}.  The menu would be displayed at the top
of the window.

@code{mnu} is a programmer defined name which identifies the menu.  This
name is defined through the resource editor.  Any menu previously associated
with the window is pushed on a stack so that previous menus may be easily
returned to in a last-in-first-out basis using the @code{PopMenu} method.
All menus associated with a window will be destroyed when the window
is destroyed.

Note that the @code{LoadMenu} method is used more frequently since the
resource editors assign macros to each menu name.

The value returned is an object representing the menu loaded, or
@code{NULL} if the menu was not found.
@example
@group
@exdent Example:

object  myWind;

gLoadMenuStr(myWind, "mymenu");
@end group
@end example
@sp 1
See also:  @code{Associate, LoadMenu, LoadStr::ExternalMenu, Use}
@end deffn












@deffn {LoadSystemCursor} LoadSystemCursor::Window
@sp 2
@example
@group
r = gLoadSystemCursor(wind, csr);

object   wind;  /*  a window object    */
LPCSTR   csr;   /*  cursor identifier  */
object   r;     /*  cursor object      */
@end group
@end example
This method is used to load a Windows predefined cursor and associate it
with window @code{wind}.  The cursor would then be displayed if the pointer
was positioned over the window.

@code{csr} is a Windows defined macro which identifies the cursor.  The
available options are defined in the Windows documentation under the
function named @code{LoadCursor} and normally begin with @code{IDC_}.  The
cursor object will be automatically destroyed whenever the window is
destroyed or if a new cursor is loaded.

The value returned is an object representing the cursor loaded, or
@code{NULL} if the cursor was not found.
@example
@group
@exdent Example:

object  myWind;

gLoadSystemCursor(myWind, IDC_CROSS);
@end group
@end example
@sp 1
See also:  @code{LoadCursor, LoadSys::SystemCursor, Use}
@end deffn











@deffn {LoadSystemFont} LoadSystemFont::Window
@sp 2
@example
@group
r = gLoadSystemFont(wind, fnt);

object   wind;  /*  a window object  */
unsigned fnt;   /*  font identifier  */
object   r;     /*  font object      */
@end group
@end example
This method is used to load a Windows predefined font and associate it
with window @code{wind}.  The last font associated with a window is the
one which will be used when any text is output to the window.

@code{fnt} is a Windows defined macro which identifies the font.  The
available options are defined in the Windows documentation under the
function named @code{GetStockObject} and normally end with @code{_FONT}.
The font object will be automatically destroyed whenever the window is
destroyed.

The value returned is an object representing the font, or
@code{NULL} if the font was not found.
@example
@group
@exdent Example:

object  myWind;

gLoadSystemFont(myWind, SYSTEM_FONT);
@end group
@end example
@sp 1
See also:  @code{LoadFont, Load::SystemFont, Use}
@end deffn
















@deffn {LoadSystemIcon} LoadSystemIcon::Window
@sp 2
@example
@group
r = gLoadSystemIcon(wind, icn);

object   wind;  /*  a window object  */
LPCSTR   icn;   /*  icon identifier  */
object   r;     /*  icon object      */
@end group
@end example
This method is used to load a Windows predefined icon and associate it
with window @code{wind}.  The icon would then be displayed if window
@code{wind} was iconized.

@code{icn} is a Windows defined macro which identifies the icon.  The
available options are defined in the Windows documentation under the
function named @code{LoadIcon} and normally begin with @code{IDI_}.  The
icon object will be automatically destroyed whenever the window is
destroyed or if a new icon is loaded.

The value returned is an object representing the icon loaded, or
@code{NULL} if the icon was not found.
@example
@group
@exdent Example:

object  myWind;

gLoadSystemIcon(myWind, IDI_APPLICATION);
@end group
@end example
@sp 1
See also:  @code{LoadIcon, LoadSys::SystemIcon, Use}
@end deffn














@deffn {MenuItemMode} MenuItemMode::Window
@sp 2
@example
@group
r = mMenuItemMode(wind, itm, mod);

object  wind;   /*  a window object     */
unsigned itm;   /*  menu item           */
unsigned mod;   /*  menu item mode      */
object   r;        /*  the menu         */
@end group
@end example
This method is used to set the mode associated with a particular item
in the menu which is currently attached to window @code{wind}.

@code{itm} is a programmer defined macro which identifies one particular
choice among those available within the menu which is currently attached
to window @code{wind}.  This macro is defined while the programmer defines
the entire menu using the resource editor.

@code{mod} may be one of @code{MF_DISABLED}, @code{MF_ENABLED} or
@code{MF_GRAYED} and is documented in the Windows documentation under
the function @code{EnableMenuItem}.

The menu object associated with the window is returned.
@example
@group
@exdent Example:

mMenuItemMode(win, ID_FILE_MESSAGE, MF_GRAYED);
@end group
@end example
@sp 1
See also:  @code{LoadMenu, Associate}
@end deffn











@deffn {Message} Message::Window
@sp 2
@example
@group
r = gMessage(wind, msg);

object  wind;   /*  window object  */
char    *msg;   /*  message        */
object  r;      /*  window object  */
@end group
@end example
This method is used to open up a temporary informational window.  The
window will contain the message given by @code{msg} and the user must
acknowledge the window prior to continuing by hitting an OK button.


The value returned is the window passed.
@example
@group
@exdent Example:

object  myWind;

gMessage(myWind, "Press OK to continue.");
@end group
@end example
@sp 1
See also:  @code{MessageWithTopic}
@end deffn












@deffn {MessageWithTopic} MessageWithTopic::Window
@sp 2
@example
@group
r = gMessageWithTopic(wind, msg, tpc);

object  wind;   /*  window object  */
char    *msg;   /*  message        */
char    *tpc;   /*  help topic     */
object  r;      /*  window object  */
@end group
@end example
This method is used to open up a temporary informational window.  The
window will contain the message given by @code{msg} and the user must
acknowledge the window prior to continuing by hitting an OK button.

If the user hits the @code{F1} key while presented with the message,
the help topic identified by @code{tpc} will get displayed via the Windows
help system.

The value returned is the window passed.
@example
@group
@exdent Example:

object  myWind;

gMessageWithTopic(myWind, "Press OK to continue.", "mytopic");
@end group
@end example
@sp 1
See also:  @code{Message} and the @code{HelpSystem} class.
@end deffn











@deffn {New} New::Window
@sp 2
@example
@group
r = vNew(Window);

object  r;      /*  new window  */
@end group
@end example
This class method is used to create a basic window object.  It is used
by all the subclasses of @code{Window} and would not normally be
used by a programmer.

The value returned is the new window created.
@example
@group
@exdent Example:

object  myWind;

myWind = vNew(Window);
@end group
@end example
@sp 1
See also:  @code{New::MainWindow, New::ChildWindow, New::PopupWindow, Show}
@iftex
@hfil @break @hglue .63in   
@end iftex
@code{Dispose}
@end deffn







@deffn {NewBuiltIn} NewBuiltIn::Window
@sp 2
@example
@group
r = gNewBuiltIn(Window, class, parent);

char    *class; /*  Windows built in class designation  */
object  parent; /*  parent window object                */
object  r;      /*  new child window                    */
@end group
@end example
This class method is used to create a child window which is used as a
Windows-built-in control.  @code{Window} represents the @code{Window}
class and is typed in as shown.  @code{class} is a string representing
one of the window classes build into Windows.  This string is defined
and documented by Windows under the function named @code{CreateWindow}.
@code{parent} represents the parent window object and must be specified.

The value returned is the new child window (control) created.
@example
@group
@exdent Example:

object  myWind, ctl;

myWind = vNew(MainWindow, "App Name");
ctl = gNewBuiltIn(Window, "button", myWind);
@end group
@end example
@sp 1
See also:  @code{New::ButtonWindow}
@end deffn












@deffn {PopMenu} PopMenu::Window
@sp 2
@example
@group
r = gPopMenu(wind);

object   wind;  /*  a window object  */
object   r;     /*  menu object      */
@end group
@end example
This method is used to remove and destroy the current menu associated
with window @code{wind} and restore the previous menu associated with
the window.  A last-in-first-out list of menus associated with a window
may be established via the @code{LoadMenu}, @code{LoadMenuStr} or
@code{Use} methods.

All function associations and modes previously associated with the
new menu will be restored.

The object returned is the new menu object.
@example
@group
@exdent Example:

object  myWind;

gPopMenu(myWind);
@end group
@end example
@sp 1
See also:  @code{LoadMenu, Use}
@end deffn











@deffn {Printf} Printf::Window
@sp 2
@example
@group
r = vPrintf(wind, fmt, ...);

object   wind;  /*  a window object       */
char    *fmt;   /*  format string         */
int      r;     /*  length of output      */
@end group
@end example
This method is used to display a string of text (@code{str}) in a sequential
fashion on window @code{wind}.  It is analogous to the standard C library
function @code{fprintf}.  All of the standard features of your C library
@code{fprintf} function are supported and that documentation should be
consulted for full documentation on the arguments.

This method is used to support the standard streams interface, is
actually defined by the @code{Stream} class, and is documented here for
convenience.

Note that this method begins with ``v'' since it takes variable arguments.
The length of the resulting output will be returned.
@example
@group
@exdent Example:

object  myWind;
int     age = 32;

vPrintf(myWind, "My age =- %d\n", age);
@end group
@end example
@sp 1
See also:  @code{TextOut, Puts, Write}
@end deffn



















@deffn {Puts} Puts::Window
@sp 2
@example
@group
r = gPuts(wind, str);

object   wind;  /*  a window object       */
char    *str;   /*  string to be output   */
int      r;     /*  length of str         */
@end group
@end example
This method is used to display a string of text (@code{str}) in a sequential
fashion on window @code{wind}.  This method is used to support the standard
streams interface, is actually defined by the @code{Stream} class, and
is documented here for convenience.

@code{str} is the text to be displayed. Text @code{str} will be
displayed on window @code{wind} and its length will be returned.
@example
@group
@exdent Example:

object  myWind;

gPuts(myWind, "Hello World\n");
@end group
@end example
@sp 1
See also:  @code{TextOut, Printf, Write}
@end deffn











@deffn {Read} Read::Window
@sp 2
@example
@group
r = gRead(wind, buf, len);

object   wind;  /*  a window object       */
char    *buf;   /*  input buffer          */
unsigned len;   /*  length of buffer      */
int      r;     /*  bytes read            */
@end group
@end example
This method is used to read (accept) @code{len} key strokes from the user
and place them in @code{buf}.

Since the @code{Window} class is a subclass of @code{Stream}, this
method is mainly provided to support the standard interface dictated by
the @code{Stream} class.

This method returns the number of characters actually accepted.  It will
not be more than @code{len}, but may be less if non-blocking io is selected
(via @code{SetBlock}) and there aren't enough characters available.
@code{SetRaw} may also be used to control whether backspace processing will
be performed.
@example
@group
@exdent Example:

object  myWind;
char    buf[80];

gRead(myWind, buf, sizeof(buf)-1);
@end group
@end example
@sp 1
See also:  @code{SetBlock, SetRaw, Gets, Getch}
@end deffn














@deffn {ScrollHorz} ScrollHorz::Window
@sp 2
@example
@group
r = gScrollHorz(wind, cols);

object  wind;   /*  a window object     */
int     cols;   /*  columns to scroll   */
object  r;      /*  the window object   */
@end group
@end example
This method is used to perform a horizontal scrolling of the text in
window @code{wind}.  This has the same effect as if the user caused
horizontal scrolling via the horizontal scroll bar at the bottom
of the window.  No text is lost, a different portion of the text is
displayed.

@code{wind} is the window which is to be affected, and @code{cols} is the
number of columns to scroll.  If @code{cols} is positive the scroll moves
the text to the left, and negative moves the text to the right.

The scaling factor associated with @code{cols} is set by
@code{SetScalingMode::Application}.
@example
@group
@exdent Example:

object  myWind;

gScrollHorz(myWind, 2);
@end group
@end example
@sp 1
See also:  @code{ScrollVert, SetScalingMode::Application}
@end deffn











@deffn {ScrollVert} ScrollVert::Window
@sp 2
@example
@group
r = gScrollVert(wind, rows);

object  wind;   /*  a window object     */
int     rows;   /*  rows to scroll      */
object  r;      /*  the window object   */
@end group
@end example
This method is used to perform a vertical scrolling of the text in
window @code{wind}.  This has the same effect as if the user caused
vertical scrolling via the vertical scroll bar at the right side
of the window.  No text is lost, a different portion of the text is
displayed.

@code{wind} is the window which is to be affected, and @code{rows} is the
number of rows to scroll.  If @code{rows} is positive the scroll moves
the text up, and negative moves the text down.

The scaling factor associated with @code{rows} is set by
@code{SetScalingMode::Application}.
@example
@group
@exdent Example:

object  myWind;

gScrollVert(myWind, 2);
@end group
@end example
@sp 1
See also:  @code{ScrollHorz, VertShift, SetScalingMode::Application}
@end deffn










@deffn {SetBlock} SetBlock::Window
@sp 2
@example
@group
r = gSetBlock(wind, flag);

object  wind;   /*  a window object     */
int     flag;   /*  enable/disable flag */
int     r;      /*  previous flag       */
@end group
@end example
This method is used to set the blocking mode associated with keyboard
IO.  It only has effect on @code{Read}, @code{Gets}, and @code{Getch}.
If flag is 1, blocking is enabled, and 0 disables blocking.  When a
new window is created it defaults to blocking enabled.

If blocking is turned on and a keyboard entry is requested (via
@code{Read}, @code{Gets} or @code{Getch}), the keyboard entry function
will not return until the request can be satisfied.  If, however,
blocking is disabled, a keyboard request is made, and insufficient
characters are available, then the input function will return immediately
with a return value indicating the result was short.

The value returned by this method is the previous blocking mode.
@example
@group
@exdent Example:

object  myWind;

gSetBlock(myWind, 0);
@end group
@end example
@sp 1
See also:  @code{Getch, SetRaw, Gets}
@end deffn















@deffn {SetMaxLines} SetMaxLines::Window
@sp 2
@example
@group
r = gSetMaxLines(wind, rows);

object  wind;   /*  a window object     */
int     rows;   /*  max rows            */
int     r;      /*  previous max rows   */
@end group
@end example
This method is used to set or obtain the maximum number of lines of text
which may be associated to a window.  This includes all lines associated
with a window, including lines not being displayed because they are
scrolled off the screen.  Whenever @code{rows} number of lines are
exceeded, WDS automatically and permanently removes the lines at the
logical top of the internal buffer in order to make room for new
lines.

WDS keeps a buffer which holds a programmer definable number of lines of
text.  The user is then able to scroll through this text.  If the
application attempts to display more lines than this maximum (via
@code{vPrintf} for example) the system will automatically call
@code{gVertShift} in order to eliminate the top line and make room for
the new line being appended.

@code{wind} is the window which is to be affected, and @code{rows} is the
maximum number of rows to retain.  If @code{rows} is zero or negative,
the value associated with the window will not be changed.  This is used
to obtain the current value without changing it.

The value returned is the previous value associated with the window.
@example
@group
@exdent Example:

object  myWind;

gSetMaxLines(myWind, 60);
@end group
@end example
@sp 1
See also:  @code{VertShift}
@end deffn













@deffn {SetName} SetName::Window
@sp 2
@example
@group
r = gSetName(wind, name);

object  wind;   /*  a window object     */
char   *name;   /*  the new name        */
object  r;      /*  w arg passed        */
@end group
@end example
This method is used to set the name associated with window @code{wind}.
The name associated with a window is what is displayed at the top of the
window, if it has a title bar.  @code{name} may also be an @code{object}
typecast to a (@code{char *}).  The value returned is the window
(@code{wind}) object passed.
@example
@group
@exdent Example:

object  myWind;

myWind = vNew(MainWindow, "Old Name");
gSetName(myWind, "New Name");
@end group
@end example
@sp 1
See also:  @code{New::MainWindow, GetName}
@end deffn













@deffn {SetParent} SetParent::Window
@sp 2
@example
@group
r = gSetParent(wind, prnt);

object  wind;   /*  child window object   */
object  prnt;   /*  parent window object  */
object  r;      /*  wind                  */
@end group
@end example
This method is used to create a child / parent window relationship.
This relationship is automatically established when a child window is
created.  However, this method is provided for increased flexibility.

Whenever the parent window is disposed, WDS will automatically dispose
of all its children windows.  This relationship should normally be
established prior to the child window being @code{Shown}.

The value returned is the child window passed.
@example
@group
@exdent Example:

object  myWind, parentWind;

gSetParent(myWind, parentWind);
@end group
@end example
@sp 1
See also:  @code{GetParent}
@end deffn
















@deffn {SetPosition} SetPosition::Window
@sp 2
@example
@group
r = gSetPosition(wind, vert, horz);

object  wind;   /*  a window object     */
int     vert;   /*  vertical position   */
int     horz;   /*  horizontal position */
object  r;      /*  wind arg passed     */
@end group
@end example
This method is used to set the initial position of a window.
@code{vert} and @code{horz} are in increments dictated by the
mode selected by @code{SetScalingMode::Application}.  The window
object passed is returned.

If this function is not called, Windows will automatically set it to a
reasonable default.
@example
@group
@exdent Example:

object  myWind;

myWind = vNew(MainWindow, "App Name");
gSetPosition(myWind, 3, 10);
@end group
@end example
@sp 1
See also:  @code{SetScalingMode::Application, SetSize, GetPosition}
@end deffn











@deffn {SetRaw} SetRaw::Window
@sp 2
@example
@group
r = gSetRaw(wind, flag);

object  wind;   /*  a window object     */
int     flag;   /*  enable/disable flag */
int     r;      /*  previous flag       */
@end group
@end example
This method is used to set the raw mode associated with keyboard
IO.  It only has effect on @code{Read}, @code{Gets}, and @code{Getch}.
If flag is 1, raw mode is enabled, and 0 disables raw mode.  The
default is raw mode disabled.

Normally, with raw mode disabled, when a user enters keyboard data,
they are able to correct mistakes by using the backspace and reentering
the correct data.  The resulting data the program receives is the
final, corrected input.

When raw mode is enabled, every key the user hits gets returned.  This
includes backspaces.  Therefore, with raw mode enabled, if the user types
``ABD'' followed by backspace and then ``C'', the program will receive all
five characters hit.  However, with raw mode disabled, the program would
only receive the resulting string, ``ABC''.

The value returned by this method is the previous raw mode.
@example
@group
@exdent Example:

object  myWind;

gSetRaw(myWind, 1);
@end group
@end example
@sp 1
See also:  @code{Getch, SetBlock, Gets}
@end deffn











@deffn {SetSize} SetSize::Window
@sp 2
@example
@group
r = gSetSize(wind, vert, horz);

object  wind;   /*  a window object     */
int     vert;   /*  vertical size       */
int     horz;   /*  horizontal size     */
object  r;      /*  wind arg passed     */
@end group
@end example
This method is used to set the initial size of a window.
@code{vert} and @code{horz} are in increments dictated by the
mode selected by @code{SetScalingMode::Application}.  The window
object passed is returned.

If this function is not called, Windows will automatically set it to a
reasonable default.
@example
@group
@exdent Example:

object  myWind;

myWind = vNew(MainWindow, "App Name");
gSetSize(myWind, 10, 40);
@end group
@end example
@sp 1
See also:  @code{SetScalingMode::Application, SetPosition, GetSize}
@end deffn









@deffn {SetStyle} SetStyle::Window
@sp 2
@example
@group
r = gSetStyle(wind, sty);

object  wind;   /*  a window object             */
DWORD   sty;    /*  the window style to use     */
object  r;      /*  returns wind                */
@end group
@end example
This method is used to set the window style associated with window
@code{wind} to style @code{sty}.  The @code{DWORD} data type and valid 
styles are defined by Windows and fully documented by the Windows
documentation.  See the Windows documentation for the function called
@code{CreateWindow}.  The style types normally begin with @code{WS_}
and would be or'ed together to form the selected style.
Note that WDS automatically assigns reasonable defaults to a window
style when it is created.
@example
@group
@exdent Example:

object  myWind;

myWind = vNew(MainWindow, "App Name");
gSetStyle(myWind, WS_OVERLAPPEDWINDOW | WS_VSCROLL | WS_HSCROLL);
@end group
@end example
@c @sp 1
@c See also:  @code{}
@end deffn








@deffn {SetTag} SetTag::Window
@sp 2
@example
@group
r = gSetTag(wind, tag);

object  wind;   /*  a window object   */
object  tag;    /*  tag               */
object  r;      /*  previous tag      */
@end group
@end example
This method is used to associated an arbitrary Dynace object with a
window object.  This may later be retrieved via the @code{GetTag} method.
Since WDS passes around the window object to all @code{Window} methods
this mechanism may be used to pass additional information with the
window.  And since Dynace treats all objects in a uniform manner, this
information attached to the window may be arbitrarily complex.

WDS does not dispose of the tag when the window object is disposed.
This method returns any previous object associated with the window or
@code{NULL}.
@example
@group
@exdent Example:

object  myWind;

gSetTag(myWind, gNewWithInt(ShortInteger, 17));
@end group
@end example
@sp 1
See also:  @code{GetTag, SetTag::Dialog}
@end deffn












@deffn {SetTopic} SetTopic::Window
@sp 2
@example
@group
pt = gSetTopic(wind, tpc);

object  wind;   /*  child window object   */
char    *tpc;   /*  help topic            */
char    *pt;    /*  previous help topic   */
@end group
@end example
This method is used to associate help text with window @code{wind}.
The help text is defined using the Windows help system and labeled
with the topic indicated by @code{tpc}.  Then, if the user hits
the F1 key while in the window, WDS will automatically bring up
the Windows help system and find the indicated topic.

WDS also supports dialog and control specific topics.  See the appropriate
sections.

This method returns any previous topic associated with the window.
@example
@group
@exdent Example:

object  myWind;

gSetTopic(myWind, "myWindHelp");
@end group
@end example
@sp 1
See also:  The @code{HelpSystem} class.
@end deffn













@deffn {Show} Show::Window
@sp 2
@example
@group
r = gShow(wind);

object  wind;   /*  a window object     */
int     r;      /*  always 0            */
@end group
@end example
Once a window object is created (via @code{vNew}) and its various attributes
set, @code{gShow} is called in order to actually create the Windows window,
with the selected attributes, and display (show) it.
@example
@group
@exdent Example:

object  myWind;

myWind = vNew(PopupWindow, "Name", 10, 40);
gShow(myWind);
@end group
@end example
@sp 1
See also:  @code{AutoShow, ProcessMessages::MainWindow}
@end deffn









@deffn {TextBrush} TextBrush::Window
@sp 2
@example
@group
r = gTextBrush(wind, brsh);

object  wind;   /*  a window object   */
object  brsh;   /*  brush object      */
object  r;      /*  wind              */
@end group
@end example
This method is used to determine what brush object is used for foreground
text which is displayed.  Any previously associated brush object will
be disposed.  This brush object will also be automatically disposed when the
window is disposed.

The window passed is returned.
@example
@group
@exdent Example:

object  myWind;

gTextBrush(myWind, vNew(SolidBrush, 255, 0, 0));
@end group
@end example
@sp 1
See also:  @code{BackBrush, Use, SetBackBrush::Application}
        and the @code{Brush} classes
@end deffn










@deffn {TextOut} TextOut::Window
@sp 2
@example
@group
r = gTextOut(wind, row, col, txt);

object  wind;   /*  a window object          */
int     row;    /*  row number of output     */
int     col;    /*  column number of output  */
char    *txt;   /*  string to be output      */
object  r;      /*  wind                     */
@end group
@end example
This method is used to display a string of text (@code{txt}) at row
@code{row} and column @code{col}.  @code{row} and @code{col} have their
origin in the upper left hand corner of the window and are scaled as
dictated by the @code{SetScalingMode::Application} method.
Their index origin is 0.

This method returns the window argument passed.
@example
@group
@exdent Example:

object  myWind;

gTextOut(myWind, 10, 30, "Hello World");
@end group
@end example
@sp 1
See also:  @code{Write, Puts, Printf}
@end deffn









@deffn {Update} Update::Window
@sp 2
@example
@group
r = gUpdate(wind);

object  wind;   /*  a window object     */
object  r;      /*  wind                */
@end group
@end example
This method is used to explicitly update the display with changes the program
may have made to window @code{wind}.  WDS normally handles this need, however,
this method is available in case the programmer performs special processing
and wishes to update the entire window at one time.

This method returns the window argument passed.
@code{Update::MainWindow}.
@example
@group
@exdent Example:

object  myWind;

gUpdate(myWind);
@end group
@end example
@c @sp 1
@c See also:  @code{}
@end deffn









@deffn {Use} Use::Window
@sp 2
@example
@group
r = gUse(wind, obj);

object  wind;   /*  a window object     */
object  obj;    /*  arbitrary object    */
object  r;      /*  the object passed   */
@end group
@end example
This method is used as a general purpose mechanism to associated a
number of object types with a window.  @code{obj} may be a @code{Font,
Icon, Cursor, Menu,} or the background @code{Brush} object.  Those
objects would have normally been created via their associated classes
and then may be associated with a window via this method.

Note that the same object should not be associated with more than one
window.  The problem is that if one window is deleted, WDS would delete
all the objects associated with that window and then the other window
would reference objects which have been deleted.  The way around this is
to use the @code{Copy} method in order to make a copy of an object prior
to associating with a new window.  This way there would be two
independent objects such that if one is deleted the other would still
exist.

Note also that the @code{Application} class may be used to set application
wide defaults for these types of objects.

The value returned is the object passed.
@example
@group
@exdent Example:

object  myWind, myOtherWind, someFont;

someFont = vNew(ExternalFont, "Times New Roman", 12);
if (someFont)  @{
        gUse(myWind, someFont);
        gUse(myOtherWind, gCopy(someFont));
@}
@end group
@end example
@sp 1
See also:  @code{LoadFont, LoadIcon, LoadCursor, LoadMenu,}
        @code{TextBrush, BackBrush}
@end deffn








@deffn {VertShift} VertShift::Window
@sp 2
@example
@group
r = gVertShift(wind, rows);

object  wind;   /*  a window object     */
int     rows;   /*  rows to shift       */
object  r;      /*  the window object   */
@end group
@end example
This method is used to perform a vertical shift of the text in the
memory buffer which is being displayed in the window.  WDS keeps
a buffer which holds a programmer definable number of lines of text.
The user is then able to scroll through this text.  If the application
attempts to display more lines than this maximum (via @code{vPrintf}
for example) the system will automatically call @code{gVertShift} in
order to eliminate the top line and make room for the new line being
appended.  This method is mainly used internally.

@code{wind} is the window which is to be affected, and @code{rows} is the
number of rows to scroll.  If @code{rows} is positive the scroll moves
the text up, and negative moves the text down.  Scrolling up permanently
destroys lines at the beginning of the buffer and scrolling down
permanently destroys lines at the end of the buffer.

The scaling factor associated with @code{rows} is set by
@code{SetScalingMode::Application} and the maximum lines associated with
a window may be set with @code{SetMaxLines}.
@example
@group
@exdent Example:

object  myWind;

gVertShift(myWind, 2);
@end group
@end example
@sp 1
See also:  @code{ScrollVert, SetMaxLines}
@end deffn








@deffn {Write} Write::Window
@sp 2
@example
@group
r = gWrite(wind, txt, len);

object   wind;  /*  a window object       */
char    *txt;   /*  string to be output   */
unsigned len;   /*  length of string      */
int      r;     /*  bytes written         */
@end group
@end example
This method is used to display a string of text (@code{txt}) in a sequential
fashion on window @code{wind}.  Since the @code{Window} class is a subclass
of @code{Stream}, this method is mainly provided to support the standard
interface dictated by the @code{Stream} class.  By providing this method,
the @code{Window} class automatically inherits the @code{Puts} and
@code{Printf} capability.

@code{txt} is the text to be displayed and @code{len} is the length of that
string.  Text @code{txt} will be displayed on window @code{wind} and
@code{len} will be returned.
@example
@group
@exdent Example:

object  myWind;

gWrite(myWind, "Hello World\n", 12);
@end group
@end example
@sp 1
See also:  @code{TextOut, Puts, Printf}
@end deffn











@subsection Main Window
This class, named @code{MainWindow}, is used to create and manipulate an
application's main window.  There is normally one main window associated
with each application and this is the first window created.

This class is a subclass of the @code{Window} class and therefore
inherits all of the @code{Window} class's functionality.  The methods
documented in this subsection are only those which are particular
to the @code{MainWindow} class.



@deffn {New} New::MainWindow
@sp 2
@example
@group
r = vNew(MainWindow, ttl);

char    *ttl;   /*  window title  */
object  r;      /*  new window    */
@end group
@end example
This class method is used to create the main application window.
Since this is a class method the first argument must be literally
@code{MainWindow}.  The window title (@code{ttl}) will appear at
the top of the window.

The value returned is the new window created.
@example
@group
@exdent Example:

object  myWind;

myWind = vNew(MainWindow, "My Application");
@end group
@end example
@sp 1
See also:  @code{New::ChildWindow, New::PopupWindow,}
      @code{Show, Dispose::Window}
@end deffn







@deffn {ProcessMessages} ProcessMessages::MainWindow
@sp 2
@example
@group
r = gProcessMessages(wind);

object  wind;   /*  a window object       */
int     r;      /*  final message result  */
@end group
@end example
Once the main application window is created (via @code{vNew}) and
its various attributes set, @code{gProcessMessages} is called in order
to actually create the Windows window, with the selected attributes,
display (show) it, and process the application's messages.  Processing
the application's messages is what allows the user to interact with the
application.

The value returned is the return value specified when the application is
terminated.  This can be specified via @code{QuitApplication::Application}.
@example
@group
@exdent Example:

object  myWind;

myWind = vNew(MainWindow, "Application Name");
gProcessMessages(myWind);
@end group
@end example
@sp 1
See also:  @code{New, Show::Window, ProcessMessages::MessageDispatcher}
@end deffn











@subsection Child Windows
The Child Window class, called @code{ChildWindow}, is used in the creation
of windows which are children of other windows.  As such, they can not
be positioned outside of their parent window, and they get iconized along
with their parent window.


This class is a subclass of the @code{Window} class and therefore
inherits all of the @code{Window} class's functionality.  The methods
documented in this subsection are only those which are particular
to the @code{ChildWindow} class.





@deffn {New} New::ChildWindow
@sp 2
@example
@group
r = vNew(ChildWindow, prnt, rows, cols);

object  prnt;   /*  parent window object  */
int     rows;   /*  length of window      */
int     cols;   /*  width of window       */
object  r;      /*  new window            */
@end group
@end example
This class method is used to create a child window.
Since this is a class method the first argument must be literally
@code{ChildWindow}.  @code{prnt} represents the parent window
object of which the new window will be a child.  @code{rows} and
@code{cols} determine the initial size of the window in are specified
in increments dictated by @code{SetScalingMode::Application}.

The default style associated with the child window is
@code{WS_CHILD | WS_VISIBLE} and may be changed with @code{SetStyle::Window}.
The position may be set with 
@iftex
@hfil @break 
@end iftex
@code{SetPosition::Window}.

The value returned is the new window created.
@example
@group
@exdent Example:

object  myWind, mainWind;

myWind = vNew(ChildWindow, mainWind, 10, 45);
@end group
@end example
@sp 1
See also:  @code{New::MainWindow, New::PopupWindow,}
      @code{Show, Dispose::Window}
@end deffn








@subsection Popup Windows
The @code{PopupWindow} class is used to create arbitrary windows which
function independently of other windows.  That is, they may overlap or
be moved outside of other windows and iconized independently of other
windows.  If, however, a popup window is associated with a parent window
(via @code{SetParent::Window}) then it may move outside the parent but
will be iconized with it.

This class is a subclass of the @code{Window} class and therefore
inherits all of the @code{Window} class's functionality.  The methods
documented in this subsection are only those which are particular
to the @code{PopupWindow} class.






@deffn {New} New::PopupWindow
@sp 2
@example
@group
r = vNew(PopupWindow, name, rows, cols);

char    *name;  /*  window title      */
int     rows;   /*  length of window  */
int     cols;   /*  width of window   */
object  r;      /*  new window        */
@end group
@end example
This class method is used to create a popup window.  Since this is a
class method the first argument must be literally @code{PopupWindow}.
@code{name} represents the window title and will appear in the window's
top border.  @code{rows} and @code{cols} determine the initial size of
the window in are specified in increments dictated by
@code{SetScalingMode::Application}.

The default style associated with the popup window is @code{WS_POPUP |
WS_VISIBLE | WS_CAPTION | WS_THICKFRAME | WS_MINIMIZEBOX |
WS_MAXIMIZEBOX | WS_SYSMENU} 
@iftex
@hfil @break 
@end iftex
and may be changed with
@code{SetStyle::Window}.  The position may be set with
@iftex
@break 
@end iftex
@code{SetPosition::Window}.

The value returned is the new window created.
@example
@group
@exdent Example:

object  myWind, mainWind;

myWind = vNew(PopupWindow, "Window Title", 10, 45);
@end group
@end example
@sp 1
See also:  @code{New::MainWindow, New::ChildWindow,}
      @code{Show, Dispose::Window}
@end deffn




@section Printing
The @code{Printer} class is used for all aspects of printing.
It is a subclass of the Dynace @code{Stream} class and as
such inherits all of its functionality.

All methods in the class use an argument referred to as @code{pntr}.
This will always be the printer object returned by @code{QueryPrinter},
@code{New} or @code{NewWithHDC} and used to identify the printer
which is to be effected.

The value returned by all methods which cause text or graphics output is
very significant.  During normal operation, the printer object passed
will be returned.  However, if an error occurs or the user aborted the
report a @code{NULL} will be returned.  In this case further output
should be avoided (although it won't hurt - it'll just be ignored) and
the printer object should be disposed.

All positioning parameters are in increments dictated by
@code{SetScale}.  Positions begin in the upper left hand corner and have
an index origin of 0.



















@deffn {Arc} Arc::Printer
@sp 2
@example
@group
r = gArc(pntr, yBeg, xBeg, yEnd, xEnd, yaBeg, xaBeg, yaEnd, xaEnd);

object  pntr;   /*  printer object         */
int     yBeg;   /*  starting row           */
int     xBeg;   /*  starting column        */
int     yEnd;   /*  ending row             */
int     xEnd;   /*  ending column          */
int     yaBeg;  /*  starting arc row       */
int     xaBeg;  /*  starting arc column    */
int     yaEnd;  /*  ending arc row         */
int     xaEnd;  /*  ending arc column      */
object  r;      /*  printer object         */
@end group
@end example
This method is used to output an elliptical arc to the printer.  The
currently selected pen object will indicate the thickness and pattern of
the outline of the shape, and the currently selected brush object will
be used to determine what pattern the shape will be filled with.  The
parameters indicate location of the beginning and ending of the shape as
well as the coordinated of the arc and are in increments dictated by
@code{SetScale}.

See the note at the beginning of this section regarding the return value.
@example
@group
@exdent Example:

object  pntr;

if (!gArc(pntr, 10, 10, 180, 140, 12, 12, 128, 135))
        abort report;
@end group
@end example
@sp 1
See also:  @code{NewPage, SetScale, Use,}
        @code{Line, Rectangle, Ellipse, RoundRect,}
@iftex
@hfil @break @hglue .63in 
@end iftex
@code{Chord, Pie}
@end deffn












@deffn {Chord} Chord::Printer
@sp 2
@example
@group
r = gChord(pntr, yBeg, xBeg, yEnd, xEnd,
                 ylBeg, xlBeg, ylEnd, xlEnd);

object  pntr;   /*  printer object         */
int     yBeg;   /*  starting row           */
int     xBeg;   /*  starting column        */
int     yEnd;   /*  ending row             */
int     xEnd;   /*  ending column          */
int     ylBeg;  /*  starting line row      */
int     xlBeg;  /*  starting line column   */
int     ylEnd;  /*  ending line row        */
int     xlEnd;  /*  ending line column     */
object  r;      /*  printer object         */
@end group
@end example
This method is used to output a chord shape to the printer.  The
currently selected pen object will indicate the thickness and pattern of
the outline of the shape, and the currently selected brush object will
be used to determine what pattern the shape will be filled with.  The
parameters indicate location of the beginning and ending of the shape as
well as the coordinated of the intersecting line and are in increments
dictated by @code{SetScale}.

See the note at the beginning of this section regarding the return value.
@example
@group
@exdent Example:

object  pntr;

if (!gChord(pntr, 10, 10, 180, 140, 12, 12, 128, 135))
        abort report;
@end group
@end example
@sp 1
See also:  @code{NewPage, SetScale, Use,}
        @code{Line, Rectangle, Ellipse, RoundRect, Pie, Arc}
@end deffn








@deffn {DeepDispose} DeepDispose::Printer
@sp 2
This method performs the same function as @code{Dispose}.  See that
method for details.
@end deffn





@deffn {Dispose} Dispose::Printer
@sp 2
@example
@group
r = gDispose(pntr); 

object  pntr;   /*  printer object  */
object  r;      /*  NULL            */
@end group
@end example
This method is used to flush the final output page, close the printer and
dispose of the printer object.  It must be called when a report is
is complete and the printer is no longer needed.

The value returned is always @code{NULL} and may be used to null out
the variable which contained the object being disposed in order to
avoid future accidental use.
@example
@group
@exdent Example:

object  pntr;

pntr = gDispose(pntr);
@end group
@end example
@c @sp 1
@c See also:  @code{}
@end deffn
















@deffn {Ellipse} Ellipse::Printer
@sp 2
@example
@group
r = gEllipse(pntr, yBeg, xBeg, yEnd, xEnd);

object  pntr;   /*  printer object   */
int     yBeg;   /*  starting row     */
int     xBeg;   /*  starting column  */
int     yEnd;   /*  ending row       */
int     xEnd;   /*  ending column    */
object  r;      /*  printer object   */
@end group
@end example
This method is used to output an ellipse to the printer.  The currently
selected pen object will indicate the thickness and pattern of the
outline of the shape, and the currently selected brush object will be
used to determine what pattern the shape will be filled with.  The
parameters indicate location of the beginning and ending of the shape
and are in increments dictated by @code{SetScale}.

See the note at the beginning of this section regarding the return value.
@example
@group
@exdent Example:

object  pntr;

if (!gEllipse(pntr, 10, 20, 30, 40))
        abort report;
@end group
@end example
@sp 1
See also:  @code{NewPage, SetScale, Use,}
        @code{Line, Rectangle, RoundRect, Chord, Pie, Arc}
@end deffn













@deffn {Handle} Handle::Printer
@sp 2
@example
@group
r = gHandle(pntr); 

object  pntr;   /*  printer object         */
HANDLE  hdc;    /*  handle device context  */
@end group
@end example
This method is used to obtain the device context handle associated with
an opened printer represented by @code{pntr}.  It is used internally
by Windows and should not normally be needed.
@example
@group
@exdent Example:

HDC     hdc;

hdc = (HDC) gHandle(pntr);
@end group
@end example
@c @sp 1
@c See also:  @code{}
@end deffn

















@deffn {Line} Line::Printer
@sp 2
@example
@group
r = gLine(pntr, yBeg, xBeg, yEnd, xEnd);

object  pntr;   /*  printer object   */
int     yBeg;   /*  starting row     */
int     xBeg;   /*  starting column  */
int     yEnd;   /*  ending row       */
int     xEnd;   /*  ending column    */
object  r;      /*  printer object   */
@end group
@end example
This method is used to output a line to the printer.  The currently
selected pen object will indicate the thickness and pattern of the line.
The parameters indicate location of the beginning and ending of the line
and are in increments dictated by @code{SetScale}.

See the note at the beginning of this section regarding the return value.
@example
@group
@exdent Example:

object  pntr;

if (!gLine(pntr, 10, 20, 30, 40))
        abort report;
@end group
@end example
@sp 1
See also:  @code{NewPage, SetScale, Use,}
        @code{Arc, Rectangle, Ellipse, RoundRect, Chord, Pie}
@end deffn












@deffn {LoadFont} LoadFont::Printer
@sp 2
@example
@group
r = gLoadFont(pntr, fname, sz);

object   pntr;  /*  printer object   */
char    *fname; /*  font name        */
int      sz;    /*  point size       */
object   r;     /*  font object      */
@end group
@end example
This method is used to load an arbitrary font by name at any point size
and associate it as the default font for future text output to printer
@code{pntr}.  @code{fname} is the full name of the font as it appears
when you list the available fonts via the control-panel / fonts Windows
utility, minus the font type in parentheses.  @code{sz} indicates the
desired point size.

Any font object previously associated with the printer will be
disposed.  When the printer object is disposed, the font object
will also be disposed.

The value returned is an object representing the font loaded, or
@code{NULL} if the font was not found.

Note that @code{fname} may also be a Dynace object cast as a (@code{char *}).
@example
@group
@exdent Example:

object  pntr;

gLoadFont(pntr, "Times New Roman", 12);
@end group
@end example
@sp 1
See also:  @code{LoadSystemFont, Indirect::ExternalFont, Use}
@end deffn














@deffn {LoadSystemFont} LoadSystemFont::Printer
@sp 2
@example
@group
r = gLoadSystemFont(pntr, fnt);

object   pntr;  /*  a printer object  */
unsigned fnt;   /*  font identifier  */
object   r;     /*  font object      */
@end group
@end example
This method is used to load a Windows predefined font and associate it
with printer @code{pntr}.  The last font associated with a printer is the
one which will be used when any text is output to the printer.

@code{fnt} is a Windows defined macro which identifies the font.  The
available options are defined in the Windows documentation under the
function named @code{GetStockObject} and normally end with @code{_FONT}.
The font object will be automatically destroyed whenever the printer
object is disposed.

The value returned is an object representing the font, or
@code{NULL} if the font was not found.
@example
@group
@exdent Example:

object  pntr;

gLoadSystemFont(pntr, SYSTEM_FONT);
@end group
@end example
@sp 1
See also:  @code{LoadFont, Load::SystemFont, Use}
@end deffn

















@deffn {New} New::Printer
@sp 2
@example
@group
pntr = vNew(Printer, pwind, rname); 

object  pwind;  /*  parent window       */
char    *rname; /*  report name         */
object  pntr;   /*  new printer object  */
@end group
@end example
This class method is used to create a new @code{Printer} object which
will be used by the @code{Printer} instance methods to control the
printer, fonts and content of output.  This method opens up the
default printer.

@code{pwind} is the window which will act as the parent to any
messages which may be displayed when a report is being printed.
@code{rname} determines what the report is referred to as in any
messages.

The value returned represents the printer to be used.  If the default
printer couldn't be opened, @code{NULL} will be returned.  If an object
is returned, it must be disposed (via @code{Dispose}) when it is no
longer needed.
@example
@group
@exdent Example:

object  pntr, pwind;

pntr = vNew(Printer, pwind, "My Report");
@end group
@end example
@sp 1
See also:  @code{QueryPrinter, NewWithHDC}
@end deffn











@deffn {NewPage} NewPage::Printer
@sp 2
@example
@group
r = gNewPage(pntr); 

object  pntr;   /*  printer object  */
object  r;      /*  printer object  */
@end group
@end example
This method is used to flush and eject the current output page (if anything
had been output to it) and prepare for a possible new page of output.

See the note at the beginning of this section regarding the return value.
@example
@group
@exdent Example:

object  pntr;

if (!gNewPage(pntr))
        abort report;
@end group
@end example
@c @sp 1
@c See also:  @code{}
@end deffn













@deffn {NewWithHDC} NewWithHDC::Printer
@sp 2
@example
@group
pntr = gNewWithHDC(Printer, pwind, rname, hdc); 

object  pwind;  /*  parent window       */
char    *rname; /*  report name         */
HDC     hdc;    /*  device context      */
object  pntr;   /*  new printer object  */
@end group
@end example
This class method is used to create a new @code{Printer} object which
will be used by the @code{Printer} instance methods to control the
printer, fonts and content of output.

This method opens up the printer identified by @code{hdc}.  This is
a Windows internal identifier and may be obtained via the
@code{PrintDialog} class.

@code{pwind} is the window which will act as the parent to any
messages which may be displayed when a report is being printed.
@code{rname} determines what the report is referred to as in any
messages.

The value returned represents the printer to be used.  If the 
printer couldn't be opened, @code{NULL} will be returned.

This method is seldom needed due to @code{QueryPrinter} and @code{New}.
Use them. If an object is returned, it must be disposed (via
@code{Dispose}) when it is no longer needed.
@example
@group
@exdent Example:

object  pntr, pwind;
HDC     hdc;

pntr = gNewWithHDC(Printer, pwind, "My Report", hdc);
@end group
@end example
@sp 1
See also:  @code{QueryPrinter, New}
@end deffn



















@deffn {Pie} Pie::Printer
@sp 2
@example
@group
r = gPie(pntr, yBeg, xBeg, yEnd, xEnd, yaBeg, xaBeg, yaEnd, xaEnd);

object  pntr;   /*  printer object         */
int     yBeg;   /*  starting row           */
int     xBeg;   /*  starting column        */
int     yEnd;   /*  ending row             */
int     xEnd;   /*  ending column          */
int     yaBeg;  /*  starting arc row       */
int     xaBeg;  /*  starting arc column    */
int     yaEnd;  /*  ending arc row         */
int     xaEnd;  /*  ending arc column      */
object  r;      /*  printer object         */
@end group
@end example
This method is used to output a pie shaped wedge to the printer.  The
currently selected pen object will indicate the thickness and pattern of
the outline of the shape, and the currently selected brush object will
be used to determine what pattern the shape will be filled with.  The
parameters indicate location of the beginning and ending of the shape as
well as the coordinated of the intersecting line and are in increments
dictated by @code{SetScale}.

See the note at the beginning of this section regarding the return value.
@example
@group
@exdent Example:

object  pntr;

if (!gPie(pntr, 10, 10, 180, 140, 12, 12, 128, 135))
        abort report;
@end group
@end example
@sp 1
See also:  @code{NewPage, SetScale, Use,}
        @code{Line, Rectangle, Ellipse, RoundRect,}
@iftex
@hfil @break @hglue .63in 
@end iftex
@code{Chord, Arc}
@end deffn
















@deffn {Printf} Printf::Printer
@sp 2
@example
@group
r = vPrintf(pntr, fmt, ...);

object   pntr;  /*  printer object     */
char    *fmt;   /*  format string      */
int      r;     /*  length of output   */
@end group
@end example
This method is used to output a string of text (@code{str}) in a sequential
fashion on printer @code{pntr}.  It is analogous to the standard C library
function @code{fprintf}.  All of the standard features of your C library
@code{fprintf} function are supported and that documentation should be
consulted for full documentation on the arguments.

This method is used to support the standard streams interface, is
actually defined by the @code{Stream} class, and is documented here for
convenience.

Note that this method begins with ``v'' since it takes variable arguments.

The value returned by this method is very significant.  During normal
operation, the length of the output string will be returned.  However,
if an error occurs or the user aborted the report a @code{-1} will be
returned.  In this case further output should be avoided (although it
won't hurt - it'll just be ignored) and the printer object should be
disposed.
@example
@group
@exdent Example:

object  pntr;
int     age = 32;

if (-1 == vPrintf(pntr, "My age =- %d\n", age))
        goto abort report;
@end group
@end example
@sp 1
See also:  @code{TextOut, Puts}
@end deffn














@deffn {Puts} Puts::Printer
@sp 2
@example
@group
r = gPuts(pntr, str);

object   pntr;  /*  printer object        */
char    *str;   /*  string to be output   */
int      r;     /*  length of str         */
@end group
@end example
This method is used to output a string of text (@code{str}) in a sequential
fashion on printer @code{pntr}.  This method is used to support the standard
streams interface, is actually defined by the @code{Stream} class, and
is documented here for convenience. @code{str} is the text to be output. 

The value returned by this method is very significant.  During normal
operation, the length of the output string will be returned.  However,
if an error occurs or the user aborted the report a @code{-1} will be
returned.  In this case further output should be avoided (although it
won't hurt - it'll just be ignored) and the printer object should be
disposed.
@example
@group
@exdent Example:

object  pntr;

if (-1 == gPuts(pntr, "Hello World\n"))
        goto abort report;
@end group
@end example
@sp 1
See also:  @code{TextOut, Printf}
@end deffn











@deffn {QueryPrinter} QueryPrinter::Printer
@sp 2
@example
@group
pntr = gQueryPrinter(Printer, pwind, rname); 

object  pwind;  /*  parent window       */
char    *rname; /*  report name         */
object  pntr;   /*  new printer object  */
@end group
@end example
This class method is used to create a new @code{Printer} object which
will be used by the @code{Printer} instance methods to control the
printer, fonts and content of output.

This method queries the user for printer selection and optional
configuration, and uses that information to open the selected
printer.

@code{pwind} is the window which will act as the parent to any
messages which may be displayed when a report is being printed.
@code{rname} determines what the report is referred to as in any
messages.

The value returned represents the printer to be used.  If the user
canceled the printer selection or the printer couldn't be opened,
@code{NULL} will be returned. If an object is returned, it must be
disposed (via @code{Dispose}) when it is no longer needed.
@example
@group
@exdent Example:

object  pntr, pwind;

pntr = gQueryPrinter(Printer, pwind, "My Report");
@end group
@end example
@sp 1
See also:  @code{New, NewWithHDC}
@end deffn
















@deffn {Rectangle} Rectangle::Printer
@sp 2
@example
@group
r = gRectangle(pntr, yBeg, xBeg, yEnd, xEnd);

object  pntr;   /*  printer object   */
int     yBeg;   /*  starting row     */
int     xBeg;   /*  starting column  */
int     yEnd;   /*  ending row       */
int     xEnd;   /*  ending column    */
object  r;      /*  printer object   */
@end group
@end example
This method is used to output a rectangle to the printer.  The currently
selected pen object will indicate the thickness and pattern of the outline
of the shape, and the currently selected brush object will be used to determine
what pattern the shape will be filled with.
The parameters indicate location of the beginning and ending of the shape
and are in increments dictated by @code{SetScale}.

See the note at the beginning of this section regarding the return value.
@example
@group
@exdent Example:

object  pntr;

if (!gRectangle(pntr, 10, 20, 30, 40))
        abort report;
@end group
@end example
@sp 1
See also:  @code{NewPage, SetScale, Use,}
        @code{Line, Ellipse, RoundRect, Chord, Pie, Arc}
@end deffn















@deffn {RoundRect} RoundRect::Printer
@sp 2
@example
@group
r = gRoundRect(pntr, yBeg, xBeg, yEnd, xEnd, eHt, eWth);

object  pntr;   /*  printer object         */
int     yBeg;   /*  starting row           */
int     xBeg;   /*  starting column        */
int     yEnd;   /*  ending row             */
int     xEnd;   /*  ending column          */
int     eHt;    /*  corner ellipse height  */
int     eWth;   /*  corner ellipse width   */
object  r;      /*  printer object         */
@end group
@end example
This method is used to output a rectangle with rounded corners to the
printer.  The currently selected pen object will indicate the thickness
and pattern of the outline of the shape, and the currently selected
brush object will be used to determine what pattern the shape will be
filled with.  The parameters indicate location of the beginning and
ending of the shape and are in increments dictated by @code{SetScale}.

See the note at the beginning of this section regarding the return value.
@example
@group
@exdent Example:

object  pntr;

if (!gRoundRect(pntr, 10, 20, 30, 40, 1, 1))
        abort report;
@end group
@end example
@sp 1
See also:  @code{NewPage, SetScale, Use,}
        @code{Line, Rectangle, Ellipse, Chord, Pie, Arc}
@end deffn
















@deffn {SetScale} SetScale::Printer
@sp 2
@example
@group
r = gSetScale(pntr, ht, wth);

object  pntr;   /*  printer object  */
int     ht;     /*  logical height  */
int     wth;    /*  logical width   */
object  r;      /*  printer object  */
@end group
@end example
This method is used to set the logical scaling factor associated with an
output device.  Each output page is divided up into @code{ht} evenly
spaced row and @code{wth} evenly spaced columns.  This scaling factor
will be used by all text and graphics output when determining output size
and location on the page.

Coordinates begin at the upper left hand of the page and are zero origin.
The defaults are set to 80 columns and 66 lines.
@example
@group
@exdent Example:

object  pntr;

gSetScale(pntr, 132, 160);
@end group
@end example
@sp 1
See also:  @code{TextOut}
@end deffn












@deffn {TextOut} TextOut::Printer
@sp 2
@example
@group
r = gTextOut(pntr, row, col, txt); 

object  pntr;   /*  printer object  */
int     row;    /*  output row      */
int     col;    /*  output column   */
char    *txt;   /*  text to output  */
object  r;      /*  printer object  */
@end group
@end example
This method is used to output a line of text (@code{txt}) to the printer.
@code{row} and @code{col} indicate the position of the text and are in
increments dictated by @code{SetScale}.  Positions begin in the upper
left hand corner and have an index origin of 0.

The text will be printed in the currently selected font.  Note that the
actual printing will not occur until the report is complete and the
print object disposed.

See the note at the beginning of this section regarding the return value.
@example
@group
@exdent Example:

object  pntr;

if (!gTextOut(pntr, 10, 20, "output text"))
        abort report;
@end group
@end example
@sp 1
See also:  @code{NewPage, Printf, SetScale, LoadFont}
@end deffn







@deffn {Use} Use::Printer
@sp 2
@example
@group
r = gUse(pntr, obj);

object  pntr;   /*  printer object     */
object  obj;    /*  arbitrary object   */
object  r;      /*  the object passed  */
@end group
@end example
This method is used as a general purpose mechanism to associated a
number of object types with a printer.  @code{obj} may be a @code{Font,
Brush} or @code{Pen} object.  Those objects would have normally been
created via their associated classes and then may be associated with a
printer via this method.

Note that the same object should not be associated with more than one
printer object.  The problem is that if one printer object is disposed, WDS
would dispose of all the objects associated with that printer and then the
other printer would reference objects which have been deleted.  The way
around this is to use the @code{Copy} method in order to make a copy of
an object prior to associating it with a new printer.  This way there would
be two independent objects such that if one is deleted the other would
still exist.

The value returned is the object passed.
@example
@group
@exdent Example:

object  myPntr, myOtherPntr, someFont;

someFont = vNew(ExternalFont, "Times New Roman", 12);
if (someFont)  @{
        gUse(myPntr, someFont);
        gUse(myOtherPntr, gCopy(someFont));
@}
@end group
@end example
@sp 1
See also:  @code{LoadFont} and the @code{Brush} and @code{Pen} classes.
@end deffn












@section Menus
The @code{Menu} class, although not used directly by an application, is
used to house the common functionality associated with the
@code{ExternalMenu} and @code{InternalMenu} classes.  Most of the
functionality of those classes is implemented and documented in this
section.

External menus (those defined in resources) would be created and used
via the @code{ExternalMenu} class.  Internal menus (those defined at runtime
by the application code) would be created and used via the
@code{InternalMenu} and @code{PopupMenu} classes.  Since both of these
classes are subclasses of the @code{Menu} class, they inherit most of
their functionality from it.







@deffn {Associate} Associate::Menu
@sp 2
@example
@group
r = mAssociate(menu, id, fun);

object  menu;   /*  a menu object     */
int     id;     /*  menu item id      */
long    (*fun)(object);  /*  function to execute  */
object  r;      /*  the menu object   */
@end group
@end example
This method is used to associate a function with a menu item such that if
the user selects the menu item, @code{fun} gets executed.

In the case of external menus, @code{id} is normally a macro created by
the resource editor while the programmer defines the menu and identifies
a particular menu option.

@code{fun} is the application specific function which automatically gets
executed whenever the user selects the indicated option.  @code{fun} has
the following structure:
@example
@group
static  long    menu_option(object wind)
@{
           .
           .
           .
        return 0L;
@}
@end group
@end example
Where @code{wind} is the window object which the menu is attached to.
The value returned is what is returned by the window procedure associated
with the window.  This value is normally 0 and is fully documented by
the Windows documentation under the @code{WM_COMMAND} message.

The value returned is the menu object passed.
@example
@group
@exdent Example:

object  myMenu;

mAssociate(myMenu, ID_FILE_OPEN, menu_option);
@end group
@end example
@sp 1
See also:  @code{AddMenuOption::InternalMenu, AddMenuOption::PopupMenu}
@end deffn








@deffn {DeepDispose} DeepDispose::Menu
@sp 2
@example
@group
r = gDeepDispose(menu);

object  menu;   /*  a menu object    */
object  r;      /*  NULL             */
@end group
@end example
This method is used to dispose of a menu object and all menu objects
which were pushed (via @code{Push}) into the same menu stack.  It is not
often needed because menus associated with a window are automatically
disposed of (via this method) when the window is disposed.

If @code{menu} is an internal menu, all associated popup menus will
also be disposed.

The value returned is always @code{NULL} and may be used to null out
the variable which contained the object being disposed in order to
avoid future accidental use.
@example
@group
@exdent Example:

object  myMenu;

myMenu = gDeepDispose(myMenu);
@end group
@end example
@sp 1
See also:  @code{Dispose, Push}
@end deffn









@deffn {Dispose} Dispose::Menu
@sp 2
@example
@group
r = gDispose(menu);

object  menu;   /*  a menu object   */
object  r;      /*  NULL            */
@end group
@end example
This method is used to dispose of a menu object when it is no longer
needed.  It is not often needed because menus associated with a window
are automatically disposed of (via @code{DeepDispose}) when the window
is disposed.

If @code{menu} is an internal menu, all associated popup menus will
also be disposed.

The value returned is always @code{NULL} and may be used to null out
the variable which contained the object being disposed in order to
avoid future accidental use.
@example
@group
@exdent Example:

object  myMenu;

myMenu = gDispose(myMenu);
@end group
@end example
@sp 1
See also:  @code{DeepDispose}
@end deffn












@deffn {Handle} Handle::Menu
@sp 2
@example
@group
h = gHandle(menu);

object  menu;   /*  menu object    */
HANDLE  h;      /*  Windows handle */
@end group
@end example
This method is used to obtain the Windows internal handle associated with
a menu object.  

Note that this method may be used with most WDS objects in order to obtain
the internal handle that Windows normally associates with each type of object.
See the appropriate documentation.
@example
@group
@exdent Example:

object  myMenu;
HANDLE  h;

h = gHandle(myMenu);
@end group
@end example
@c @sp 1
@c See also:  @code{}
@end deffn









@deffn {MenuFunction} MenuFunction::Menu
@sp 2
@example
@group
fun = mMenuFunction(menu, id);

object  menu;   /*  a menu object     */
int     id;     /*  menu item id      */
long    (*fun)(object);  /*  function to execute  */
@end group
@end example
This method is used to obtain the function which was associated with a
menu item.
@c @example
@c @group
@c @exdent Example:
@c @end group
@c @end example
@sp 1
See also:  @code{Associate}
@end deffn







@deffn {Pop} Pop::Menu
@sp 2
@example
@group
prev = gPop(top);

object  top;    /*  top menu       */
object  prev;   /*  previous menu  */
@end group
@end example
This method is used to dispose of menu @code{top} and gain access to the
next menu in menu @code{top}'s last in, first out list.  It is mainly
used by the @code{Window} class in order to support the ability to
change a menu and later return to a previous menu.  The value returned
is the next menu in the stack.
@example
@group
@exdent Example:

object  myMenu, subMenu;
HANDLE  h;

subMenu = gPop(myMenu);
@end group
@end example
@sp 1
See also:  @code{Push, Use::Window, PopMenu::Window}
@end deffn






@deffn {Push} Push::Menu
@sp 2
@example
@group
r = gPush(top, psh);

object  top;    /*  top menu     */
object  psh;    /*  pushed menu  */
object  r;      /*  top menu     */
@end group
@end example
This method is used to push menu (@code{psh}) into a last in, first out
stack accessible through menu @code{top}.  It is mainly used by the
@code{Window} class in order to support the ability to change a menu and
later return to a previous menu.  The value returned is the top menu
passed.
@example
@group
@exdent Example:

object  myMenu, subMenu;
HANDLE  h;

h = gPush(myMenu, subMenu);
@end group
@end example
@sp 1
See also:  @code{Pop, Use::Window, PopMenu::Window}
@end deffn








@deffn {SetMode} SetMode::Menu
@sp 2
@example
@group
r = mSetMode(menu, id, mode);

object  menu;   /*  menu       */
unsigned  id;   /*  menu item  */
unsigned  mode; /*  item mode  */
object    r;    /*  menu       */
@end group
@end example
This method is used to set the mode associated with a menu item.  This
mode specifies whether the menu is enabled, disabled or grayed.

In the case of external menus, @code{id} is normally a macro created by
the resource editor while the programmer defines the menu and identifies
a particular menu option.  In the case of internal menu's, @code{id}
is the number returned by @code{AddMenuOption::InternalMenu} or
@code{AddMenuOption::PopupMenu}.

@code{mode} is a Windows defined macro and may be one of the following:
@example
@group
MF_ENABLED      to enable the menu item
MF_DISABLED     to disable the menu item
MF_GRAYED       to gray the menu option
@end group
@end example
The value returned is the menu passed.
@example
@group
@exdent Example:

object  myMenu;

mSetMode(myMenu, ID_FILE_OPEN, MF_DISABLED);
@end group
@end example
@sp 1
See also:  @code{AddMenuOption::InternalMenu, AddMenuOption::PopupMenu}
@end deffn






@subsection External Menus
The @code{ExternalMenu} class is used to access menus which were defined
using the resource editor.  Since this class is a subclass of @code{Menu},
all of @code{Menu}'s functionality is accessible through instance of this
class.  Therefore, see the @code{Menu} class for additional functionality.







@deffn {Load} Load::ExternalMenu
@sp 2
@example
@group
menu = mLoad(ExternalMenu, id);

unsigned  id;   /*  menu id  */
object  menu;   /*  menu     */
@end group
@end example
This method is used to create a new menu object representing a menu defined
with the resource editor.

@code{id} is normally a macro created by the resource editor which identifies
a particular menu.

The value returned is a new object representing the menu or @code{NULL} if
the menu identified is not found.
@example
@group
@exdent Example:

object  myMenu;

myMenu = mLoad(ExternalMenu, MY_MENU);
@end group
@end example
@sp 1
See also:  @code{LoadStr, LoadMenu::Window, LoadMenuStr::Window}
@end deffn







@deffn {LoadStr} LoadStr::ExternalMenu
@sp 2
@example
@group
menu = gLoadStr(ExternalMenu, id);

char    *id;    /*  menu id  */
object  menu;   /*  menu     */
@end group
@end example
This method is used to create a new menu object representing a menu defined
with the resource editor.

@code{id} is a string which names the desired menu.  This name is
defined by the resource editor and identifies a particular menu.

The value returned is a new object representing the menu or @code{NULL} if
the menu identified is not found.
@example
@group
@exdent Example:

object  myMenu;

myMenu = gLoadStr(ExternalMenu, "mymenu");
@end group
@end example
@sp 1
See also:  @code{Load, LoadMenu::Window, LoadMenuStr::Window}
@end deffn








@subsection Internal Menus
The @code{InternalMenu} class is used to enable an application to create
menus on the fly -- without the need to pre-specify the menu's structure
with the resource editor.  Since this class is a subclass of
@code{Menu}, all of @code{Menu}'s functionality is accessible through
instance of this class.  Therefore, see the @code{Menu} class for
additional functionality.








@deffn {AddMenuOption} AddMenuOption::InternalMenu
@sp 2
@example
@group
id = gAddMenuOption(mnu, ttl, fun)

object  menu;   /*  menu     */
char    *ttl;   /*  title    */
long    (*fun)(); /*  function  */
int     id;     /*  item id  */
@end group
@end example
This method is used to append a new option to internal menu object
@code{mnu} and associate it to an application specific function
(@code{fun}).  Once this is done, if the user selects the new menu
option the function @code{fun} will be executed.

@code{ttl} represents the character string which will be displayed
in the menu.  @code{ttl} may also contain an embedded ampersand (&)
character.  This causes the character following the ampersand to
be underlined.  It also causes the following character to be the
Alt key selection the user can use to quickly select a menu option.

@code{fun} is the function which will be executed when the user selects
the menu option and has the following form:
@example
@group
long    fun(object wind)
@{
        .
        .
        .
        return 0L;
@}
@end group
@end example
The function executed (@code{fun}) is passed the window object and
returns a long.  The return value is documented in the Windows documentation
under the message named @code{WM_COMMAND}.  It should normally be @code{0L}.

This method returns a unique WDS generated id which identifies this
particular menu option and may be used to control the status of the option.
@example
@group
@exdent Example:

static  long    file_message(object wind)
@{
        gMessage(wind, "File_Message");
        return 0L;
@}

        .
        .
        mnu = vNew(InternalMenu);
        gAddMenuOption(mnu, "&File", file_message);
@end group
@end example
@sp 1
See also:  @code{AddPopupMenu}
@end deffn











@deffn {AddPopupMenu} AddPopupMenu::InternalMenu
@sp 2
@example
@group
id = gAddPopupMenu(mnu, ttl, pm)

object  mnu;    /*  menu        */
char    *ttl;   /*  title       */
object  pm;     /*  Popup menu  */
object  r;      /*  menu        */
@end group
@end example
This method is used to append a new option to menu object @code{mnu} and
associate it to a popup menu.  Once this is done, if the user selects
the new menu option the popup menu will be displayed.

@code{ttl} represents the character string which will be displayed
in the menu.  @code{ttl} may also contain an embedded ampersand (&)
character.  This causes the character following the ampersand to
be underlined.  It also causes the following character to be the
Alt key selection the user can use to quickly select a menu option.

@code{pm} is a popup menu object created with the @code{PopupMenu}
class.

This method returns @code{mnu}.
@example
@group
@exdent Example:

object  mnu, pm;

mnu = vNew(InternalMenu);
pm  = vNew(PopupMenu, mnu);
gAddPopupMenu(mnu, "&File", pm);
@end group
@end example
@sp 1
See also:  @code{AddMenuOption, New::PopupMenu}
@end deffn








@deffn {New} New::InternalMenu
@sp 2
@example
@group
menu = vNew(InternalMenu);

object  menu;   /*  menu     */
@end group
@end example
This method is used to create a new internal menu.  An internal menu is
one which is not pre-specified with the resource editor.  It may therefore
be structured at runtime.

The object returned is a new internal menu object with no items associated
with it.
@example
@group
@exdent Example:

object  myMenu;

myMenu = vNew(InternalMenu);
@end group
@end example
@sp 1
See also:  @code{AddMenuOption, AddMenu, Use::Window, LoadMenu::Window}
@end deffn





@section Popup Menus
The @code{PopupMenu} class is used in conjunction with the
@code{InternalMenu} class in order to provide a nested menu structure.
Nesting of popup menus is supported to arbitrary levels.








@deffn {AddMenuOption} AddMenuOption::PopupMenu
@sp 2
@example
@group
id = gAddMenuOption(mnu, ttl, fun)

object  menu;   /*  menu     */
char    *ttl;   /*  title    */
long    (*fun)(); /*  function  */
int     id;     /*  item id  */
@end group
@end example
This method works exactly like @code{AddMenuOption::InternalMenu}.
See that description for full documentation.
@c @example
@c @group
@c @exdent Example:
@c @end group
@c @end example
@sp 1
See also:  @code{AddPopupMenu, AddSeparator}
@end deffn









@deffn {AddPopupMenu} AddPopupMenu::PopupMenu
@sp 2
@example
@group
id = gAddPopupMenu(mnu, ttl, pm)

object  mnu;    /*  menu        */
char    *ttl;   /*  title       */
object  pm;     /*  Popup menu  */
object  r;      /*  menu        */
@end group
@end example
@code{mnu} is the popup menu which is to act as the parent of @code{pm}
and may be arbitrarily nested.

This method is the same as @code{AddPopupMenu::InternalMenu}.  See that
method for complete documentation.
@c @example
@c @group
@c @exdent Example:
@c @end group
@c @end example
@sp 1
See also:  @code{AddMenuOption, AddSeparator}
@end deffn










@deffn {AddSeparator} AddSeparator::PopupMenu
@sp 2
@example
@group
r = gAddSeparator(mnu)

object  mnu;    /*  menu        */
object  r;      /*  menu        */
@end group
@end example
This method is used to append a horizontal bar to the end of a popup
menu.  Any menu items added subsequently will appear after the
horizontal separator.

The menu object passed is returned.
@example
@group
@exdent Example:

gAddSeparator(pm);
@end group
@end example
@sp 1
See also:  @code{AddMenuOption}
@end deffn











@deffn {Dispose} Dispose::PopupMenu
@sp 2
@example
@group
r = gDispose(menu);

object  menu;   /*  a menu object   */
object  r;      /*  NULL            */
@end group
@end example
This method is used to dispose of a menu object when it is no longer
needed.  It is not often needed because popup menus associated with
other menus are automatically disposed of (via @code{DeepDispose}) when
the top level menu is disposed.

This class also provides a @code{DeepDispose} which performs the same
function.

The value returned is always @code{NULL} and may be used to null out
the variable which contained the object being disposed in order to
avoid future accidental use.
@example
@group
@exdent Example:

object  myMenu;

myMenu = gDispose(myMenu);
@end group
@end example
@sp 1
See also:  @code{DeepDispose::Menu}
@end deffn








@deffn {Handle} Handle::PopupMenu
@sp 2
@example
@group
h = gHandle(menu);

object  menu;   /*  menu object    */
HANDLE  h;      /*  Windows handle */
@end group
@end example
This method is used to obtain the Windows internal handle associated with
a popup menu object.  

Note that this method may be used with most WDS objects in order to obtain
the internal handle that Windows normally associates with each type of object.
See the appropriate documentation.
@example
@group
@exdent Example:

object  myMenu;
HANDLE  h;

h = gHandle(myMenu);
@end group
@end example
@c @sp 1
@c See also:  @code{Message}
@end deffn








@deffn {MenuFunction} MenuFunction::PopupMenu
@sp 2
@example
@group
fun = mMenuFunction(menu, id);

object  menu;   /*  a menu object     */
int     id;     /*  menu item id      */
long    (*fun)(object);  /*  function to execute  */
@end group
@end example
This method is used to obtain the function which was associated with a
menu item.
@c @example
@c @group
@c @exdent Example:
@c @end group
@c @end example
@sp 1
See also:  @code{AddMenuOption}
@end deffn








@deffn {New} New::PopupMenu
@sp 2
@example
@group
pm = vNew(PopupMenu, tm);

object  tm;     /*  top menu    */
object  pm;     /*  popup menu  */
@end group
@end example
This method is used to create a new popup menu.  A popup menu is one
which pops up or appears when a user selects an associated menu option.
Popup menus are used in conjunction with internal menus in order to
create arbitrarily complex menu structures.

@code{tm} is the top menu object.  It must be an @code{InternalMenu},
regardless of where the new popup menu will be nested.

The object returned represents the new popup menu created and may be
used to add items to that menu.
@example
@group
@exdent Example:

object  myMenu, pm;

myMenu = vNew(InternalMenu);
pm = vNew(PopupMenu, myMenu);
gAddPopupMenu(myMenu, "&File", pm);
@end group
@end example
@sp 1
See also:  @code{AddMenuOption, AddPopupMenu, AddSeparator,}
@iftex
@hfil @break @hglue .63in     
@end iftex
@code{AddPopupMenu::InternalMenu}
@end deffn







@section Dialogs
The @code{Dialog} class implements all the functionality which is common
to the @code{ModalDialog} and @code{ModelessDialog} classes, and although
this class is not used directly, most of their functionality is documented
in this section.

@subsection  Standard Dialog Method Arguments

Since all dialog instance methods have the dialog object as the first
argument (referred to as @code{dlg}), this argument will not be described
each time.  It always refers to the dialog object which you wish to
perform the desired operation on.

Many of the methods associated with this class take an argument
identified as @code{id}.  This is a macro defined by the programmer via
the resource editor when the dialog is being defined and is used to
uniquely identify a particular control within the dialog.  Due to the
fact that this argument has the same meaning for every method which
uses it, it will not be defined each time.

The variable @code{ctl} will always refer to a control object.  Each
control object represents a unique control within a dialog.  Each
control object will be an instance of one of the subclasses of the
@code{Control} class (such as @code{TextControl} or
@code{NumericControl}).



@subsection Dialog Methods






@deffn {AddControl} AddControl::Dialog
@sp 2
@example
@group
ctl = mAddControl(dlg, ctlClass, id);

object  dlg;       /*  a dialog object   */
object  ctlClass;  /*  class of control  */
unsigned  id;      /*  control id        */
object  ctl;       /*  control object    */
@end group
@end example
This method is used to create a new control object and associate it to
a control defined in the dialog.  @code{ctlClass} must be literally
one of the subclasses of the @code{Control} class (such as
@code{TextControl}, @code{CheckBox}, @code{RadioButton}, etc).
See the class hierarchy for a complete list.

@code{id} identifies which control within dialog @code{dlg} to associate
the new control object with.

The value returned (@code{ctl}) is the new control object created and
will be an instance of the class indicated by the @code{ctlClass}
argument.  This object can be configured and queried via the protocol
defined by its class.  See that section for available options.  Setting
various options associated with this object will control default values
associated with the control and how this control functions within the
dialog.
@example
@group
@exdent Example:

object  dlg, ctl;

ctl = mAddControl(dlg, TextControl, PERSONS_NAME);
@end group
@end example
@sp 1
See also:  @code{GetControl}
@end deffn






@deffn {AddDlgHandlerAfter} AddDlgHandlerAfter::Dialog
@sp 2
@example
@group
r = gAddDlgHandlerAfter(dlg, msg, func);

object   dlg;      /*  a dialog object  */
unsigned msg;      /*  message          */
BOOL    (*func)(); /*  function pointer */
object  r;         /*  the dialog obj   */
@end group
@end example
This method is used to associate function @code{func} with Windows dialog
message @code{msg} for dialog @code{dlg}.  Whenever dialog @code{dlg}
receives message @code{msg}, @code{func} will be called.

@code{dlg} is the dialog object who's messages you wish to process.
@code{msg} is the particular message you wish to trap.  These messages
are fully documented in the Windows documentation in the Messages
section.  They normally begin with @code{WM_}.

@code{func} is the function which gets called whenever the specified
message gets received and takes the following form:
@example
@group
BOOL    func(object     dlg,
             HWND       hdlg, 
             UINT       mMsg, 
             WPARAM     wParam, 
             LPARAM     lParam)
@{
        .
        .
        .
        return FALSE;  /* or whatever is appropriate  */
@}
@end group
@end example
Where @code{dlg} is the dialog being sent the message.  The remaining
arguments and return value is fully documented in the Windows documentation
under the @code{DialogProc} function and the Windows Messages documentation.

WDS keeps a list of functions associated with each message associated
with each dialog.  When a particular message is received the appropriate
list of handler functions gets executed sequentially.
@code{AddDlgHandlerAfter} appends the new function to the end of this list,
and @code{AddDlgHandlerBefore} adds the new function to the beginning of
the list.

Windows will only see the return value of the last message handler executed.
@example
@group
@exdent Example:

int     hSize, vSize;

static  BOOL    process_wm_size(object  dlg, 
                                HWND    hwnd, 
                                UINT    mMsg, 
                                WPARAM  wParam, 
                                LPARAM  lParam)
@{
        hSize = LOWORD(lParam);
        vSize = HIWORD(lParam);
        return 0L;
@}

        .
        .
        gAddDlgHandlerAfter(dlg, (unsigned) WM_SIZE,
                                      process_wm_size);
        .
        .
@end group
@end example
@sp 1
See also:  @code{AddDlgHandlerBefore}
@end deffn












@deffn {AddDlgHandlerBefore} AddDlgHandlerBefore::Dialog
@sp 2
@example
@group
r = gAddDlgHandlerBefore(dlg, msg, func);

object   dlg;     /*  a dialog object  */
unsigned msg;      /*  message          */
BOOL    (*func)(); /*  function pointer */
object  r;         /*  the dialog obj   */
@end group
@end example
This function is fully documented under @code{AddDlgHandlerAfter}.
@c @example
@c @group
@c @exdent Example:
@c @end group
@c @end example
@sp 1
See also:  @code{AddDlgHandlerAfter}
@end deffn













@deffn {AutoDispose} AutoDispose::Dialog
@sp 2
@example
@group
pflg = gAutoDispose(dlg, flg);

object  dlg;    /*  a dialog object    */
int     flg;    /*  desired mode       */
int     pflg;   /*  previous mode set  */
@end group
@end example
This method is used to control the auto disposal facility associated with
a given dialog.  The default is disabled for modal dialogs and enabled
for modeless dialogs.

If enabled, the auto dispose feature will cause the dialog object
(@code{dlg}) to be automatically disposed of whenever the user accepts
or aborts the dialog.  If disabled, the application may continue to use
the object (for querying control values for example) after the dialog
has been terminated.  If the auto dispose feature is disabled the
application must explicitly dispose of the object when it is no longer
needed.

Note that if the auto dispose feature is used, the @code{gCompletionFunction}
method may be used to access the control values immediately prior to
the object's automatic disposal.

Set @code{flg} to 1 to enable the feature and 0 to disable it.  The
previous state will be returned.
@example
@group
@exdent Example:

object  dlg;

gAutoDispose(dlg, 1);
@end group
@end example
@sp 1
See also:  @code{CompletionFunction, Perform, SetResult}
@end deffn














@deffn {BackBrush} BackBrush::Dialog
@sp 2
@example
@group
r = gBackBrush(dlg, brsh);

object  dlg;    /*  a dialog object   */
object  brsh;   /*  brush object      */
object  r;      /*  dlg               */
@end group
@end example
This method is used to set the brush object which is used for the
background of the dialog.  Any previously associated brush object will
be disposed.  This brush object will also be automatically disposed when
the dialog is disposed.

The dialog passed is returned.
@example
@group
@exdent Example:

object  dlg;

gBackBrush(dlg, vNew(SolidBrush, 0, 255, 0));
@end group
@end example
@sp 1
See also:  @code{TextBrush, SetBackBrush::Application}
        and the @code{Brush} classes
@end deffn










@deffn {CompletionFunction} CompletionFunction::Dialog
@sp 2
@example
@group
r = gCompletionFunction(dlg, fun);

object  dlg;       /*  dialog object        */
int     (*fun)();  /*  completion function  */
object  r;         /*  dialog object        */
@end group
@end example
This method is used to associated a C function with a dialog such that
when the user accepts or aborts the dialog, function @code{fun} will
get executed by WDS.  The function will get executed prior to any automatic
disposal of the dialog object.

The completion function facility is best used in conjunction with
modeless dialogs and the auto dispose facility in order to process
control values when the user accepts the dialog and prior to the
automatic disposal of the dialog object.

The format of @code{fun} is as follows:
@example
@group
r = fun(dlg, res);

object  dlg;    /*  the dialog object  */
int     res;    /*  result status of the dialog  */
int     r;      /*  result for gPerform  */
@end group
@end example
Where @code{res} will be @code{TRUE} if the user accepted the dialog or
@code{FALSE} if the user canceled the dialog.  @code{r} is the value
which will become the result of the dialog which is returned by @code{gPerform}
(if it was a modal dialog).  @code{r} will normally be @code{res}.
@example
@group
@exdent Example:

static  int     fun(object dlg, int res)
@{
        .
        .
        return res;
@}

        gCompletionFunction(dlg, fun);
@end group
@end example
@sp 1
See also:  @code{AutoDispose, Perform, SetTag, SetResult}
@end deffn














@deffn {CtlDoubleValue} CtlDoubleValue::Dialog
@sp 2
@example
@group
val = mCtlDoubleValue(dlg, id);

object  dlg;    /*  a dialog object  */
unsigned id;    /*  control id       */
double  val;    /*  control value    */
@end group
@end example
This method is used to gain access to the value associated with a
particular control as a C double.  For example, if the control
was a numeric control and the user entered ``3.141'', then this method
would return a double ``3.141''.
@example
@group
@exdent Example:

object  dlg;
double  val;

val = mCtlDoubleValue(dlg, PI_VALUE);
@end group
@end example
@sp 1
See also:  @code{CtlStringValue, CtlShortValue, ...}
        @code{CtlValue}
@end deffn

















@deffn {CtlLongValue} CtlLongValue::Dialog
@sp 2
@example
@group
val = mCtlLongValue(dlg, id);

object  dlg;    /*  a dialog object  */
unsigned id;    /*  control id       */
long    val;    /*  control value    */
@end group
@end example
This method is used to gain access to the value associated with a
particular control as a C long.  For example, if the control
was a numeric control and the user entered ``6'', then this method
would return a long ``6''.

In the case of date controls, the value returned will be in the
form YYYYMMDD.  For example November 24, 1956 would be
19561124.
@example
@group
@exdent Example:

object  dlg;
long    val;

val = mCtlLongValue(dlg, PERSONS_AGE);
@end group
@end example
@sp 1
See also:  @code{CtlStringValue, CtlShortValue, ...}
        @code{CtlValue}
@end deffn














@deffn {CtlShortValue} CtlShortValue::Dialog
@sp 2
@example
@group
val = mCtlShortValue(dlg, id);

object  dlg;    /*  a dialog object  */
unsigned id;    /*  control id       */
short   val;    /*  control value    */
@end group
@end example
This method is used to gain access to the value associated with a
particular control as a C short.  For example, if the control
was a numeric control and the user entered ``6'', then this method
would return a short ``6''.

In the case of a list box or combo box the value returned will be in
terms of an ordinal value. This value returned is a zero based index
from top to bottom indicating the selection made by the user.  If no
selection was made negative value will be returned.  

In the case of check boxes and radio buttons this method will return
0 if the button is not selected, 1 if the button is selected, and
2 if the button is grayed.

In the case of scroll bars, this method will return a short value
between the minimum and maximum set for the control which indicates
where the control was set to.

@code{mCtlLongValue} should be used for date controls.
@example
@group
@exdent Example:

object  dlg;
short   val;

val = mCtlShortValue(dlg, PERSONS_AGE);
@end group
@end example
@sp 1
See also:  @code{CtlStringValue, CtlLongValue, ...}
        @code{CtlValue}
@end deffn















@deffn {CtlStringValue} CtlStringValue::Dialog
@sp 2
@example
@group
val = mCtlStringValue(dlg, id);

object  dlg;    /*  a dialog object  */
unsigned id;    /*  control id       */
char    *val;   /*  control value    */
@end group
@end example
This method is used to gain access to the value associated with a
particular control as a C character string.  For example, if the control
was a text control and the user entered ``Miami'', then this method
would return a pointer to the C string ``Miami''.

If the control being accessed is a list box or combo box, the value
returned will be a string representation of the text associated with the
choice made by the user.  See @code{CtlShortValue} or @code{IndexValue}
to get an ordinal value for this control type.

The character pointer returned will not be valid once the dialog or
control has been disposed.  Therefore, the value should be kept by
some other means if it is desired past the life of the dialog.
@example
@group
@exdent Example:

object  dlg;
char    *val;

val = mCtlStringValue(dlg, PERSONS_NAME);
@end group
@end example
@sp 1
See also:  @code{CtlShortValue, CtlLongValue, ...}
        @code{CtlValue}
@end deffn











@deffn {CtlUnsignedShortValue} CtlUnsignedShortValue::Dialog
@sp 2
@example
@group
val = mCtlUnsignedShortValue(dlg, id);

object  dlg;    /*  a dialog object  */
unsigned id;    /*  control id       */
unsigned short  val;    /*  control value    */
@end group
@end example
This method is used to gain access to the value associated with a
particular control as a C unsigned short.  For example, if the control
was a numeric control and the user entered ``6'', then this method
would return a unsigned short ``6''.

@code{mCtlLongValue} should be used for date controls.
@example
@group
@exdent Example:

object  dlg;
unsigned short  val;

val = mCtlUnsignedShortValue(dlg, PERSONS_AGE);
@end group
@end example
@sp 1
See also:  @code{CtlStringValue, CtlLongValue, ...}
        @code{CtlValue}
@end deffn














@deffn {CtlValue} CtlValue::Dialog
@sp 2
@example
@group
val = mCtlValue(dlg, id);

object  dlg;    /*  a dialog object  */
unsigned id;    /*  control id       */
object  val;    /*  control value    */
@end group
@end example
This method is used to gain access to the value associated with a
particular control as a Dynace object.  For example, if the control
was a text control and user entered ``Miami'', then this method
would return a Dynace string object (not a normal C string) which
represents the value typed in.  This returned value may be used
and accessed like any other Dynace object using the appropriate
interface mechanism.

If the control being accessed is a list box or combo box, the value
returned will be a representation of the text associated with the choice
made by the user.  See @code{IndexValue} or @code{CtlShortValue} to get
a numeric representation for this control type.

The object returned will be automatically disposed when the dialog is
disposed, therefore, a copy (via @code{gCopy}) should be made of it if
its existence is required subsequent to the dialog's disposal.

WDS also provides several methods to obtain normal C data types directly.
@example
@group
@exdent Example:

object  dlg, val;

val = mCtlValue(dlg, PERSONS_NAME);
@end group
@end example
@sp 1
See also:  @code{CtlStringValue, CtlShortValue, CtlLongValue, ...}
@end deffn












@deffn {DeepDispose} DeepDispose::Dialog
@sp 2
This method performs the same function as @code{Dispose}.  See that
method for details.
@end deffn









@deffn {Dispose} Dispose::Dialog
@sp 2
@example
@group
r = gDispose(dlg);

object  dlg;   /*  a dialog object   */
object  r;     /*  NULL              */
@end group
@end example
This method is used to remove and dispose of a dialog object when it
is no longer needed.  This method should be called on all dialogs
when they are no longer needed.  In addition to disposing of the dialog,
this method will also dispose of all the control objects associated
with the dialog.

The value returned is always @code{NULL} and may be used to null out
the variable which contained the object being disposed in order to
avoid future accidental use.
@example
@group
@exdent Example:

object  dlg;

dlg = gDispose(dlg);
@end group
@end example
@sp 1
See also:  @code{AutoDispose}
@end deffn










@deffn {GetBackBrush} GetBackBrush::Dialog
@sp 2
@example
@group
brsh = gGetBackBrush(dlg);

object  dlg;    /*  dialog object  */
object  brsh;   /*  brush object   */
@end group
@end example
This method is used to obtain the background brush object which has been
associated with the dialog.  If no specific background brush was
associated with the dialog this method will return the application
default background brush, which is what the dialog uses when no specific
brush is specified.
@example
@group
@exdent Example:

object  dlg, brsh;

brsh = gGetBackBrush(dlg);
@end group
@end example
@sp 1
See also:  @code{BackBrush, GetTextBrush, GetBackBrush::Application}
@end deffn









@deffn {GetControl} GetControl::Dialog
@sp 2
@example
@group
ctl = mGetControl(dlg, id);

object  dlg;    /*  a dialog object  */
unsigned id;    /*  control id       */
object  ctl;    /*  control object   */
@end group
@end example
This method is used to gain access to a control object which has been
previously associated with a particular control via its associated
control id.  The control object is returned.
@example
@group
@exdent Example:

object  dlg, ctl;

ctl = mGetControl(dlg, PERSONS_NAME);
@end group
@end example
@sp 1
See also:  @code{AddControl}
@end deffn












@deffn {GetParent} GetParent::Dialog
@sp 2
@example
@group
prnt = gGetParent(dlg);

object  dlg;    /*  dialog object         */
object  prnt;   /*  parent window object  */
@end group
@end example
This method is used to obtain the parent window object associated
with dialog @code{dlg}.  This parent window would have established
when the dialog object was created.
@example
@group
@exdent Example:

object  dlg, parentWind;

parentWind = gGetParent(dlg);
@end group
@end example
@sp 1
See also:  @code{NewDialog::ModalDialog, NewDialog::ModelessDialog}
@end deffn















@deffn {GetTag} GetTag::Dialog
@sp 2
@example
@group
r = gGetTag(dlg);

object  dlg;   /*  dialog object  */
object  r;     /*  tag            */
@end group
@end example
This method is used to obtain a Dynace object which has been associated
with a dialog via @code{SetTag}.  The value return is the object which has
been associated with the dialog object @code{dlg}.  If there is no object
associated with the dialog, @code{NULL} will be returned.
@example
@group
@exdent Example:

object  dlg, someObj;

someObj = gGetTag(dlg);
@end group
@end example
@sp 1
See also:  @code{SetTag, SetTag::Window}
@end deffn










@deffn {GetTextBrush} GetTextBrush::Dialog
@sp 2
@example
@group
brsh = gGetTextBrush(dlg);

object  dlg;    /*  dialog object  */
object  brsh;   /*  brush object   */
@end group
@end example
This method is used to obtain the text brush object which has been associated
with the dialog.  If no specific text brush was associated with the dialog
this method will return the application default text brush, which is what
the dialog uses when no specific brush is specified.
@example
@group
@exdent Example:

object  dlg, brsh;

brsh = gGetTextBrush(dlg);
@end group
@end example
@sp 1
See also:  @code{TextBrush, GetBackBrush, GetTextBrush::Application}
@end deffn











@deffn {Handle} Handle::Dialog
@sp 2
@example
@group
h = gHandle(dlg);

object  dlg;    /*  dialog object   */
HANDLE  h;      /*  Windows handle  */
@end group
@end example
This method is used to obtain the Windows internal handle associated with
a dialog object.  It will only return a valid handle while the dialog
is being performed via @code{gPerform}.  @code{NULL} will be returned
otherwise.

Note that this method may be used with most WDS objects in order to obtain
the internal handle that Windows normally associates with each type of object.
See the appropriate documentation.
@example
@group
@exdent Example:

object  dlg;
HANDLE  h;

h = gHandle(dlg);
@end group
@end example
@c @sp 1
@c See also:  @code{Message}
@end deffn








@deffn {IndexValue} IndexValue::Dialog
@sp 2
@example
@group
val = mIndexValue(dlg, id);

object  dlg;    /*  a dialog object  */
unsigned id;    /*  control id       */
object  val;    /*  control value    */
@end group
@end example
This method is used to gain access to a Dynace object (as opposed to a
normal C integer) representing the value associated with a list box or
combo box in terms of an ordinal value. This value returned is a zero
based index from top to bottom indicating the selection made by the
user.  If no selection was made the object will represent a negative
value.  This returned value may be used and accessed like any other
Dynace object using the appropriate interface mechanisms.

The object returned will be automatically disposed when the dialog is
disposed, therefore, a copy (via @code{gCopy}) should be made of it if
its existence is required subsequent to the dialog's disposal.

WDS also provides several methods to obtain normal C data types directly.
@example
@group
@exdent Example:

object  dlg, val;

val = mIndexValue(dlg, SOME_LISTBOX);
@end group
@end example
@sp 1
See also:  @code{CtlShortValue, CtlValue}
@end deffn








@deffn {InDialog} InDialog::Dialog
@sp 2
@example
@group
flg = gInDialog(dlg);

object  dlg;    /*  a dialog object  */
int     flg;    /*  in dialog flag   */
@end group
@end example
This method is used to determine whether a given dialog is currently
being ``performed'' via @code{gPerform}.  Prior and subsequent to
performing a dialog this method will return 0.  A 1 will be returned
if the dialog is currently being performed.
@example
@group
@exdent Example:

object  dlg;
int     flg;

flg = gInDialog(dlg);
@end group
@end example
@sp 1
See also:  @code{Perform}
@end deffn















@deffn {Message} Message::Dialog
@sp 2
@example
@group
r = gMessage(dlg, msg);

object  dlg;    /*  dialog object  */
char    *msg;   /*  message        */
object  r;      /*  dialog object  */
@end group
@end example
This method is used to open up a temporary informational window.  The
window will contain the message given by @code{msg} and the user must
acknowledge the window prior to continuing by hitting an OK button.

This method may only be used either while the dialog is being performed
or at any time if it has an associated parent window.

The value returned is the dialog passed.
@example
@group
@exdent Example:

object  dlg;

gMessage(dlg, "Press OK to continue.");
@end group
@end example
@sp 1
See also:  @code{MessageWithTopic}
@end deffn












@deffn {MessageWithTopic} MessageWithTopic::Dialog
@sp 2
@example
@group
r = gMessageWithTopic(dlg, msg, tpc);

object  dlg;    /*  dialog object  */
char    *msg;   /*  message        */
char    *tpc;   /*  help topic     */
object  r;      /*  dialog object  */
@end group
@end example
This method is used to open up a temporary informational dialog.  The
dialog will contain the message given by @code{msg} and the user must
acknowledge the dialog prior to continuing by hitting an OK button.

If the user hits the @code{F1} key while presented with the message,
the help topic identified by @code{tpc} will get displayed via the Windows
help system.

This method may only be used either while the dialog is being performed
or at any time if it has an associated parent window.

The value returned is the dialog passed.
@example
@group
@exdent Example:

object  dlg;

gMessageWithTopic(dlg, "Press OK to continue.", "mytopic");
@end group
@end example
@sp 1
See also:  @code{Message} and the @code{HelpSystem} class.
@end deffn














@deffn {Perform} Perform::Dialog
@sp 2
@example
@group
r = gPerform(dlg);

object  dlg;    /*  a dialog object  */
int     r;      /*  result           */
@end group
@end example
Once a dialog is created and fully specified, this method is used to
actually display the dialog and allow the user to interact with it.
The dialog will remain active until the user pushes the push buttons
with the label @code{IDOK} or @code{IDCANCEL}.

What occurs after calling @code{gPerform} depends on whether the dialog
was a modal or modeless dialog.

Control initial values should be set prior to calling @code{gPerform}.

MODAL DIALOGS

If the dialog being performed is a modal, then once @code{gPerform} is
evoked the user will be able to interact with the dialog and @code{gPerform}
will not return until the user either accepts or aborts the entire dialog.
This, in effect, will prevent the user from switching to any other window
or dialog within the application until the dialog is completed.

Finally, the value returned by @code{gPerform} will be the Windows
constant @code{FALSE} if the user canceled the dialog, the Windows
constant @code{TRUE} if the user accepted the dialog, or the
value returned by any completion function which was attached to the
dialog (via @code{CompletionFunction}).

Once @code{gPerform} returns, the dialog object may be queried in
order to obtain the final values associated with each control.

Unless the auto dispose feature is manually set for a modal dialog,
the dialog object is not automatically disposed and must be when
no longer needed.

MODELESS DIALOGS

Modeless dialogs operate different from modal dialogs.  When
@code{gPerform} is evoked on a modeless dialog, @code{gPerform} returns
immediately with a 0 value.  At that point the user is presented with the
dialog and allowed to edit it.  During that time your program would typically
return to the menu and wait for further user selections.  The user may
continue with the current dialog, abort it, or switch to another window
within the application with the ability to return at any point.  All the
user interaction is automatically handled by Windows and WDS.

Since the application code has gone on about its business after a call
to @code{gPerform} two things are left to be completed whenever the user
finally accepts or aborts the dialog.  First, the application will
typically want to do something with the data the user has entered on the
dialog.  And second, the dialog object will need to be disposed when
it's no longer needed.

When using modeless dialogs the programmer normally attaches a completion
function to the dialog via @code{gCompletionFunction}.  What this does
is cause the completion function to be executed whenever the user
accepts or aborts the dialog.  This application specific function can be
used to perform the final processing of the dialog's data.  See
@code{gCompletionFunction} for complete details.

By default, the auto dispose flag associated with modeless dialogs is
normally enabled.  This causes WDS to automatically dispose of a dialog
object and all associated controls immediately subsequent to executing
the completion function.  This way everything is cleaned up automatically.
This feature may be disabled via @code{gAutoDispose}, in which case the
application must be sure to dispose of the unneeded object.
@example
@group
@exdent Example:

object  dlg;
int     r;

r = gPerform(dlg);
@end group
@end example
@sp 1
See also:  @code{CompletionFunction, AutoDispose}
@end deffn













@deffn {SetResult} SetResult::Dialog
@sp 2
@example
@group
r = gSetResult(dlg, obj);

object  dlg;    /*  a dialog object  */
object  obj;    /*  result object    */
object  r;      /*  dlg              */
@end group
@end example
This method is used to associated a Dynace short integer object with a
dialog such that when the dialog is completed or canceled by the user,
the associated object will be set to the dialog's result value.  This
feature is most often needed in conjunction with modeless dialogs and
the auto dispose feature.  It can be used to find out what happened to a
dialog subsequent to the dialog object being disposed.

The object associated with the dialog will never be disposed by WDS
and must be manually disposed when it is no longer needed.
@example
@group
@exdent Example:

object  dlg, obj;

gSetResult(dlg, obj = gNewWithInt(ShortInteger, -1));
@end group
@end example
@sp 1
See also:  @code{AutoDispose, Perform, SetTag}
@end deffn










@deffn {SetTag} SetTag::Dialog
@sp 2
@example
@group
r = gSetTag(dlg, tag);

object  dlg;   /*  a dialog object   */
object  tag;    /*  tag               */
object  r;      /*  previous tag      */
@end group
@end example
This method is used to associated an arbitrary Dynace object with a
dialog object.  This may later be retrieved via the @code{GetTag} method.
Since WDS passes around the dialog object to all @code{Dialog} methods
this mechanism may be used to pass additional information with the
dialog.  And since Dynace treats all objects in a uniform manner, this
information attached to the dialog may be arbitrarily complex.

WDS does not dispose of the tag when the dialog object is disposed.
This method returns any previous object associated with the dialog or
@code{NULL}.
@example
@group
@exdent Example:

object  dlg;

gSetTag(dlg, gNewWithInt(ShortInteger, 17));
@end group
@end example
@sp 1
See also:  @code{GetTag, SetTag::Window}
@end deffn











@deffn {SetTopic} SetTopic::Dialog
@sp 2
@example
@group
pt = gSetTopic(dlg, tpc);

object  dlg;    /*  dialog object         */
char    *tpc;   /*  help topic            */
char    *pt;    /*  previous help topic   */
@end group
@end example
This method is used to associate help text with dialog @code{dlg}.
The help text is defined using the Windows help system and labeled
with the topic indicated by @code{tpc}.  Then, if the user hits
the F1 key while in the dialog, WDS will automatically bring up
the Windows help system and find the indicated topic.

WDS also supports window and control specific topics.  See the appropriate
sections.

This method returns any previous topic associated with the dialog.
@example
@group
@exdent Example:

object  dlg;

gSetTopic(dlg, "myDialogHelp");
@end group
@end example
@sp 1
See also:  The @code{HelpSystem} class and @code{SetTopic::Control} 
@end deffn















@deffn {TextBrush} TextBrush::Dialog
@sp 2
@example
@group
r = gTextBrush(dlg, brsh);

object  dlg;    /*  a dialog object   */
object  brsh;   /*  brush object      */
object  r;      /*  dlg               */
@end group
@end example
This method is used to set the brush object which is used for foreground
text which is displayed.  Any previously associated brush object will
be disposed.  This brush object will also be automatically disposed when the
dialog is disposed.

The dialog passed is returned.
@example
@group
@exdent Example:

object  dlg;

gTextBrush(dlg, vNew(SolidBrush, 255, 0, 0));
@end group
@end example
@sp 1
See also:  @code{BackBrush, SetTextBrush::Application}
        and the @code{Brush} classes
@end deffn



















@subsection Modal Dialogs
Modal dialogs are created with the methods described in this section.
However, since the @code{ModalDialog} class is a subclass of @code{Dialog},
and the @code{Dialog} class implements all the functionality common to
both the @code{ModalDialog} and @code{ModelessDialog} classes, the
majority of the functionality is inherited from and documented in
the @code{Dialog} class.  Therefore, see the @code{Dialog} class
for documentation on additional functionality available to this class.



@deffn {NewDialog} NewDialog::ModalDialog
@sp 2
@example
@group
dlg = mNewDialog(ModalDialog, id, wnd);

unsigned  id;   /*  dialog id      */
object    wnd;  /*  parent window  */
object    dlg;  /*  dialog object  */
@end group
@end example
This method is used to create a new modal dialog.  The dialog must have
been previously laid out using the resource editor.  The @code{id}
is a macro generated by the resource editor while the programmer defines
the dialog.  This id uniquely identifies a particular dialog.

@code{wnd} allows the specification of the parent window object associated
with the new dialog.  This is an optional parameter and may be @code{NULL}
if it not desired.  If you do specify a parent window, the dialog will
iconize when the window is iconized and will be disposed if the window
is disposed.

The object returned represents the new dialog created and may be
used to further control the dialog.
@example
@group
@exdent Example:

object  wind, dlg;

dlg = mNewDialog(ModalDialog, MY_DIALOG, wind);
@end group
@end example
@sp 1
See also:  @code{NewDialogStr, Perform::Dialog, NewDialog::ModelessDialog}
@end deffn










@deffn {NewDialogStr} NewDialogStr::ModalDialog
@sp 2
@example
@group
dlg = gNewDialogStr(ModalDialog, id, wnd);

char      *id;  /*  dialog id      */
object    wnd;  /*  parent window  */
object    dlg;  /*  dialog object  */
@end group
@end example
This method is used to create a new modal dialog.  The dialog must have
been previously laid out using the resource editor.  The @code{id} is
the string name associated with the dialog.  This id uniquely identifies
a particular dialog.

@code{wnd} allows the specification of the parent window object associated
with the new dialog.  This is an optional parameter and may be @code{NULL}
if it not desired.  If you do specify a parent window, the dialog will
iconize when the window is iconized and will be disposed if the window
is disposed.

The object returned represents the new dialog created and may be
used to further control the dialog.
@example
@group
@exdent Example:

object  wind, dlg;

dlg = gNewDialogStr(ModalDialog, "mydialog", wind);
@end group
@end example
@sp 1
See also:  @code{NewDialog, Perform::Dialog, NewDialogStr::ModelessDialog}
@end deffn









@subsection Modeless Dialogs
Modeless dialogs are created with the methods described in this section.
However, since the @code{ModelessDialog} class is a subclass of
@code{Dialog}, and the @code{Dialog} class implements all the
functionality common to both the @code{ModalDialog} and
@code{ModelessDialog} classes, the majority of the functionality is
inherited from and documented in the @code{Dialog} class.  Therefore,
see the @code{Dialog} class for documentation on additional
functionality available to this class.




@deffn {NewDialog} NewDialog::ModelessDialog
@sp 2
@example
@group
dlg = mNewDialog(ModelessDialog, id, wnd);

unsigned  id;   /*  dialog id      */
object    wnd;  /*  parent window  */
object    dlg;  /*  dialog object  */
@end group
@end example
This method is used to create a new modeless dialog.  The dialog must have
been previously laid out using the resource editor.  The @code{id}
is a macro generated by the resource editor while the programmer defines
the dialog.  This id uniquely identifies a particular dialog.

@code{wnd} allows the specification of the parent window object associated
with the new dialog.  This is an optional parameter and may be @code{NULL}
if it not desired.  If you do specify a parent window, the dialog will
iconize when the window is iconized and will be disposed if the window
is disposed.

When a new modeless dialog is created, its auto dispose mode is enabled.
This means that, unless disabled, the object representing the dialog
will be automatically disposed when the user closes the window.  This
is done due to the independent nature of modeless dialogs and the need
to be sure that object no longer needed are disposed.  Note that the
programmer may specify a function which gets executed prior to the
disposal of a modeless dialog in order to capture any data.

The object returned represents the new dialog created and may be
used to further control the dialog.
@example
@group
@exdent Example:

object  wind, dlg;

dlg = mNewDialog(ModelessDialog, MY_DIALOG, wind);
@end group
@end example
@sp 1
See also:  @code{NewDialogStr, Perform::Dialog, NewDialog::ModalDialog,}
@iftex
@hfil @break @hglue .63in 
@end iftex
@code{CompletionFunction::Dialog, AutoDispose::Dialog}
@end deffn












@deffn {NewDialogStr} NewDialogStr::ModelessDialog
@sp 2
@example
@group
dlg = gNewDialogStr(ModlessDialog, id, wnd);

char      *id;  /*  dialog id      */
object    wnd;  /*  parent window  */
object    dlg;  /*  dialog object  */
@end group
@end example
This method is used to create a new modeless dialog.  The dialog must have
been previously laid out using the resource editor.  The @code{id} is
the string name associated with the dialog.  This id uniquely identifies
a particular dialog.

@code{wnd} allows the specification of the parent window object associated
with the new dialog.  This is an optional parameter and may be @code{NULL}
if it not desired.  If you do specify a parent window, the dialog will
iconize when the window is iconized and will be disposed if the window
is disposed.

When a new modeless dialog is created, its auto dispose mode is enabled.
This means that, unless disabled, the object representing the dialog
will be automatically disposed when the user closes the window.  This
is done due to the independent nature of modeless dialogs and the need
to be sure that object no longer needed are disposed.  Note that the
programmer may specify a function which gets executed prior to the
disposal of a modeless dialog in order to capture any data.

The object returned represents the new dialog created and may be
used to further control the dialog.
@example
@group
@exdent Example:

object  wind, dlg;

dlg = gNewDialogStr(ModelessDialog, "mydialog", wind);
@end group
@end example
@sp 1
See also:  @code{NewDialog, Perform::Dialog, NewDialogStr::ModalDialog,}
@iftex
@hfil @break @hglue .63in 
@end iftex
@code{CompletionFunction::Dialog, AutoDispose::Dialog}
@end deffn






@section Controls
The @code{Control} class is never used directly.  It is used to group
functionality common to all the control types which are all subclasses
of this class.  Therefore, this section documents all the methods which
are accessible to all subclasses of this class.


Control instances are normally created via @code{AddControl::Dialog}
and may be subsequently accessed via @code{GetControl::Dialog}.



@subsection  Standard Control Method Arguments

Since all control instance methods have the control object as their
first argument (referred to as @code{ctl}), this argument will not be
described each time.  It always refers to the control object which you
wish to perform the desired operation on.  Each control object will be
an instance of one of the subclasses of the @code{Control} class (such
as @code{TextControl} or @code{NumericControl}).


Many of the methods associated with this class take an argument
identified as @code{id}.  This is a macro defined by the programmer via
the resource editor when the dialog is being defined and is used to
uniquely identify a particular control within the dialog.  Due to the
fact that this argument has the same meaning for every method which
uses it, it will not be defined each time.


@subsection Control Methods












@deffn {AddHandlerAfter} AddHandlerAfter::Control
@sp 2
@example
@group
r = gAddHandlerAfter(ctl, msg, func);

object   ctl;      /*  a control object  */
unsigned msg;      /*  message           */
long    (*func)(); /*  function pointer  */
object  r;         /*  the control obj   */
@end group
@end example
This method is used to associate function @code{func} with Windows control
message @code{msg} for control @code{ctl}.  Whenever control @code{ctl}
receives message @code{msg}, @code{func} will be called.

@code{ctl} is the control object who's messages you wish to process.
@code{msg} is the particular message you wish to trap.  These messages
are fully documented in the Windows documentation in the Messages
section.  They normally begin with @code{WM_}.

@code{func} is the function which gets called whenever the specified
message gets received and takes the following form:
@example
@group
long    func(object     ctl,
             HWND       hwnd, 
             UINT       mMsg, 
             WPARAM     wParam, 
             LPARAM     lParam)
@{
        .
        .
        .
        return 0L;  /* or whatever is appropriate  */
@}
@end group
@end example
Where @code{ctl} is the control being sent the message.  The remaining
arguments and return value is fully documented in the Windows documentation
under the @code{WindowProc} function and the Windows Messages documentation.

WDS keeps a list of functions associated with each message associated
with each control.  When a particular message is received the appropriate
list of handler functions gets executed sequentially.
@code{AddHandlerAfter} appends the new function to the end of this list,
and @code{AddHandlerBefore} adds the new function to the beginning of
the list.

WDS may also, and optionally, execute the Windows default procedure
associated with a given message either before or after the user added
list of functions.  This behavior may be controlled via
@code{DefaultProcessingMode}.

Windows will only see the return value of the last message handler executed
including, if applicable, the default.
@example
@group
@exdent Example:

int     hSize, vSize;

static  long    process_wm_size(object  ctl, 
                                HWND    hwnd, 
                                UINT    mMsg, 
                                WPARAM  wParam, 
                                LPARAM  lParam)
@{
        hSize = LOWORD(lParam);
        vSize = HIWORD(lParam);
        return 0L;
@}

        .
        .
        gAddHandlerAfter(ctl, (unsigned) WM_SIZE, process_wm_size);
        .
        .
@end group
@end example
@sp 1
See also:  @code{DefaultProcessingMode, AddHandlerBefore, CallDefaultProc}
@end deffn






@deffn {AddHandlerBefore} AddHandlerBefore::Control
@sp 2
@example
@group
r = gAddHandlerBefore(ctl, msg, func);

object   ctl;      /*  a control object  */
unsigned msg;      /*  message           */
long    (*func)(); /*  function pointer  */
object  r;         /*  the control obj   */
@end group
@end example
This function is fully documented under @code{AddHandlerAfter}.
@c @example
@c @group
@c @exdent Example:
@c @end group
@c @end example
@sp 1
See also:  @code{AddHandlerAfter}
@end deffn









@deffn {CallDefaultProc} CallDefaultProc::Control
@sp 2
@example
@group
r = gCallDefaultProc(ctl, msg, wp, lp)

object   ctl;   /*  a control object  */
unsigned msg;   /*  message           */
WPARAM   wp;    /*  wParam value      */
LPARAM   lp;    /*  lParam value      */
long     r;     /*  result of call    */
@end group
@end example
This method is used to explicitly call the default Windows message handler
associated with message @code{msg}.  This method is normally called from
within a programmer defined message handler (see @code{AddHandlerAfter})
which is provided with the @code{wp} and @code{lp} parameters.  These
parameters are fully documented in the Windows Messages documentation
and relate directly to the particular message.

@code{msg} is the particular message you wish to affect.  These messages
are fully documented in the Windows documentation in the Messages
section.  They normally begin with @code{WM_}.
@example
@group
@exdent Example:

gCallDefaultProc(ctl, msg, wParam, lParam);
@end group
@end example
@sp 1
See also:  @code{DefaultProcessingMode, AddHandlerAfter}
@end deffn















@deffn {CheckFunction} CheckFunction::Control
@sp 2
@example
@group
r = gCheckFunction(ctl, fun);

object  ctl;      /*  control object  */
int     (*fun)(); /*  check function  */
object  r;        /*  ctl object      */
@end group
@end example
This method is used to associate an auxiliary checking function to a control.
When a dialog is accepted, WDS goes through each control to assure the
validity of the data associated with each control as defined by the control
object.  The ability to associated an additional function for validating
the data associated with a control gives the programmer the ability to perform
any additional application specific checking against a control's data.

The auxiliary checking function takes the following form:
@example
@group
int     fun(object ctl, object val, char *buf)
@{
        ....
@}
@end group
@end example
Where @code{ctl} is the control object, and @code{val} is a Dynace object which
represents the value the control has associated with it.  @code{buf} is a
pointer to a buffer area which @code{fun} must set to an appropriate error
message if an error occurs.  @code{fun} returns a 1 if an error occurs and
0 otherwise.

All controls support this feature except the @code{PushButton} class,
because it is an immediate action control with no associated value.
@example
@group
@exdent Example:

object  ctl;

gCheckFunction(ctl, fun);
@end group
@end example
@sp 1
See also:  @code{GetControl::Dialog}
@end deffn















@deffn {CheckValue} CheckValue::Control
@sp 2
@example
@group
r = gCheckValue(ctl)

object   ctl;   /*  a control object  */
int      r;     /*  result of call    */
@end group
@end example
This method is used to explicitly cause all error checking functions and
parameters associated with control @code{ctl} to be checked.  If an error
is encountered an appropriate error message will be displayed and a 1
will be returned.  If no error is encountered a 0 is returned.

This method is implemented by all subclasses of @code{Control} and mainly
used internally when the dialog is accepted.
@example
@group
@exdent Example:

gCheckValue(ctl);
@end group
@end example
@sp 1
See also:  @code{CheckFunction::Control}
@end deffn


















@deffn {DeepDispose} DeepDispose::Control
@sp 2
This method performs the same function as @code{Dispose}.  See that
method for details.
@end deffn









@deffn {DefaultProcessingMode} DefaultProcessingMode::Control
@sp 2
@example
@group
r = gDefaultProcessingMode(ctl, msg, mode);

object   ctl;   /*  a control object         */
unsigned msg;   /*  message                  */
int      mode;  /*  default processing mode  */
object   r;     /*  the control obj          */
@end group
@end example
This method is used to determine when or if the Windows default message
procedure is processed for a given message (@code{msg}) associated with
a particular control (@code{ctl}).

WDS allows a programmer to specify an arbitrary number of functions
to be executed whenever a control receives a specific message (via
@code{AddHandlerAfter} and @code{AddHandlerBefore}).  Windows has
default procedures associated with many control messages.  At times
it is necessary to replace or augment this default functionality.
@code{DefaultProcessingMode} gives the programmer control over when and
if this default Windows functionality.  @code{mode} is used to specify
the desired mode.  The following table indicates the valid modes:

@table @code
@item 0
Do not execute the Windows default processing
@item 1
Execute default processing @emph{after} programmer defined handlers
@item 2
Execute default processing @emph{before} programmer defined handlers
@end table

Note that the default mode is always @code{1}, and must be explicitly
changed, if desired, for each message associated with each control.

@code{msg} is the particular message you wish to affect.  These messages
are fully documented in the Windows documentation in the Messages
section.  They normally begin with @code{WM_}.
@example
@group
@exdent Example:

gDefaultProcessingMode(ctl, (unsigned) WM_SIZE, 0);
@end group
@end example
@sp 1
See also:  @code{CallDefaultProc, AddHandlerAfter}
@end deffn











@deffn {Dispose} Dispose::Control
@sp 2
@example
@group
r = gDispose(ctl);

object  ctl;   /*  a control object  */
object  r;     /*  NULL              */
@end group
@end example
This method is used to remove and dispose of a control object when it
is no longer needed.  This method is rarely needed due to the fact that
when a dialog is disposed it automatically calls this method on each of
its associated controls.

The value returned is always @code{NULL} and may be used to null out
the variable which contained the object being disposed in order to
avoid future accidental use.
@example
@group
@exdent Example:

object  ctl;

ctl = gDispose(ctl);
@end group
@end example
@sp 1
See also:  @code{DeepDispose}
@end deffn














@deffn {Handle} Handle::Control
@sp 2
@example
@group
h = gHandle(ctl);

object  ctl;    /*  control object   */
HANDLE  h;      /*  Windows handle  */
@end group
@end example
This method is used to obtain the Windows internal handle associated with
a control object.  It will only return a valid handle while the dialog
is being performed via @code{gPerform}.  @code{NULL} will be returned
otherwise.

Note that this method may be used with most WDS objects in order to obtain
the internal handle that Windows normally associates with each type of object.
See the appropriate documentation.
@example
@group
@exdent Example:

object  ctl;
HANDLE  h;

h = gHandle(ctl);
@end group
@end example
@sp 1
See also:  @code{GetControl::Dialog}
@end deffn











@deffn {SetTopic} SetTopic::Control
@sp 2
@example
@group
pt = gSetTopic(ctl, tpc);

object  ctl;    /*  control object        */
char    *tpc;   /*  help topic            */
char    *pt;    /*  previous help topic   */
@end group
@end example
This method is used to associate help text with specific control @code{ctl}.
The help text is defined using the Windows help system and labeled
with the topic indicated by @code{tpc}.  Then, if the user hits
the F1 key while in the control, WDS will automatically bring up
the Windows help system and find the indicated topic.  If no control
specific help is specified, WDS will use any help associated with the
dialog as a whole.

WDS also supports dialog and control specific topics.  See the appropriate
sections.

This method returns any previous topic associated with the control.
@example
@group
@exdent Example:

object  ctl;

gSetTopic(ctl, "myControlHelp");
@end group
@end example
@sp 1
See also:  The @code{HelpSystem} class and @code{SetTopic::Dialog}
@end deffn











@subsection Text Control
The @code{TextControl} class is a control which enables the user to
enter arbitrary alphanumeric data on a single line.  This control
has numerous facilities to control the length and type of input the
user is allowed to enter.

The @code{TextControl} class is a subclass of @code{Control} and as such
inherits all the functionality associated with that class.  This section
only documents functionality particular to this class.

Standard control arguments are documented in the section entitled
``Standard Control Method Arguments''.












@deffn {Attach} Attach::TextControl
@sp 2
@example
@group
r = gAttach(ctl, val);

object  ctl;   /*  control object  */
object  val;   /*  ctl value       */
object  r;     /*  control object  */
@end group
@end example
This method is used to associate an independent Dynace object with the
value associated with a control object.  @code{val} should be an
instance of the @code{String} class and will be automatically updated
to reflect the value associated with the control.

This object (@code{val}) will never be disposed by WDS, even after
the control or associated dialog are disposed.  Therefore, this
is one way of gaining access to a control's value after the life
of the control.  It is the programmer's responsibility to dispose of
the object when it is no longer needed.
@example
@group
@exdent Example:

object  ctl, val;

val = gNew(String);
gAttach(ctl, val);
@end group
@end example
@sp 1
See also:  @code{Value, StringValue, SetValue}
@end deffn












@deffn {Capitalize} Capitalize::TextControl
@sp 2
@example
@group
r = gCapitalize(ctl);

object  ctl;   /*  control object  */
object  r;     /*  control object  */
@end group
@end example
This method is used to enable the auto-capitalize feature of the text
control.  If enabled, all alpha entry made by the user will be converted
to upper case both on the display and internally to the control.
@example
@group
@exdent Example:

object  ctl;

gCapitalize(ctl);
@end group
@end example
@sp 1
See also:  @code{TextRange}
@end deffn













@deffn {MaxLength} MaxLength::TextControl
@sp 2
@example
@group
r = gMaxLength(ctl, len);

object  ctl;   /*  control object  */
int     len;   /*  maximum length  */
object  r;     /*  control object  */
@end group
@end example
This method is used to set the maximum number of characters the user may
enter into a text control.  If the user attempts to enter more than
@code{len} characters, including spaces, the system will just beep.  If
the number of characters exceeds the field display width, the display
will scroll.
@example
@group
@exdent Example:

object  ctl;

gMaxLength(ctl, 25);
@end group
@end example
@sp 1
See also:  @code{MinLength, TextRange, Capitalize, CheckFunction::Control}
@end deffn












@deffn {MinLength} MinLength::TextControl
@sp 2
@example
@group
r = gMinLength(ctl, len);

object  ctl;   /*  control object  */
int     len;   /*  minimum length  */
object  r;     /*  control object  */
@end group
@end example
This method is used to set the minimum number of characters the user must
enter into a text control.  If the user attempts to accept the dialog
without entering @code{len} characters, including spaces, WDS will
issue an error message and return the user to the incomplete field.
@example
@group
@exdent Example:

object  ctl;

gMinLength(ctl, 5);
@end group
@end example
@sp 1
See also:  @code{MaxLength, TextRange, Capitalize, CheckFunction::Control}
@end deffn









@deffn {NewCtl} NewCtl::TextControl
@sp 2
@example
@group
ctl = mNewCtl(TextControl, id);

unsigned  id;   /*  control id      */
object   ctl;   /*  control object  */
@end group
@end example
This method is used to create a new control object to be identified as
@code{id} (see section ``Standard Control Method Arguments'').  This
method is mainly used internally.  A programmer would more often
use @code{AddControl::Dialog} to create controls and associated them
to a dialog.
@example
@group
@exdent Example:

object  ctl;

ctl = mNewCtl(TextControl, PERSON_NAME);
@end group
@end example
@sp 1
See also:  @code{AddControl::Dialog, GetControl::Dialog}
@end deffn










@deffn {SetStringValue} SetStringValue::TextControl
@sp 2
@example
@group
r = gSetStringValue(ctl, val);

object  ctl;    /*  control object  */
char    *val;   /*  ctl value       */
object  r;      /*  control object  */
@end group
@end example
This method is used to set the value associated with a control.  It is
often used to set the initial value prior to performing a dialog.
@code{val} should be a pointer to a character string which represents
the desired value for the control.

Any previously associated value object will be disposed when the new
value is set.  
@example
@group
@exdent Example:

object  ctl;

gSetStringValue(ctl, "Some value");
@end group
@end example
@sp 1
See also:  @code{SetValue, StringValue}
@end deffn








@deffn {SetValue} SetValue::TextControl
@sp 2
@example
@group
r = gSetValue(ctl, val);

object  ctl;    /*  control object  */
object  val;    /*  ctl value       */
object  r;      /*  control object  */
@end group
@end example
This method is used to set the value associated with a control.  It is
often used to set the initial value prior to performing a dialog.
@code{val} should be a Dynace object which is an instance of the
@code{String} class and initialized to the value desired for the control.

Any previously associated object will be disposed when the new value is set.
Also, @code{val} will automatically be disposed when the control or associated
dialog is disposed.
@example
@group
@exdent Example:

object  ctl;

gSetValue(ctl, gNewWithStr(String, "Some value"));
@end group
@end example
@sp 1
See also:  @code{SetStringValue, Value, Attach}
@end deffn









@deffn {StringValue} StringValue::TextControl
@sp 2
@example
@group
val = gStringValue(ctl);

object  ctl;   /*  control object  */
char    *val;  /*  ctl value       */
@end group
@end example
This method is used to obtain a C character pointer which represents the
value associated with the control.  This pointer will not be valid
after the control or associated dialog are disposed.
@example
@group
@exdent Example:

object  ctl;
char    *val;

val = gStringValue(ctl);
@end group
@end example
@sp 1
See also:  @code{Value, Attach, SetStringValue, CtlStringValue::Dialog}
@end deffn














@deffn {TextRange} TextRange::TextControl
@sp 2
@example
@group
r = gTextRange(ctl, min, max);

object  ctl;   /*  control object  */
int     min;   /*  minimum length  */
int     max;   /*  maximum length  */
object  r;     /*  control object  */
@end group
@end example
This method is used to set the minimum number of characters the user must
enter into a text control as well as the maximum number of characters
allowed.

If the user attempts to enter more than @code{max} characters, including
spaces, the system will just beep.  If the number of characters exceeds
the field display width, the display will scroll.

If the user attempts to accept the dialog without entering @code{min}
characters, including spaces, WDS will issue an error message and return
the user to the incomplete field.
@example
@group
@exdent Example:

object  ctl;

gTextRange(ctl, 5, 25);
@end group
@end example
@sp 1
See also:  @code{MaxLength, MinLength, Capitalize, CheckFunction::Control}
@end deffn











@deffn {Value} Value::TextControl
@sp 2
@example
@group
val = gValue(ctl);

object  ctl;   /*  control object  */
object  val;   /*  ctl value       */
@end group
@end example
This method is used to obtain a Dynace object which represents the value
associated with the control.  The object returned will be an instance of
the @code{String} class.  This object will be disposed by WDS when the
control object or associated dialog is disposed.
@example
@group
@exdent Example:

object  ctl, val;

val = gValue(ctl);
@end group
@end example
@sp 1
See also:  @code{StringValue, Attach, SetValue}
@end deffn








@subsection Numeric Control
The @code{NumericControl} class is a control which enables the user to
enter arbitrary numeric data.  This control has numerous facilities to
control the range and type of input the user is allowed to enter.

The @code{NumericControl} class is a subclass of @code{Control} and as such
inherits all the functionality associated with that class.  This section
only documents functionality particular to this class.

Standard control arguments are documented in the section entitled
``Standard Control Method Arguments''.












@deffn {Attach} Attach::NumericControl
@sp 2
@example
@group
r = gAttach(ctl, val);

object  ctl;   /*  control object  */
object  val;   /*  ctl value       */
object  r;     /*  control object  */
@end group
@end example
This method is used to associate an independent Dynace object with the
value associated with a control object.  @code{val} should be an
instance of one of the Dynace numeric classes which are subclasses of
the @code{Number} class, such as @code{ShortInteger} or
@code{DoubleFloat}.  This numeric object will be automatically updated
to reflect the value associated with the control.

This object (@code{val}) will never be disposed by WDS, even after
the control or associated dialog are disposed.  Therefore, this
is one way of gaining access to a control's value after the life
of the control.  It is the programmer's responsibility to dispose of
the object when it is no longer needed.
@example
@group
@exdent Example:

object  ctl, val;

val = gNewWithDouble(DoubleFloat, 0.0);
gAttach(ctl, val);
@end group
@end example
@sp 1
See also:  @code{Value, DoubleValue, ShortValue, SetValue}
@end deffn












@deffn {DoubleValue} DoubleValue::NumericControl
@sp 2
@example
@group
val = gDoubleValue(ctl);

object  ctl;   /*  control object  */
double  val;   /*  ctl value       */
@end group
@end example
This method is used to obtain a C double which represents the
value associated with the control.  
@example
@group
@exdent Example:

object  ctl;
double  val;

val = gDoubleValue(ctl);
@end group
@end example
@sp 1
See also:  @code{Value, Attach, SetDoubleValue, CtlDoubleValue::Dialog}
@iftex
@hfil @break @hglue .63in 
@end iftex
@code{UnsignedShortValue, ShortValue, LongValue}
@end deffn











@deffn {LongValue} LongValue::NumericControl
@sp 2
@example
@group
val = gLongValue(ctl);

object  ctl;   /*  control object  */
long    val;   /*  ctl value       */
@end group
@end example
This method is used to obtain a C long integer which represents the
value associated with the control.  
@example
@group
@exdent Example:

object  ctl;
long    val;

val = gLongValue(ctl);
@end group
@end example
@sp 1
See also:  @code{Value, Attach, SetLongValue, CtlLongValue::Dialog}
@iftex
@hfil @break @hglue .63in 
@end iftex
@code{UnsignedShortValue, ShortValue, DoubleValue}
@end deffn








@deffn {NewCtl} NewCtl::NumericControl
@sp 2
@example
@group
ctl = mNewCtl(NumericControl, id);

unsigned  id;   /*  control id      */
object   ctl;   /*  control object  */
@end group
@end example
This method is used to create a new control object to be identified as
@code{id} (see section ``Standard Control Method Arguments'').  This
method is mainly used internally.  A programmer would more often
use @code{AddControl::Dialog} to create controls and associated them
to a dialog.
@example
@group
@exdent Example:

object  ctl;

ctl = mNewCtl(NumericControl, PERSON_AGE);
@end group
@end example
@sp 1
See also:  @code{AddControl::Dialog, GetControl::Dialog}
@end deffn









@deffn {NumericRange} NumericRange::NumericControl
@sp 2
@example
@group
r = gNumericRange(ctl, min, max, dp);

object  ctl;   /*  control object  */
double  min;   /*  minimum value   */
double  max;   /*  maximum value   */
int     dp;    /*  decimal places  */
object  r;     /*  control object  */
@end group
@end example
This method is used to set the minimum and maximum allowable values
the control will accept from the user.  It also sets the number of
digits to the right of the decimal point the control will accept.

If the user enters a number outside the specified range, WDS will
issue an error message and prompt the user for an acceptable value.
@example
@group
@exdent Example:

object  ctl;

gNumericRange(ctl, 5.0, 25.0, 1);
@end group
@end example
@sp 1
See also:  @code{CheckFunction::Control}
@end deffn














@deffn {SetDoubleValue} SetDoubleValue::NumericControl
@sp 2
@example
@group
r = gSetDoubleValue(ctl, val);

object  ctl;    /*  control object  */
double  val;    /*  ctl value       */
object  r;      /*  control object  */
@end group
@end example
This method is used to set the value associated with a control.  It is
often used to set the initial value prior to performing a dialog.
@code{val} should be the value which represents the desired value for
the control.

Any previously associated value object (associated via @code{SetValue})
will be disposed when the new value is set.
@example
@group
@exdent Example:

object  ctl;

gSetDoubleValue(ctl, 3.14159265358979);
@end group
@end example
@sp 1
See also:  @code{SetValue, DoubleValue, CtlDoubleValue::Dialog, SetUShortValue,}
@iftex
@hfil @break @hglue .64in   
@end iftex
@code{SetShortValue, SetLongValue}
@end deffn















@deffn {SetLongValue} SetLongValue::NumericControl
@sp 2
@example
@group
r = gSetLongValue(ctl, val);

object  ctl;    /*  control object  */
long    val;    /*  ctl value       */
object  r;      /*  control object  */
@end group
@end example
This method is used to set the value associated with a control.  It is
often used to set the initial value prior to performing a dialog.
@code{val} should be the value which represents the desired value for
the control.

Any previously associated value object (associated via @code{SetValue})
will be disposed when the new value is set.
@example
@group
@exdent Example:

object  ctl;

gSetLongValue(ctl, 66L);
@end group
@end example
@sp 1
See also:  @code{SetValue, LongValue, CtlLongValue::Dialog, SetUShortValue,}
@iftex
@hfil @break @hglue .63in 
@end iftex
@code{SetShortValue, SetDoubleValue}
@end deffn














@deffn {SetShortValue} SetShortValue::NumericControl
@sp 2
@example
@group
r = gSetShortValue(ctl, val);

object  ctl;    /*  control object  */
int     val;    /*  ctl value       */
object  r;      /*  control object  */
@end group
@end example
This method is used to set the value associated with a control.  It is
often used to set the initial value prior to performing a dialog.
@code{val} should be the value which represents the desired value for
the control.

Any previously associated value object (associated via @code{SetValue})
will be disposed when the new value is set.
@example
@group
@exdent Example:

object  ctl;

gSetShortValue(ctl, 66);
@end group
@end example
@sp 1
See also:  @code{SetValue, ShortValue, CtlShortValue::Dialog, SetUShortValue,}
@iftex
@hfil @break @hglue .63in 
@end iftex
@code{SetLongValue, SetDoubleValue}
@end deffn











@deffn {SetUShortValue} SetUShortValue::NumericControl
@sp 2
@example
@group
r = gSetUShortValue(ctl, val);

object    ctl;  /*  control object  */
unsigned  val;  /*  ctl value       */
object    r;    /*  control object  */
@end group
@end example
This method is used to set the value associated with a control.  It is
often used to set the initial value prior to performing a dialog.
@code{val} should be the value which represents the desired value for
the control.

Any previously associated value object (associated via @code{SetValue})
will be disposed when the new value is set.
@example
@group
@exdent Example:

object  ctl;

gSetUShortValue(ctl, 66);
@end group
@end example
@sp 1
See also:  @code{SetValue, UnsignedShortValue, CtlUnsignedShortValue::Dialog}
@iftex
@hfil @break @hglue .63in 
@end iftex
@code{SetShortValue, SetLongValue, SetDoubleValue}
@end deffn
















@deffn {SetValue} SetValue::NumericControl
@sp 2
@example
@group
r = gSetValue(ctl, val);

object  ctl;    /*  control object  */
object  val;    /*  ctl value       */
object  r;      /*  control object  */
@end group
@end example
This method is used to set the value associated with a control.  It is
often used to set the initial value prior to performing a dialog.
@code{val} should be a Dynace object which is an instance of one of the
subclasses of the @code{Number} class and initialized to the value
desired for the control.

Any previously associated object will be disposed when the new value is set.
Also, @code{val} will automatically be disposed when the control or associated
dialog is disposed.
@example
@group
@exdent Example:

object  ctl;

gSetValue(ctl, gNewWithDouble(DoubleFloat, 3.141));
@end group
@end example
@sp 1
See also:  @code{Value, SetShortValue, SetUShortValue}
        @code{SetLongValue, SetDoubleValue}
@end deffn











@deffn {ShortValue} ShortValue::NumericControl
@sp 2
@example
@group
val = gShortValue(ctl);

object  ctl;   /*  control object  */
short   val;   /*  ctl value       */
@end group
@end example
This method is used to obtain a C short integer which represents the
value associated with the control.  
@example
@group
@exdent Example:

object  ctl;
short   val;

val = gShortValue(ctl);
@end group
@end example
@sp 1
See also:  @code{Value, Attach, SetShortValue, CtlShortValue::Dialog}
@iftex
@hfil @break @hglue .63in 
@end iftex
@code{UnsignedShortValue, LongValue, DoubleValue}
@end deffn











@deffn {UnsignedShortValue} UnsignedShortValue::NumericControl
@sp 2
@example
@group
val = gUnsignedShortValue(ctl);

object  ctl;           /*  control object  */
unsigned short  val;   /*  ctl value       */
@end group
@end example
This method is used to obtain a C unsigned short integer which represents the
value associated with the control.  
@example
@group
@exdent Example:

object  ctl;
unsigned short  val;

val = gUnsignedShortValue(ctl);
@end group
@end example
@sp 1
See also:  @code{Value, Attach, SetUShortValue, CtlUnsignedShortValue::Dialog}
@iftex
@hfil @break @hglue .63in 
@end iftex
@code{ShortValue, LongValue, DoubleValue}
@end deffn










@deffn {Value} Value::NumericControl
@sp 2
@example
@group
val = gValue(ctl);

object  ctl;   /*  control object  */
object  val;   /*  ctl value       */
@end group
@end example
This method is used to obtain a Dynace object which represents the value
associated with the control.  The object returned will be an instance of
one of the subclasses of the @code{Number} class.  This object will be
disposed by WDS when the control object or associated dialog is
disposed.
@example
@group
@exdent Example:

object  ctl, val;

val = gValue(ctl);
@end group
@end example
@sp 1
See also:  @code{ShortValue, DoubleValue, Attach, SetValue}
@end deffn









@subsection Date Control
The @code{DateControl} class is a control which enables the user to
enter date information.  The user enters dates in the form mm/dd/yy
or mm/dd/yyyy, and when the user leaves the field it changes to the
form MMM DD, YYYY.  For example, the user would enter 6/8/59 and
when the field is accepted it will be changed to display Jun 8, 1959.

Dates are stored internally as C longs in the form YYYYMMDD so that
6/8/59 would be represented as the long 19590608.  This control has
numerous facilities to control the range and type of input the user is
allowed to enter.

The @code{DateControl} class is a subclass of @code{Control} and as such
inherits all the functionality associated with that class.  This section
only documents functionality particular to this class.

Standard control arguments are documented in the section entitled
``Standard Control Method Arguments''.












@deffn {Attach} Attach::DateControl
@sp 2
@example
@group
r = gAttach(ctl, val);

object  ctl;   /*  control object  */
object  val;   /*  ctl value       */
object  r;     /*  control object  */
@end group
@end example
This method is used to associate an independent Dynace object with the
value associated with a control object.  @code{val} should be an
instance of the @code{Date} class.  This date object will be
automatically updated to reflect the value associated with the control.

This object (@code{val}) will never be disposed by WDS, even after
the control or associated dialog are disposed.  Therefore, this
is one way of gaining access to a control's value after the life
of the control.  It is the programmer's responsibility to dispose of
the object when it is no longer needed.
@example
@group
@exdent Example:

object  ctl, val;

val = vNew(Date, 0L);
gAttach(ctl, val);
@end group
@end example
@sp 1
See also:  @code{Value, LongValue, SetValue, SetLongValue}
@end deffn
















@deffn {DateRange} DateRange::DateControl
@sp 2
@example
@group
r = gDateRange(ctl, min, max, an);

object  ctl;   /*  control object  */
long    min;   /*  minimum value   */
long    max;   /*  maximum value   */
int     an;    /*  allow none      */
object  r;     /*  control object  */
@end group
@end example
This method is used to set the minimum and maximum allowable values
the control will accept from the user.  It also controls whether
or not the control will accept no date.

@code{min} and @code{max} are of the form YYYYMMDD.  For example
6/8/59 would be represented as 19590608.  If @code{an} is 0
the user must enter a valid date between @code{min} and @code{max}.
If, however, @code{an} is 1, the user must either enter a valid
date between the given ranges or may also leave the field blank.
If the user does not enter a date the field value will be 0L.

If the user enters an invalid date or one outside the specified range,
WDS will issue an error message and prompt the user for an acceptable
value.
@example
@group
@exdent Example:

object  ctl;

gDateRange(ctl, 19990101L, 19991231L, 1);
@end group
@end example
@sp 1
See also:  @code{CheckFunction::Control}
@end deffn








@deffn {LongValue} LongValue::DateControl
@sp 2
@example
@group
val = gLongValue(ctl);

object  ctl;   /*  control object  */
long    val;   /*  ctl value       */
@end group
@end example
This method is used to obtain a C long integer which represents the
value associated with the control.  It will be in the form YYYYMMDD.
For example 6/8/59 would be represented as 19590608.
@example
@group
@exdent Example:

object  ctl;
long    val;

val = gLongValue(ctl);
@end group
@end example
@sp 1
See also:  @code{Value, Attach, SetLongValue, CtlLongValue::Dialog}
@end deffn










@deffn {NewCtl} NewCtl::DateControl
@sp 2
@example
@group
ctl = mNewCtl(DateControl, id);

unsigned  id;   /*  control id      */
object   ctl;   /*  control object  */
@end group
@end example
This method is used to create a new control object to be identified as
@code{id} (see section ``Standard Control Method Arguments'').  This
method is mainly used internally.  A programmer would more often
use @code{AddControl::Dialog} to create controls and associated them
to a dialog.
@example
@group
@exdent Example:

object  ctl;

ctl = mNewCtl(DateControl, PERSON_BD);
@end group
@end example
@sp 1
See also:  @code{AddControl::Dialog, GetControl::Dialog}
@end deffn










@deffn {SetLongValue} SetLongValue::DateControl
@sp 2
@example
@group
r = gSetLongValue(ctl, val);

object  ctl;    /*  control object  */
long    val;    /*  ctl value       */
object  r;      /*  control object  */
@end group
@end example
This method is used to set the value associated with a control.  It is
often used to set the initial value prior to performing a dialog.
@code{val} should be the value which represents the desired value for
the control of the form YYYYMMDD.  For example 6/8/59 would be
19590608.

Any previously associated value object (associated via @code{SetValue})
will be disposed when the new value is set.
@example
@group
@exdent Example:

object  ctl;

gSetLongValue(ctl, 19990101L);
@end group
@end example
@sp 1
See also:  @code{SetValue, LongValue, CtlLongValue::Dialog}
@end deffn













@deffn {SetValue} SetValue::DateControl
@sp 2
@example
@group
r = gSetValue(ctl, val);

object  ctl;    /*  control object  */
object  val;    /*  ctl value       */
object  r;      /*  control object  */
@end group
@end example
This method is used to set the value associated with a control.  It is
often used to set the initial value prior to performing a dialog.
@code{val} should be a Dynace object which is an instance of the
@code{Date} class and initialized to the value desired for the control.

Any previously associated object will be disposed when the new value is set.
Also, @code{val} will automatically be disposed when the control or associated
dialog is disposed.
@example
@group
@exdent Example:

object  ctl;

gSetValue(ctl, vNew(Date, 19940101L));
@end group
@end example
@sp 1
See also:  @code{SetLongValue, Value}
@end deffn














@deffn {Value} Value::DateControl
@sp 2
@example
@group
val = gValue(ctl);

object  ctl;   /*  control object  */
object  val;   /*  ctl value       */
@end group
@end example
This method is used to obtain a Dynace object which represents the value
associated with the control.  The object returned will be an instance of
the @code{Date} class.  This object will be disposed by WDS when the
control object or associated dialog is disposed.
@example
@group
@exdent Example:

object  ctl, val;

val = gValue(ctl);
@end group
@end example
@sp 1
See also:  @code{LongValue, Attach, SetValue}
@end deffn

















@subsection Push Buttons
The @code{PushButton} class represents a control which allows the user
to perform an immediate action.  That means that there is no particular
data associated with the control.  The user pushes the button and an
associated action gets performed at that point.  The @code{PushButton}
class allows the programmer to associate a C function to the control
such that whenever the user clicks on the button the specified function
will be evoked.

The @code{PushButton} class is a subclass of @code{Control} and as such
inherits all the functionality associated with that class.  This section
only documents functionality particular to this class.

Standard control arguments are documented in the section entitled
``Standard Control Method Arguments''.





@deffn {NewCtl} NewCtl::PushButton
@sp 2
@example
@group
ctl = mNewCtl(PushButton, id);

unsigned  id;   /*  control id      */
object   ctl;   /*  control object  */
@end group
@end example
This method is used to create a new control object to be identified as
@code{id} (see section ``Standard Control Method Arguments'').  This
method is mainly used internally.  A programmer would more often
use @code{AddControl::Dialog} to create controls and associated them
to a dialog.
@example
@group
@exdent Example:

object  ctl;

ctl = mNewCtl(PushButton, MY_BUTTON);
@end group
@end example
@sp 1
See also:  @code{AddControl::Dialog, GetControl::Dialog}
@end deffn









@deffn {Perform} Perform::PushButton
@sp 2
@example
@group
r = gPerform(ctl);

object  ctl;      /*  control object  */
int     r;        /*  result          */
@end group
@end example
This method is used to manually evoke the function a programmer associated 
with a push button.  The value returned is the result of the function
attached to the button.

This method is seldom needed because WDS automatically evokes the function
when the user clicks the button.
@example
@group
@exdent Example:

object  ctl;

gPerform(ctl);
@end group
@end example
@sp 1
See also:  @code{SetFunction}
@end deffn






@deffn {SetFunction} SetFunction::PushButton
@sp 2
@example
@group
r = gSetFunction(ctl, fun);

object  ctl;      /*  control object           */
int     (*fun)(); /*  check function           */
ofun    r;        /*  previous check function  */
@end group
@end example
This method is used to associate a C function to a push button such that
if the user clicks the button the C function will immediately get
evoked.  The value returned is the previous function, if any, associated
with the control.

The C function takes the following form:
@example
@group
int     fun(object ctl, object dlg)
@{
        ....
@}
@end group
@end example
Where @code{ctl} is the control object, and @code{dlg} is the object
representing the dialog the control is associated with.  The value
returned by @code{fun} only has meaning if the button was either
@code{IDOK} or @code{IDCANCEL} (the ones normally used to exit the
dialog).  If the return value is 1 then the dialog will exit like normal.
If the return value is 0 then the dialog will not exit.
@example
@group
@exdent Example:

object  ctl;

gSetFunction(ctl, fun);
@end group
@end example
@sp 1
See also:  @code{Perform}
@end deffn















@subsection Check Boxes
The @code{CheckBox} class represents a control which allows the user
to select or de-select a particular option.  Unlike radio buttons,
check boxes are not mutually exclusive.

The @code{CheckBox} class is a subclass of @code{Control} and as such
inherits all the functionality associated with that class.  This section
only documents functionality particular to this class.

Standard control arguments are documented in the section entitled
``Standard Control Method Arguments''.








@deffn {Attach} Attach::CheckBox
@sp 2
@example
@group
r = gAttach(ctl, val);

object  ctl;   /*  control object  */
object  val;   /*  ctl value       */
object  r;     /*  control object  */
@end group
@end example
This method is used to associate an independent Dynace object with the
state associated with a control object.  @code{val} should be an
instance of the @code{ShortInteger} class.  This object will be
automatically updated to reflect the state associated with the control.

The object will represent 0 if the box is not checked, 1 if it is
checked, or 2 if the box is grayed.

This object (@code{val}) will never be disposed by WDS, even after
the control or associated dialog are disposed.  Therefore, this
is one way of gaining access to a control's value after the life
of the control.  It is the programmer's responsibility to dispose of
the object when it is no longer needed.
@example
@group
@exdent Example:

object  ctl, val;

val = gNewWithInt(ShortInteger, 0);
gAttach(ctl, val);
@end group
@end example
@sp 1
See also:  @code{Value, ShortValue, SetValue, SetShortValue}
@end deffn












@deffn {NewCtl} NewCtl::CheckBox
@sp 2
@example
@group
ctl = mNewCtl(CheckBox, id);

unsigned  id;   /*  control id      */
object   ctl;   /*  control object  */
@end group
@end example
This method is used to create a new control object to be identified as
@code{id} (see section ``Standard Control Method Arguments'').  This
method is mainly used internally.  A programmer would more often
use @code{AddControl::Dialog} to create controls and associated them
to a dialog.
@example
@group
@exdent Example:

object  ctl;

ctl = mNewCtl(CheckBox, MY_CHECKBOX);
@end group
@end example
@sp 1
See also:  @code{AddControl::Dialog, GetControl::Dialog}
@end deffn







@deffn {SetShortValue} SetShortValue::CheckBox
@sp 2
@example
@group
r = gSetShortValue(ctl, val);

object  ctl;    /*  control object  */
int     val;    /*  ctl value       */
object  r;      /*  control object  */
@end group
@end example
This method is used to set the value associated with a control.  It is
often used to set the initial value prior to performing a dialog.
@code{val} should be the value which represents the desired state associated
with the check box.

@code{val} should be set to 0 to uncheck the box, 1 to check it, or 2 to
gray the box (only work on 3-state boxes).

Any previously associated value object (associated via @code{SetValue})
will be disposed when the new value is set.
@example
@group
@exdent Example:

object  ctl;

gSetShortValue(ctl, 1);
@end group
@end example
@sp 1
See also:  @code{SetValue, ShortValue, CtlShortValue::Dialog}
@end deffn
















@deffn {SetValue} SetValue::CheckBox
@sp 2
@example
@group
r = gSetValue(ctl, val);

object  ctl;    /*  control object  */
object  val;    /*  ctl value       */
object  r;      /*  control object  */
@end group
@end example
This method is used to set the value associated with a control.  It is
often used to set the initial value prior to performing a dialog.
@code{val} should be a Dynace object which is an instance of the
@code{ShortInteger} class and initialized to the value desired for the
control.

@code{val} should represent 0 to uncheck the box, 1 to check it, or 2 to
gray the box (only work on 3-state boxes).

Any previously associated object will be disposed when the new value is set.
Also, @code{val} will automatically be disposed when the control or associated
dialog is disposed.
@example
@group
@exdent Example:

object  ctl;

gSetValue(ctl, gNewWithInt(ShortInteger, 1));
@end group
@end example
@sp 1
See also:  @code{SetShortValue, Value}
@end deffn













@deffn {ShortValue} ShortValue::CheckBox
@sp 2
@example
@group
val = gShortValue(ctl);

object  ctl;   /*  control object  */
short   val;   /*  ctl value       */
@end group
@end example
This method is used to obtain a C short integer which represents the
state associated with the control.  The returned integer will be 0 if
the box is not checked, 1 if it is checked, or 2 if the box is grayed.
@example
@group
@exdent Example:

object  ctl;
short   val;

val = gShortValue(ctl);
@end group
@end example
@sp 1
See also:  @code{Value, Attach, SetShortValue, CtlShortValue::Dialog}
@end deffn













@deffn {Value} Value::CheckBox
@sp 2
@example
@group
val = gValue(ctl);

object  ctl;   /*  control object  */
object  val;   /*  ctl value       */
@end group
@end example
This method is used to obtain a Dynace object which represents the value
associated with the control.  The object returned will be an instance of
the @code{ShortInteger} class.  The returned object will represent 0 if
the box is not checked, 1 if it is checked, or 2 if the box is grayed.
This object will be disposed by WDS when the control object or
associated dialog is disposed.
@example
@group
@exdent Example:

object  ctl, val;

val = gValue(ctl);
@end group
@end example
@sp 1
See also:  @code{ShortValue, Attach, SetValue}
@end deffn















@subsection Radio Buttons
The @code{RadioButton} class represents a control which allows the user
to select or de-select a particular option.  Unlike check boxes, however,
radio buttons are mutually exclusive within a group - similar to the
buttons on old fashioned radios, when one button gets selected all the
other ones in the group get de-selected.

When using radio buttons, be sure to use the @code{Group} function in order
to tell WDS which radio buttons should be grouped together.

The @code{RadioButton} class is a subclass of @code{Control} and as such
inherits all the functionality associated with that class.  This section
only documents functionality particular to this class.

Standard control arguments are documented in the section entitled
``Standard Control Method Arguments''.






@deffn {Attach} Attach::RadioButton
@sp 2
@example
@group
r = gAttach(ctl, val);

object  ctl;   /*  control object  */
object  val;   /*  ctl value       */
object  r;     /*  control object  */
@end group
@end example
This method is used to associate an independent Dynace object with the
state associated with a control object.  @code{val} should be an
instance of the @code{ShortInteger} class.  This object will be
automatically updated to reflect the state associated with the control.

The object will represent 0 if the button is not selected, 1 if it is
selected, or 2 if the button is grayed.

This object (@code{val}) will never be disposed by WDS, even after
the control or associated dialog are disposed.  Therefore, this
is one way of gaining access to a control's value after the life
of the control.  It is the programmer's responsibility to dispose of
the object when it is no longer needed.
@example
@group
@exdent Example:

object  ctl, val;

val = gNewWithInt(ShortInteger, 0);
gAttach(ctl, val);
@end group
@end example
@sp 1
See also:  @code{Value, ShortValue, SetValue, SetShortValue}
@end deffn











@deffn {Group} Group::RadioButton
@sp 2
@example
@group
r = mGroup(ctl, id1, id2);

object  ctl;    /*  control object   */
unsigned id1;   /*  starting ctl id  */
unsigned id2;   /*  ending ctl id    */
object  r;      /*  control object   */
@end group
@end example
This method is used to tell WDS which radio buttons are to be considered
part of the same mutually exclusive group.  @code{id1} and @code{id2}
are the resource editor defined id's associated with the beginning and
end of the group.  WDS uses this information to automatically de-select
all radio buttons which are part of a group when one of the buttons is
selected.
@example
@group
@exdent Example:

object  ctl;

mGroup(ctl, FIRST_RB, LAST_RB);
@end group
@end example
@c @sp 1
@c See also:  @code{}
@end deffn











@deffn {NewCtl} NewCtl::RadioButton
@sp 2
@example
@group
ctl = mNewCtl(RadioButton, id);

unsigned  id;   /*  control id      */
object   ctl;   /*  control object  */
@end group
@end example
This method is used to create a new control object to be identified as
@code{id} (see section ``Standard Control Method Arguments'').  This
method is mainly used internally.  A programmer would more often
use @code{AddControl::Dialog} to create controls and associated them
to a dialog.
@example
@group
@exdent Example:

object  ctl;

ctl = mNewCtl(RadioButton, MY_RADIOBUTTON);
@end group
@end example
@sp 1
See also:  @code{AddControl::Dialog, GetControl::Dialog, Group}
@end deffn







@deffn {SetShortValue} SetShortValue::RadioButton
@sp 2
@example
@group
r = gSetShortValue(ctl, val);

object  ctl;    /*  control object  */
int     val;    /*  ctl value       */
object  r;      /*  control object  */
@end group
@end example
This method is used to set the value associated with a control.  It is
often used to set the initial value prior to performing a dialog.
@code{val} should be the value which represents the desired state associated
with the radio button.

@code{val} should be set to 0 to de-select the button, 1 to select it, or 2 to
gray the button (only work on 3-state buttons).

Any previously associated value object (associated via @code{SetValue})
will be disposed when the new value is set.
@example
@group
@exdent Example:

object  ctl;

gSetShortValue(ctl, 1);
@end group
@end example
@sp 1
See also:  @code{SetValue, ShortValue, CtlShortValue::Dialog, Group}
@end deffn
















@deffn {SetValue} SetValue::RadioButton
@sp 2
@example
@group
r = gSetValue(ctl, val);

object  ctl;    /*  control object  */
object  val;    /*  ctl value       */
object  r;      /*  control object  */
@end group
@end example
This method is used to set the value associated with a control.  It is
often used to set the initial value prior to performing a dialog.
@code{val} should be a Dynace object which is an instance of the
@code{ShortInteger} class and initialized to the value desired for the
control.

@code{val} should represent 0 to de-select the button, 1 to select it, or 2 to
gray the button (only work on 3-state buttons).

Any previously associated object will be disposed when the new value is set.
Also, @code{val} will automatically be disposed when the control or associated
dialog is disposed.
@example
@group
@exdent Example:

object  ctl;

gSetValue(ctl, gNewWithInt(ShortInteger, 1));
@end group
@end example
@sp 1
See also:  @code{SetShortValue, Value, Group}
@end deffn













@deffn {ShortValue} ShortValue::RadioButton
@sp 2
@example
@group
val = gShortValue(ctl);

object  ctl;   /*  control object  */
short   val;   /*  ctl value       */
@end group
@end example
This method is used to obtain a C short integer which represents the
state associated with the control.  The returned integer will be 0 if
the button is not selected, 1 if it is selected, or 2 if the button is grayed.
@example
@group
@exdent Example:

object  ctl;
short   val;

val = gShortValue(ctl);
@end group
@end example
@sp 1
See also:  @code{Value, Attach, SetShortValue, CtlShortValue::Dialog}
@end deffn













@deffn {Value} Value::RadioButton
@sp 2
@example
@group
val = gValue(ctl);

object  ctl;   /*  control object  */
object  val;   /*  ctl value       */
@end group
@end example
This method is used to obtain a Dynace object which represents the value
associated with the control.  The object returned will be an instance of
the @code{ShortInteger} class.  The returned object will represent 0 if
the button is not selected, 1 if it is selected, or 2 if the button is grayed.
This object will be disposed by WDS when the control object or
associated dialog is disposed.
@example
@group
@exdent Example:

object  ctl, val;

val = gValue(ctl);
@end group
@end example
@sp 1
See also:  @code{ShortValue, Attach, SetValue}
@end deffn











@subsection List Boxes
The @code{ListBox} class represents a control which allows the user
to select from a list of choices.  These choices are presented in
a vertical, possibly drop down, list of text choices.  This control
supports access to the user's choice via the text option selected or
an ordinal value.

The application specifies, at runtime, the choices located in the list box.

The main difference between list boxes and combo boxes is that with list
boxes the user can only select from the available choices, however, in
addition, combo boxes allow the user to enter a choice independent of
the available options.

The @code{ListBox} class is a subclass of @code{Control} and as such
inherits all the functionality associated with that class.  This section
only documents functionality particular to this class.

Standard control arguments are documented in the section entitled
``Standard Control Method Arguments''.













@deffn {AddOption} AddOption::ListBox
@sp 2
@example
@group
r = gAddOption(ctl, op);

object  ctl;    /*  control object  */
char    *op;    /*  new option      */
object  r;      /*  control object  */
@end group
@end example
This method is used to append a new option to the list of options associated
with a list box.  This method may be used prior to or during the execution
(via @code{gPerform}) of a dialog.

@code{op} may also be a Dynace object, an instance of the @code{String}
class, representing the new choice.  If a Dynace object is used, it must
be typecast to a string and it will be disposed when the control or
associated dialog are disposed.
@example
@group
@exdent Example:

object  ctl;

gAddOption(ctl, "The first choice");
gAddOption(ctl, "The second choice");
@end group
@end example
@sp 1
See also:  @code{Alphabetize::ListBox, AddOptionAt::ListBox}
@end deffn















@deffn {Alphabetize} Alphabetize::ListBox
@sp 2
@example
@group
r = gAlphabetize(ctl);

object  ctl;    /*  control object  */
object  r;      /*  control object  */
@end group
@end example
This method is used to cause the automatic alphabetization of all
options added via the @code{AddOption} method.  In order to function
properly, @code{Alphabetize} must be called prior to any calls to
@code{AddOption}.  This method simply returns the object passed.
@c @example
@c @group
@c @exdent Example:
@c 
@c @end group
@c @end example
@sp 1
See also:  @code{AddOption::ListBox}
@end deffn















@deffn {Attach} Attach::ListBox
@sp 2
@example
@group
r = gAttach(ctl, val);

object  ctl;   /*  control object  */
object  val;   /*  ctl value       */
object  r;     /*  control object  */
@end group
@end example
This method is used to associate an independent Dynace object with the
state associated with a control object.  @code{val} should be an
instance of the @code{ShortInteger} class or the @code{String} class.
This object will be automatically updated to reflect the state
associated with the control.

If @code{val} is an instance of the @code{ShortIntgeger} class it will
be used to represent the ordinal value of the selected item.  This value
is a 0 based index from the top of the list to the bottom.  -1 is used
to represent no valid selection.

If @code{val} is an instance of the @code{String} class it will be used
to represent the string represented by the user's choice.  If no choice is
made it will represent "".

This object (@code{val}) will never be disposed by WDS, even after
the control or associated dialog are disposed.  Therefore, this
is one way of gaining access to a control's value after the life
of the control.  It is the programmer's responsibility to dispose of
the object when it is no longer needed.
@example
@group
@exdent Example:

object  ctl, val;

val = gNewWithInt(ShortInteger, 0);
gAttach(ctl, val);
@end group
@end example
@sp 1
See also:  @code{Value, ShortValue, StringValue}
@end deffn
















@deffn {GetSelections} GetSelections::ListBox
@sp 2
@example
@group
ary = gGetSelections(ctl);

object  ctl;   /*  control object       */
object  ary;   /*  array of selections  */
@end group
@end example
This method is used to obtain an array representing the items selected
by the user.  It is most useful with list boxes which have multi-selection
enabled.  The array returned is an instance of the @code{IntegerArray} class
and @emph{must} be explicitly disposed when no longer needed.  Each element
of the array represents a zero based index of one of the user's selections,
and the size of the array represents the number of items selected.

This method may only be used while a dialog is active and will return
@code{NULL} otherwise.
@c @example
@c @group
@c @exdent Example:
@c 
@c @end group
@c @end example
@sp 1
See also:  @code{ValueAt, NumbSelected, Value, ListIndex}
@end deffn
















@deffn {ListIndex} ListIndex::ListBox
@sp 2
@example
@group
val = gListIndex(ctl);

object  ctl;   /*  control object  */
object  val;   /*  ctl value       */
@end group
@end example
This method is used to obtain a Dynace object which represents the value
associated with the control.  The object returned will be an instance of
the @code{ShortInteger} class.  The returned object will represent the
ordinal value of the choice made by the user.  This ordinal value is a
zero based index from top to bottom.  -1 indicates that no choice was
made.

This object will be disposed by WDS when the control object or
associated dialog is disposed.
@example
@group
@exdent Example:

object  ctl, val;

val = gListIndex(ctl);
@end group
@end example
@sp 1
See also:  @code{ShortValue, Value, StringValue, GetSelections, SetValue}
@end deffn




@deffn {NewCtl} NewCtl::ListBox
@sp 2
@example
@group
ctl = mNewCtl(ListBox, id);

unsigned  id;   /*  control id      */
object   ctl;   /*  control object  */
@end group
@end example
This method is used to create a new control object to be identified as
@code{id} (see section ``Standard Control Method Arguments'').  This
method is mainly used internally.  A programmer would more often
use @code{AddControl::Dialog} to create controls and associated them
to a dialog.
@example
@group
@exdent Example:

object  ctl;

ctl = mNewCtl(ListBox, MY_LISTBOX);
@end group
@end example
@sp 1
See also:  @code{AddControl::Dialog, GetControl::Dialog, AddOption}
@end deffn











@deffn {NumbSelected} NumbSelected::ListBox
@sp 2
@example
@group
n = gNumbSelected(ctl);

object  ctl;    /*  control object   */
int     n;      /*  number selected  */
@end group
@end example
This method is used to obtain the number of list box items the user
has selected.  It is mainly used with multi-selection list boxes.
This method should only be used while the dialog is active and
will return -1 otherwise.
@c @example
@c @group
@c @exdent Example:
@c 
@c @end group
@c @end example
@sp 1
See also:  @code{GetSelections}
@end deffn














@deffn {RemoveAll} RemoveAll::ListBox
@sp 2
@example
@group
r = gRemoveAll(ctl);

object  ctl;    /*  control object  */
object  r;      /*  control object  */
@end group
@end example
This method is used to remove all items from a list box.  It may be used
prior to or during the execution of a dialog.  The control object passed
is returned.
@c @example
@c @group
@c @exdent Example:
@c 
@c @end group
@c @end example
@sp 1
See also:  @code{RemoveStr, RemoveInt}
@end deffn
















@deffn {RemoveInt} RemoveInt::ListBox
@sp 2
@example
@group
r = gRemoveInt(ctl, idx);

object  ctl;    /*  control object  */
int     idx;    /*  index           */
object  r;      /*  control object  */
@end group
@end example
This method is used to remove an item from a list box while the dialog
is active.  @code{idx} is a zero based index of the item to be removed.
If the operation secceeded @code{ctl} is returned, otherwise @code{NULL}
is returned.

This method may only be used while a dialog is active.  It should not be
used prior to performing (via @code{gPerform}) a dialog or after the user
has accepted or canceled the dialog.
@example
@group
@exdent Example:

object  ctl;

gRemoveInt(ctl, 1);
@end group
@end example
@sp 1
See also:  @code{RemoveStr, RemoveAll}
@end deffn














@deffn {RemoveStr} RemoveStr::ListBox
@sp 2
@example
@group
r = gRemoveStr(ctl, itm);

object  ctl;    /*  control object  */
char   *itm;    /*  item            */
object  r;      /*  control object  */
@end group
@end example
This method is used to remove an item from a list box while the dialog
is active.  @code{itm} is the string to be removed.
If the operation secceeded @code{ctl} is returned, otherwise @code{NULL}
is returned.

This method may only be used while a dialog is active.  It should not be
used prior to performing (@code{gPerform}) a dialog or after the user
has accepted or canceled the dialog.
@example
@group
@exdent Example:

object  ctl;

gRemoveStr(ctl, "Some Option");
@end group
@end example
@sp 1
See also:  @code{RemoveInt, FindMode}
@end deffn















@deffn {Required} Required::ListBox
@sp 2
@example
@group
r = gRequired(ctl, mode);

object  ctl;    /*  control object  */
int     mode;   /*  required mode   */
object  r;      /*  control object  */
@end group
@end example
This method is used to determine whether or not the user is required to
make a valid selection prior to WDS allowing the acceptance of the dialog.
If it is required and the user does not make a valid selection, WDS
will issue an error message and return them to the control.

@code{mode} is 1 to make it required and 0 otherwise.
@example
@group
@exdent Example:

object  ctl;

gRequired(ctl, 1);
@end group
@end example
@sp 1
See also:  @code{CheckFunction::Control, CheckValue::Control}
@end deffn








@deffn {SetFunction} SetFunction::ListBox
@sp 2
@example
@group
r = gSetFunction(ctl, fun);

object  ctl;      /*  control object           */
int     (*fun)(); /*  check function           */
ofun    r;        /*  previous check function  */
@end group
@end example
This method is used to associate a C function to a list box such that if
the user double clicks an item in the list box the C function will
immediately get evoked.  The value returned by this method is the
function, if any, which was previously associated with the control.

The C function takes the following form:
@example
@group
int     fun(object ctl, object dlg)
@{
        ....
@}
@end group
@end example
Where @code{ctl} is the control object, and @code{dlg} is the object
representing the dialog the control is associated with.  The value
returned by @code{fun} is ignored.
@example
@group
@exdent Example:

object  ctl;

gSetFunction(ctl, fun);
@end group
@end example
@sp 1
See also:  @code{Perform, SetChgFunction}
@end deffn


















@deffn {SetShortValue} SetShortValue::ListBox
@sp 2
@example
@group
r = gSetShortValue(ctl, val);

object  ctl;    /*  control object  */
int     val;    /*  ctl value       */
object  r;      /*  control object  */
@end group
@end example
This method is used to set the default selection associated with the
control.  It is often used to set the initial value prior to performing
a dialog.

@code{val} will be used as the ordinal value of the selected item.  This
value is a 0 based index from the top of the list to the bottom.

Any previously associated object (via @code{Value}) will be disposed
when the new value is set.
@example
@group
@exdent Example:

object  ctl;

gSetShortValue(ctl, 1);
@end group
@end example
@sp 1
See also:  @code{SetStringValue, SetValue}
@end deffn









@deffn {SetStringValue} SetStringValue::ListBox
@sp 2
@example
@group
r = gSetStringValue(ctl, val);

object  ctl;    /*  control object  */
char    *val;   /*  ctl value       */
object  r;      /*  control object  */
@end group
@end example
This method is used to set the default selection associated with the
control.  It is often used to set the initial value prior to performing
a dialog.  

@code{val} will be used to select the option which matches the string.

Any previously associated object (via @code{val}) will be disposed when
the new value is set.  
@example
@group
@exdent Example:

object  ctl;

gSetStringValue(ctl, "The choice");
@end group
@end example
@sp 1
See also:  @code{SetShortValue, SetValue}
@end deffn










@deffn {SetValue} SetValue::ListBox
@sp 2
@example
@group
r = gSetValue(ctl, val);

object  ctl;    /*  control object  */
object  val;    /*  ctl value       */
object  r;      /*  control object  */
@end group
@end example
This method is used to set the default selection associated with the
control.  It is often used to set the initial value prior to performing
a dialog.  @code{val} should be a Dynace object which is an instance of
either the @code{String} class or the @code{ShortInteger} class and
initialized to the value desired for the control.

If @code{val} is an instance of the @code{ShortIntgeger} class it will be
used as the ordinal value of the selected item.  This value is a 0 based
index from the top of the list to the bottom.

If @code{val} is an instance of the @code{String} class it will be used
to select the option which matches the string represented by @code{val}.

Any previously associated object will be disposed when the new value is set.
Also, @code{val} will automatically be disposed when the control or associated
dialog is disposed.
@example
@group
@exdent Example:

object  ctl;

gSetValue(ctl, gNewWithInt(ShortInteger, 1));
        or
gSetValue(ctl, gNewWithStr(String, "The choice"));
@end group
@end example
@sp 1
See also:  @code{SetShortValue, SetStringValue}
@end deffn













@deffn {ShortValue} ShortValue::ListBox
@sp 2
@example
@group
val = gShortValue(ctl);

object  ctl;   /*  control object  */
short   val;   /*  ctl value       */
@end group
@end example
This method is used to obtain a C short which represents the value
associated with the control.  The returned value will represent the
ordinal value of the choice made by the user.  This ordinal value is a
zero based index from top to bottom.  -1 indicates that no choice was
made.
@example
@group
@exdent Example:

object  ctl;
short   val;

val = gShortValue(ctl);
@end group
@end example
@sp 1
See also:  @code{ListIndex, Value, GetSelections, StringValue, SetValue}
@end deffn










@deffn {StringValue} StringValue::ListBox
@sp 2
@example
@group
val = gStringValue(ctl);

object  ctl;   /*  control object  */
char    *val;  /*  ctl value       */
@end group
@end example
This method is used to obtain a character string pointer which
represents the value associated with the control.  The returned pointer
will represent the text associated with the user selected choices.
It will point to "" if the user had not made a valid selection.

The returned pointer will not be valid once the control object or
associated dialog is disposed.
@example
@group
@exdent Example:

object  ctl;
char    *val;

val = gStringValue(ctl);
@end group
@end example
@sp 1
See also:  @code{ShortValue, Value, ListIndex, GetSelections, SetValue}
@end deffn










@deffn {Value} Value::ListBox
@sp 2
@example
@group
val = gValue(ctl);

object  ctl;   /*  control object  */
object  val;   /*  ctl value       */
@end group
@end example
This method is used to obtain a Dynace object which represents the value
associated with the control.  The object returned will be an instance of
the @code{String} class.  The returned object will represent the text
associated with the user selected choices.  If no selection was made
the object will represent "".

This object will be disposed by WDS when the control object or
associated dialog is disposed.
@example
@group
@exdent Example:

object  ctl, val;

val = gValue(ctl);
@end group
@end example
@sp 1
See also:  @code{ShortValue, StringValue, ValueAt, ListIndex, GetSelections, SetValue}
@end deffn









@deffn {ValueAt} ValueAt::ListBox
@sp 2
@example
@group
val = gValueAt(ctl, idx);

object  ctl;   /*  control object  */
int     idx;   /*  index           */
object  val;   /*  value at idx    */
@end group
@end example
This method is used to obtain a Dynace object which represents the text
associated with the line indexed by the zero based index @code{idx}.
The object returned will be an instance of the @code{String} class.  If
the index is out of range this method will return @code{NULL}.

This method may only be called when the dialog is active.  The object
returned @emph{must} be explicitly disposed when no longer needed.
@c @example
@c @group
@c @exdent Example:
@c 
@c @end group
@c @end example
@sp 1
See also:  @code{ShortValue, Value, ListIndex, GetSelections, SetValue}
@end deffn








@subsection Combo Boxes
The @code{ComboBox} class represents a control which allows the user to
select from a list of choices or enter an alternate text string.  These
choices are presented in a vertical, possibly drop down, list of text
choices.  This control supports access to the user's choice via the text
option selected or an ordinal value.

The application specifies, at runtime, the choices located in the list box.

The main difference between list boxes and combo boxes is that with list
boxes the user can only select from the available choices, however, in
addition, combo boxes allow the user to enter a choice independent of
the available options.

The @code{ComboBox} class is a subclass of @code{Control} and as such
inherits all the functionality associated with that class.  This section
only documents functionality particular to this class.

Standard control arguments are documented in the section entitled
``Standard Control Method Arguments''.










@deffn {AddEdtHandlerAfter} AddEdtHandlerAfter::ComboBox
@sp 2
@example
@group
r = gAddEdtHandlerAfter(ctl, msg, func);

object   ctl;      /*  a control object  */
unsigned msg;      /*  message           */
long    (*func)(); /*  function pointer  */
object  r;         /*  the control obj   */
@end group
@end example
This method is used to associate function @code{func} with the text
entry window portion of the combo box, message @code{msg} for
control @code{ctl}.  Whenever the text entry window of control
@code{ctl} receives message @code{msg}, @code{func} will be called.

@code{ctl} is the control object who's text window messages you wish to
process.  @code{msg} is the particular message you wish to trap.  These
messages are fully documented in the Windows documentation in the
Messages section.  They normally begin with @code{WM_}.

@code{func} is the function which gets called whenever the specified
message gets received and takes the following form:
@example
@group
long    func(object     ctl,
             HWND       hwnd, 
             UINT       mMsg, 
             WPARAM     wParam, 
             LPARAM     lParam)
@{
        .
        .
        .
        return 0L;  /* or whatever is appropriate  */
@}
@end group
@end example
Where @code{ctl} is the control being sent the message.  The remaining
arguments and return value is fully documented in the Windows documentation
under the @code{WindowProc} function and the Windows Messages documentation.

WDS keeps a list of functions associated with each message associated
with each control.  When a particular message is received the appropriate
list of handler functions gets executed sequentially.
@code{AddEdtHandlerAfter} appends the new function to the end of this list,
and @code{AddEdtHandlerBefore} adds the new function to the beginning of
the list.

WDS may also, and optionally, execute the Windows default procedure
associated with a given message either before or after the user added
list of functions.  This behavior may be controlled via
@code{DefaultEdtProcessingMode}.

Windows will only see the return value of the last message handler executed
including, if applicable, the default.
@example
@group
@exdent Example:

int     hSize, vSize;

static  long    process_wm_size(object  ctl, 
                                HWND    hwnd, 
                                UINT    mMsg, 
                                WPARAM  wParam, 
                                LPARAM  lParam)
@{
        hSize = LOWORD(lParam);
        vSize = HIWORD(lParam);
        return 0L;
@}

        .
        .
        gAddEdtHandlerAfter(ctl, (unsigned) WM_SIZE,
                                       process_wm_size);
        .
        .
@end group
@end example
@sp 1
See also:  @code{DefaultProcessingMode, AddEdtHandlerBefore,}
@iftex
@hfil @break @hglue .63in 
@end iftex
@code{AddHandlerAfter::Control}
@end deffn






@deffn {AddEdtHandlerBefore} AddEdtHandlerBefore::ComboBox
@sp 2
@example
@group
r = gAddEdtHandlerBefore(ctl, msg, func);

object   ctl;      /*  a control object  */
unsigned msg;      /*  message           */
long    (*func)(); /*  function pointer  */
object  r;         /*  the control obj   */
@end group
@end example
This function is fully documented under @code{AddEdtHandlerAfter}.
@c @example
@c @group
@c @exdent Example:
@c @end group
@c @end example
@sp 1
See also:  @code{AddEdtHandlerAfter, AddHandlerAfter::Control}
@end deffn













@deffn {AddOption} AddOption::ComboBox
@sp 2
@example
@group
r = gAddOption(ctl, op);

object  ctl;    /*  control object  */
char    *op;    /*  new option      */
object  r;      /*  control object  */
@end group
@end example
This method is used to append a new option to the list of options associated
with a combo box.  This method may be used prior to or during the execution
(via @code{gPerform}) of a dialog.

@code{op} may also be a Dynace object, an instance of the @code{String}
class, representing the new choice.  If a Dynace object is used, it must
be typecast to a string and it will be disposed when the control or
associated dialog are disposed.
@example
@group
@exdent Example:

object  ctl;

gAddOption(ctl, "The first choice");
gAddOption(ctl, "The second choice");
@end group
@end example
@sp 1
See also:  @code{Alphabetize::ComboBox, AddOptionAt::ComboBox}
@end deffn












@deffn {Alphabetize} Alphabetize::ComboBox
@sp 2
@example
@group
r = gAlphabetize(ctl);

object  ctl;    /*  control object  */
object  r;      /*  control object  */
@end group
@end example
This method is used to cause the automatic alphabetization of all
options added via the @code{AddOption} method.  In order to function
properly, @code{Alphabetize} must be called prior to any calls to
@code{AddOption}.  This method simply returns the object passed.
@c @example
@c @group
@c @exdent Example:
@c 
@c @end group
@c @end example
@sp 1
See also:  @code{AddOption::ComboBox}
@end deffn












@deffn {Attach} Attach::ComboBox
@sp 2
@example
@group
r = gAttach(ctl, val);

object  ctl;   /*  control object  */
object  val;   /*  ctl value       */
object  r;     /*  control object  */
@end group
@end example
This method is used to associate an independent Dynace object with the
state associated with a control object.  @code{val} should be an
instance of the @code{ShortInteger} class or the @code{String} class.
This object will be automatically updated to reflect the state
associated with the control.

If @code{val} is an instance of the @code{ShortIntgeger} class it will
be used to represent the ordinal value of the selected item.  This value
is a 0 based index from the top of the list to the bottom.  -1 is used
to represent no valid selection.

If @code{val} is an instance of the @code{String} class it will be used
to represent the string represented by the user's choice.  If no choice is
made it will represent "".

This object (@code{val}) will never be disposed by WDS, even after
the control or associated dialog are disposed.  Therefore, this
is one way of gaining access to a control's value after the life
of the control.  It is the programmer's responsibility to dispose of
the object when it is no longer needed.
@example
@group
@exdent Example:

object  ctl, val;

val = gNewWithInt(ShortInteger, 0);
gAttach(ctl, val);
@end group
@end example
@sp 1
See also:  @code{Value, ShortValue, StringValue}
@end deffn












@deffn {DefaultEdtProcessingMode} DefaultEdtProcessingMode::ComboBox
@sp 2
@example
@group
r = gDefaultEdtProcessingMode(ctl, msg, mode);

object   ctl;   /*  a control object         */
unsigned msg;   /*  message                  */
int      mode;  /*  default processing mode  */
object   r;     /*  the control obj          */
@end group
@end example
This method is used to determine when or if the Windows default message
procedure is processed for a given message (@code{msg}) associated with
the text entry window portion of control (@code{ctl}).

WDS allows a programmer to specify an arbitrary number of functions to
be executed whenever the text entry portion of a combo box control
receives a specific message (via @code{AddEdtHandlerAfter} and
@code{AddEdtHandlerBefore}).  Windows has default procedures associated
with many control messages.  At times it is necessary to replace or
augment this default functionality.  @code{DefaultEdtProcessingMode}
gives the programmer control over when and if this default Windows
functionality.  @code{mode} is used to specify the desired mode.  The
following table indicates the valid modes:

@table @code
@item 0
Do not execute the Windows default processing
@item 1
Execute default processing @emph{after} programmer defined handlers
@item 2
Execute default processing @emph{before} programmer defined handlers
@end table

Note that the default mode is always @code{1}, and must be explicitly
changed, if desired, for each message associated with each control.

@code{msg} is the particular message you wish to affect.  These messages
are fully documented in the Windows documentation in the Messages
section.  They normally begin with @code{WM_}.
@example
@group
@exdent Example:

gDefaultEdtProcessingMode(ctl, (unsigned) WM_SIZE, 0);
@end group
@end example
@sp 1
See also:  @code{AddEdtHandlerAfter}
@end deffn

















@deffn {ListIndex} ListIndex::ComboBox
@sp 2
@example
@group
val = gListIndex(ctl);

object  ctl;   /*  control object  */
object  val;   /*  ctl value       */
@end group
@end example
This method is used to obtain a Dynace object which represents the value
associated with the control.  The object returned will be an instance of
the @code{ShortInteger} class.  The returned object will represent the
ordinal value of the choice made by the user.  This ordinal value is a
zero based index from top to bottom.  -1 indicates that no choice was
made.

This object will be disposed by WDS when the control object or
associated dialog is disposed.
@example
@group
@exdent Example:

object  ctl, val;

val = gListIndex(ctl);
@end group
@end example
@sp 1
See also:  @code{ShortValue, Value, StringValue, Attach, SetValue}
@end deffn




@deffn {NewCtl} NewCtl::ComboBox
@sp 2
@example
@group
ctl = mNewCtl(ComboBox, id);

unsigned  id;   /*  control id      */
object   ctl;   /*  control object  */
@end group
@end example
This method is used to create a new control object to be identified as
@code{id} (see section ``Standard Control Method Arguments'').  This
method is mainly used internally.  A programmer would more often
use @code{AddControl::Dialog} to create controls and associated them
to a dialog.
@example
@group
@exdent Example:

object  ctl;

ctl = mNewCtl(ComboBox, MY_COMBOBOX);
@end group
@end example
@sp 1
See also:  @code{AddControl::Dialog, GetControl::Dialog, AddOption}
@end deffn




















@deffn {RemoveAll} RemoveAll::ComboBox
@sp 2
@example
@group
r = gRemoveAll(ctl);

object  ctl;    /*  control object  */
object  r;      /*  control object  */
@end group
@end example
This method is used to remove all items from a combo box.  It may be used
prior to or during the execution of a dialog.  The control object passed
is returned.
@c @example
@c @group
@c @exdent Example:
@c 
@c @end group
@c @end example
@sp 1
See also:  @code{RemoveStr, RemoveInt}
@end deffn

















@deffn {RemoveInt} RemoveInt::ComboBox
@sp 2
@example
@group
r = gRemoveInt(ctl, idx);

object  ctl;    /*  control object  */
int     idx;    /*  index           */
object  r;      /*  control object  */
@end group
@end example
This method is used to remove an item from a combo box while the dialog
is active.  @code{idx} is a zero based index of the item to be removed.
If the operation secceeded @code{ctl} is returned, otherwise @code{NULL}
is returned.

This method may only be used while a dialog is active.  It should not be
used prior to performing (@code{gPerform}) a dialog or after the user
has accepted or canceled the dialog.
@example
@group
@exdent Example:

object  ctl;

gRemoveInt(ctl, 1);
@end group
@end example
@sp 1
See also:  @code{RemoveStr, RemoveAll}
@end deffn














@deffn {RemoveStr} RemoveStr::ComboBox
@sp 2
@example
@group
r = gRemoveStr(ctl, itm);

object  ctl;    /*  control object  */
char   *itm;    /*  item            */
object  r;      /*  control object  */
@end group
@end example
This method is used to remove an item from a combo box while the dialog
is active.  @code{itm} is the string to be removed.
If the operation secceeded @code{ctl} is returned, otherwise @code{NULL}
is returned.

This method may only be used while a dialog is active.  It should not be
used prior to performing (@code{gPerform}) a dialog or after the user
has accepted or canceled the dialog.
@example
@group
@exdent Example:

object  ctl;

gRemoveStr(ctl, "Some Option");
@end group
@end example
@sp 1
See also:  @code{RemoveInt::ComboBox, FindMode::ComboBox}
@end deffn

















@deffn {Required} Required::ComboBox
@sp 2
@example
@group
r = gRequired(ctl, mode);

object  ctl;    /*  control object  */
int     mode;   /*  required mode   */
object  r;      /*  control object  */
@end group
@end example
This method is used to determine whether or not the user is required to
make a valid selection prior to WDS allowing the acceptance of the dialog.
If it is required and the user does not make a valid selection, WDS
will issue an error message and return them to the control.

@code{mode} is 1 to make it required and 0 otherwise.
@example
@group
@exdent Example:

object  ctl;

gRequired(ctl, 1);
@end group
@end example
@sp 1
See also:  @code{CheckFunction::Control, CheckValue::Control}
@end deffn



















@deffn {SetFunction} SetFunction::ComboBox
@sp 2
@example
@group
r = gSetFunction(ctl, fun);

object  ctl;      /*  control object           */
int     (*fun)(); /*  check function           */
ofun    r;        /*  previous check function  */
@end group
@end example
This method is used to associate a C function to a combo box such that if
the user double clicks an item in the combo box the C function will
immediately get evoked.  The value returned by this method is the
function, if any, which was previously associated with the control.

The C function takes the following form:
@example
@group
int     fun(object ctl, object dlg)
@{
        ....
@}
@end group
@end example
Where @code{ctl} is the control object, and @code{dlg} is the object
representing the dialog the control is associated with.  The value
returned by @code{fun} is ignored.
@example
@group
@exdent Example:

object  ctl;

gSetFunction(ctl, fun);
@end group
@end example
@sp 1
See also:  @code{Perform, SetChgFunction}
@end deffn






















@deffn {SetShortValue} SetShortValue::ComboBox
@sp 2
@example
@group
r = gSetShortValue(ctl, val);

object  ctl;    /*  control object  */
int     val;    /*  ctl value       */
object  r;      /*  control object  */
@end group
@end example
This method is used to set the default selection associated with the
control.  It is often used to set the initial value prior to performing
a dialog.

@code{val} will be used as the ordinal value of the selected item.  This
value is a 0 based index from the top of the list to the bottom.

Any previously associated object (via @code{Value}) will be disposed
when the new value is set.
@example
@group
@exdent Example:

object  ctl;

gSetShortValue(ctl, 1);
@end group
@end example
@sp 1
See also:  @code{SetStringValue, SetValue}
@end deffn









@deffn {SetStringValue} SetStringValue::ComboBox
@sp 2
@example
@group
r = gSetStringValue(ctl, val);

object  ctl;    /*  control object  */
char    *val;   /*  ctl value       */
object  r;      /*  control object  */
@end group
@end example
This method is used to set the default selection associated with the
control.  It is often used to set the initial value prior to performing
a dialog.  

@code{val} will be used to select the option which matches the string.

Any previously associated object (via @code{val}) will be disposed when
the new value is set.  
@example
@group
@exdent Example:

object  ctl;

gSetStringValue(ctl, "The choice");
@end group
@end example
@sp 1
See also:  @code{SetShortValue, SetValue}
@end deffn










@deffn {SetValue} SetValue::ComboBox
@sp 2
@example
@group
r = gSetValue(ctl, val);

object  ctl;    /*  control object  */
object  val;    /*  ctl value       */
object  r;      /*  control object  */
@end group
@end example
This method is used to set the default selection associated with the
control.  It is often used to set the initial value prior to performing
a dialog.  @code{val} should be a Dynace object which is an instance of
either the @code{String} class or the @code{ShortInteger} class and
initialized to the value desired for the control.

If @code{val} is an instance of the @code{ShortIntgeger} class it will be
used as the ordinal value of the selected item.  This value is a 0 based
index from the top of the list to the bottom.

If @code{val} is an instance of the @code{String} class it will be used
to select the option which matches the string represented by @code{val}.

Any previously associated object will be disposed when the new value is set.
Also, @code{val} will automatically be disposed when the control or associated
dialog is disposed.
@example
@group
@exdent Example:

object  ctl;

gSetValue(ctl, gNewWithInt(ShortInteger, 1));
        or
gSetValue(ctl, gNewWithStr(String, "The choice"));
@end group
@end example
@sp 1
See also:  @code{SetShortValue, SetStringValue}
@end deffn













@deffn {ShortValue} ShortValue::ComboBox
@sp 2
@example
@group
val = gShortValue(ctl);

object  ctl;   /*  control object  */
short   val;   /*  ctl value       */
@end group
@end example
This method is used to obtain a C short which represents the value
associated with the control.  The returned value will represent the
ordinal value of the choice made by the user.  This ordinal value is a
zero based index from top to bottom.  -1 indicates that no choice was
made.
@example
@group
@exdent Example:

object  ctl;
short   val;

val = gShortValue(ctl);
@end group
@end example
@sp 1
See also:  @code{ListIndex, Value, StringValue, Attach, SetValue}
@end deffn










@deffn {StringValue} StringValue::ComboBox
@sp 2
@example
@group
val = gStringValue(ctl);

object  ctl;   /*  control object  */
char    *val;  /*  ctl value       */
@end group
@end example
This method is used to obtain a character string pointer which
represents the value associated with the control.  The returned pointer
will represent the text associated with the user selected choices.
It will point to "" if the user had not made a valid selection.

The returned pointer will not be valid once the control object or
associated dialog is disposed.
@example
@group
@exdent Example:

object  ctl;
char    *val;

val = gStringValue(ctl);
@end group
@end example
@sp 1
See also:  @code{ShortValue, Value, ListIndex, Attach, SetValue}
@end deffn










@deffn {Value} Value::ComboBox
@sp 2
@example
@group
val = gValue(ctl);

object  ctl;   /*  control object  */
object  val;   /*  ctl value       */
@end group
@end example
This method is used to obtain a Dynace object which represents the value
associated with the control.  The object returned will be an instance of
the @code{String} class.  The returned object will represent the text
associated with the user selected choices.  If no selection was made
the object will represent "".

This object will be disposed by WDS when the control object or
associated dialog is disposed.
@example
@group
@exdent Example:

object  ctl, val;

val = gValue(ctl);
@end group
@end example
@sp 1
See also:  @code{ShortValue, StringValue, ValueAt, ListIndex, Attach, SetValue}
@end deffn






@deffn {ValueAt} ValueAt::ComboBox
@sp 2
@example
@group
val = gValueAt(ctl, idx);

object  ctl;   /*  control object  */
int     idx;   /*  index           */
object  val;   /*  value at idx    */
@end group
@end example
This method is used to obtain a Dynace object which represents the text
associated with the line indexed by the zero based index @code{idx}.
The object returned will be an instance of the @code{String} class.  If
the index is out of range this method will return @code{NULL}.

This method may only be called when the dialog is active.  The object
returned @emph{must} be explicitly disposed when no longer needed.
@c @example
@c @group
@c @exdent Example:
@c 
@c @end group
@c @end example
@sp 1
See also:  @code{ShortValue, Value, ListIndex, SetValue}
@end deffn









@subsection Scroll Bars
The @code{ScrollBar} class represents a control which allows the user to
select a number between two ranges via a linear, visually intuitive
control.  The range defaults to 0 to 100 and may be controlled via
@code{ScrollBarRange}.

The @code{ScrollBar} class is a subclass of @code{Control} and as such
inherits all the functionality associated with that class.  This section
only documents functionality particular to this class.

Standard control arguments are documented in the section entitled
``Standard Control Method Arguments''.










@deffn {Attach} Attach::ScrollBar
@sp 2
@example
@group
r = gAttach(ctl, val);

object  ctl;   /*  control object  */
object  val;   /*  ctl value       */
object  r;     /*  control object  */
@end group
@end example
This method is used to associate an independent Dynace object with the
state associated with a control object.  @code{val} should be an
instance of the @code{ShortInteger} class.  This object will be
automatically updated to reflect the position of the scroll bar.

This object (@code{val}) will never be disposed by WDS, even after
the control or associated dialog are disposed.  Therefore, this
is one way of gaining access to a control's value after the life
of the control.  It is the programmer's responsibility to dispose of
the object when it is no longer needed.
@example
@group
@exdent Example:

object  ctl, val;

val = gNewWithInt(ShortInteger, 0);
gAttach(ctl, val);
@end group
@end example
@sp 1
See also:  @code{Value, ShortValue, SetValue, SetShortValue,}
        @code{ScrollBarRange}
@end deffn










@deffn {Increment} Increment::ScrollBar
@sp 2
@example
@group
r = gIncrement(ctl, val);

object  ctl;   /*  control object  */
int     val;   /*  increment value */
object  r;     /*  control object  */
@end group
@end example
This method is used to increment the position of the scroll bar
@code{val} points from its current position.  @code{val} may
be negative or positive, which will determine the direction of movement.
@example
@group
@exdent Example:

object  ctl;

gIncrement(ctl, 10);
@end group
@end example
@sp 1
See also:  @code{SetShortValue, SetValue, ScrollBarRange}
@end deffn











@deffn {NewCtl} NewCtl::ScrollBar
@sp 2
@example
@group
ctl = mNewCtl(ScrollBar, id);

unsigned  id;   /*  control id      */
object   ctl;   /*  control object  */
@end group
@end example
This method is used to create a new control object to be identified as
@code{id} (see section ``Standard Control Method Arguments'').  This
method is mainly used internally.  A programmer would more often
use @code{AddControl::Dialog} to create controls and associated them
to a dialog.
@example
@group
@exdent Example:

object  ctl;

ctl = mNewCtl(ScrollBar, MY_SCROLLBAR);
@end group
@end example
@sp 1
See also:  @code{AddControl::Dialog, GetControl::Dialog}
@end deffn






@deffn {ScrollBarRange} ScrollBarRange::ScrollBar
@sp 2
@example
@group
r = gScrollBarRange(ctl, min, max, pginc, lininc)

object  ctl;    /*  control object  */
int     min;    /*  minimum range   */
int     max;    /*  maximum range   */
int     pginc;  /*  page increment  */
int     lineinc;/*  line increment  */
object  r;      /*  control object  */
@end group
@end example
This method is used to control the range and scrolling characteristics
associated with a given scroll bar control.

@code{min} and @code{max} are used to determine the range of the
control.  These are the values which are returned when the control
is at either end of its extremes.  The default values are 0 and
100 respectively.

@code{pginc} is used to control exactly how much the position of the
scroll bar will change when the user clicks the control in such a way
as to cause a page jump in the position.  This is normally done by
clicking in the area on either side of position indicating element of the
control.  The default value for the variable is 10.  It must be less than
the difference between @code{max} and @code{min}, and is normally larger
than @code{lineinc}.

@code{lineinc} us used to control exactly how much the position of the
scroll bar will change when the user clicks the line increment arrows
located on the extreme ends of the control.  The default value of this
control is 2.
@example
@group
@exdent Example:

object  ctl;

gScrollBarRange(ctl, 0, 100, 10, 2);
@end group
@end example
@c @sp 1
@c See also:  @code{}
@end deffn














@deffn {SetShortValue} SetShortValue::ScrollBar
@sp 2
@example
@group
r = gSetShortValue(ctl, val);

object  ctl;    /*  control object  */
int     val;    /*  ctl value       */
object  r;      /*  control object  */
@end group
@end example
This method is used to set the value associated with a control.  It is
often used to set the initial value prior to performing a dialog.
@code{val} should be the value which represents the desired
position of the scroll bar.  It must be between the minimum and
maximum range associated with the control.

Any previously associated value object (associated via @code{SetValue})
will be disposed when the new value is set.
@example
@group
@exdent Example:

object  ctl;

gSetShortValue(ctl, 50);
@end group
@end example
@sp 1
See also:  @code{SetValue, Increment, CtlShortValue::Dialog,}
        @code{ScrollBarRange}
@end deffn













@deffn {SetValue} SetValue::ScrollBar
@sp 2
@example
@group
r = gSetValue(ctl, val);

object  ctl;    /*  control object  */
object  val;    /*  ctl value       */
object  r;      /*  control object  */
@end group
@end example
This method is used to set the value associated with a control.  It is
often used to set the initial value prior to performing a dialog.
@code{val} should be a Dynace object which is an instance of the
@code{ShortInteger} class and initialized to the value desired for the
control.

@code{val} must represent a number between the minimum and maximum range
associated with the scroll bar.

Any previously associated object will be disposed when the new value is set.
Also, @code{val} will automatically be disposed when the control or associated
dialog is disposed.
@example
@group
@exdent Example:

object  ctl;

gSetValue(ctl, gNewWithInt(ShortInteger, 50));
@end group
@end example
@sp 1
See also:  @code{SetShortValue, Increment, Value, ScrollBarRange}
@end deffn













@deffn {ShortValue} ShortValue::ScrollBar
@sp 2
@example
@group
val = gShortValue(ctl);

object  ctl;   /*  control object  */
short   val;   /*  ctl value       */
@end group
@end example
This method is used to obtain a C short integer which represents the
position of the scroll bar.  The returned value will represent the
position of the scroll bar control.  This number will be between the
minimum and maximum range associated with the control via
@code{ScrollBarRange}.
@example
@group
@exdent Example:

object  ctl;
short   val;

val = gShortValue(ctl);
@end group
@end example
@sp 1
See also:  @code{Value, Attach, SetShortValue, CtlShortValue::Dialog}
@end deffn









@deffn {Value} Value::ScrollBar
@sp 2
@example
@group
val = gValue(ctl);

object  ctl;   /*  control object  */
object  val;   /*  ctl value       */
@end group
@end example
This method is used to obtain a Dynace object which represents the value
associated with the control.  The object returned will be an instance of
the @code{ShortInteger} class.  The returned object will represent
the position of the scroll bar control.  This number will be between
the minimum and maximum range associated with the control via
@code{ScrollBarRange}.

This object will be disposed by WDS when the control object or
associated dialog is disposed.
@example
@group
@exdent Example:

object  ctl, val;

val = gValue(ctl);
@end group
@end example
@sp 1
See also:  @code{ShortValue, Attach, SetValue}
@end deffn









@subsection Custom Controls


@section Cursors
The @code{Cursor} class (as well as its subclasses) are used to represent
cursor objects.  This class is never used by an application.  It is used
to house the functionality common to its subclasses.  Therefore, this
class documents functionality which is common to all of its subclasses.

Note that a more convenient mechanism for using cursors is provided by
@code{LoadCursor::Window} and @code{LoadSystemCursor::Window}









@deffn {Copy} Copy::Cursor
@sp 2
@example
@group
c = gCopy(csr);

object  csr;    /*  cursor object          */
object  c;      /*  copy of cursor object  */
@end group
@end example
This method is used to create a new cursor object which is a copy of a
given cursor object.  The value of this is to be able to associate a
cursor with multiple objects which will automatically dispose of the
cursor when they are disposed.  If copies weren't used the second object
referencing the cursor would reference a disposed cursor object.

Each cursor object must be disposed (either automatically or manually)
when it is no longer needed.
@example
@group
@exdent Example:

object  c1, c2;

c2 = gCopy(c1);
@end group
@end example
@sp 1
See also:  @code{DeepCopy}
@end deffn










@deffn {DeepCopy} DeepCopy::Cursor
@sp 2
This method performs the same function as @code{Copy}.  See that
method for details.
@end deffn










@deffn {DeepDispose} DeepDispose::Cursor
@sp 2
This method performs the same function as @code{Dispose}.  See that
method for details.
@end deffn







@deffn {Dispose} Dispose::Cursor
@sp 2
@example
@group
r = gDispose(csr);

object  csr;   /*  a cursor object    */
object  r;     /*  NULL               */
@end group
@end example
This method is used to dispose of a cursor object when it
is no longer needed.  This method is rarely needed due to the fact that
when a window is disposed it automatically calls this method to
dispose of its associated cursor object.

The value returned is always @code{NULL} and may be used to null out
the variable which contained the object being disposed in order to
avoid future accidental use.
@example
@group
@exdent Example:

object  csr;

csr = gDispose(csr);
@end group
@end example
@sp 1
See also:  @code{DeepDispose}
@end deffn










@deffn {Handle} Handle::Cursor
@sp 2
@example
@group
h = gHandle(csr);

object  csr;    /*  cursor object   */
HANDLE  h;      /*  Windows handle  */
@end group
@end example
This method is used to obtain the Windows internal handle associated with
a cursor object.  

Note that this method may be used with most WDS objects in order to obtain
the internal handle that Windows normally associates with each type of object.
See the appropriate documentation.
@example
@group
@exdent Example:

object  csr;
HCURSOR h;

h = gHandle(csr);
@end group
@end example
@c @sp 1
@c See also:  @code{}
@end deffn









@subsection System Cursors
The @code{SystemCursor} class represents Windows predefined cursors.
This class is a subclass of @code{Cursor} and as such inherits most
of its functionality from it.  This section documents the methods
particular to this class.  See the @code{Cursor} class for additional
functionality.











@deffn {LoadSys} LoadSys::SystemCursor
@sp 2
@example
@group
csr = gLoadSys(SystemCursor, id);

char    *id;    /*  cursor id      */
object  csr;    /*  cursor object  */
@end group
@end example
This method is used to create a new object which represents one of the
Windows predefined cursors.  @code{id} should be one of the Windows
defined macros specifying the desired cursor.  It is defined in the
Windows documentation under the function @code{LoadCursor} and
starts with @code{IDC_}.
@example
@group
@exdent Example:

object  csr;

csr = gLoadSys(SystemCursor, IDC_IBEAM);
@end group
@end example
@sp 1
See also:  @code{LoadSystemCursor::Window, SetCursor::Application}
@end deffn





@subsection External Cursors
The @code{ExternalCursor} class represents arbitrary application defined
cursors.  This class is a subclass of @code{Cursor} and as such inherits
most of its functionality from it.  This section documents the methods
particular to this class.  See the @code{Cursor} class for additional
functionality.






@deffn {Load} Load::ExternalCursor
@sp 2
@example
@group
csr = mLoad(ExternalCursor, id);

unsigned id;    /*  cursor id      */
object   csr;   /*  cursor object  */
@end group
@end example
This method is used to load a programmer defined application specific
cursor.

@code{id} is a programmer defined unsigned integer which identifies
the cursor.  This identifier is normally a macro and defined through the
resource editor.  

The value returned is an object representing the cursor loaded, or
@code{NULL} if the cursor was not found.
@example
@group
@exdent Example:

object  csr;

csr = mLoad(ExternalCursor, MY_CURSOR);
@end group
@end example
@sp 1
See also:  @code{LoadCursor::Window, SetCursor::Application}
@end deffn











@deffn {LoadStr} LoadStr::ExternalCursor
@sp 2
@example
@group
csr = mLoadStr(ExternalCursor, id);

char    *id;    /*  cursor id      */
object   csr;   /*  cursor object  */
@end group
@end example
This method is used to load a programmer defined application specific
cursor by name.

@code{id} is a programmer defined name which identifies the cursor.
This identifier is normally defined through the resource editor.

The value returned is an object representing the cursor loaded, or
@code{NULL} if the cursor was not found.
@example
@group
@exdent Example:

object  csr;

csr = mLoadStr(ExternalCursor, "mycursor");
@end group
@end example
@sp 1
See also:  @code{LoadCursor::Window, SetCursor::Application,}
        @code{Use::Window}
@end deffn




@section Icons
The @code{Icon} class (as well as its subclasses) are used to represent
icon objects.  This class is never used by an application.  It is used
to house the functionality common to its subclasses.  Therefore, this
class documents functionality which is common to all of its subclasses.

Note that a more convenient mechanism for using icons is provided by
@code{LoadIcon::Window} and @code{LoadSystemIcon::Window}









@deffn {Copy} Copy::Icon
@sp 2
@example
@group
c = gCopy(icn);

object  icn;    /*  icon object          */
object  c;      /*  copy of icon object  */
@end group
@end example
This method is used to create a new icon object which is a copy of a
given icon object.  The value of this is to be able to associate a
icon with multiple objects which will automatically dispose of the
icon when they are disposed.  If copies weren't used the second object
referencing the icon would reference a disposed icon object.

Each icon object must be disposed (either automatically or manually)
when it is no longer needed.
@example
@group
@exdent Example:

object  i1, i2;

i2 = gCopy(i1);
@end group
@end example
@sp 1
See also:  @code{DeepCopy}
@end deffn










@deffn {DeepCopy} DeepCopy::Icon
@sp 2
This method performs the same function as @code{Copy}.  See that
method for details.
@end deffn










@deffn {DeepDispose} DeepDispose::Icon
@sp 2
This method performs the same function as @code{Dispose}.  See that
method for details.
@end deffn







@deffn {Dispose} Dispose::Icon
@sp 2
@example
@group
r = gDispose(icn);

object  icn;   /*  an icon object  */
object  r;     /*  NULL            */
@end group
@end example
This method is used to dispose of an icon object when it is no longer
needed.  This method is rarely needed due to the fact that when a window
is disposed it automatically calls this method to dispose of its
associated icon object.

The value returned is always @code{NULL} and may be used to null out
the variable which contained the object being disposed in order to
avoid future accidental use.
@example
@group
@exdent Example:

object  icn;

icn = gDispose(icn);
@end group
@end example
@sp 1
See also:  @code{DeepDispose}
@end deffn










@deffn {Handle} Handle::Icon
@sp 2
@example
@group
h = gHandle(icn);

object  icn;    /*  icon object     */
HANDLE  h;      /*  Windows handle  */
@end group
@end example
This method is used to obtain the Windows internal handle associated with
an icon object.  

Note that this method may be used with most WDS objects in order to obtain
the internal handle that Windows normally associates with each type of object.
See the appropriate documentation.
@example
@group
@exdent Example:

object  icn;
HICON   h;

h = gHandle(icn);
@end group
@end example
@c @sp 1
@c See also:  @code{}
@end deffn









@subsection System Icons
The @code{SystemIcon} class represents Windows predefined icons.
This class is a subclass of @code{Icon} and as such inherits most
of its functionality from it.  This section documents the methods
particular to this class.  See the @code{Icon} class for additional
functionality.











@deffn {LoadSys} LoadSys::SystemIcon
@sp 2
@example
@group
icn = gLoadSys(SystemIcon, id);

char    *id;    /*  icon id      */
object  icn;    /*  icon object  */
@end group
@end example
This method is used to create a new object which represents one of the
Windows predefined icons.  @code{id} should be one of the Windows
defined macros specifying the desired icon.  It is defined in the
Windows documentation under the function @code{LoadIcon} and
starts with @code{IDI_}.
@example
@group
@exdent Example:

object  icn;

icn = gLoadSys(SystemIcon, IDI_HAND);
@end group
@end example
@sp 1
See also:  @code{LoadSystemIcon::Window, SetIcon::Application}
@end deffn





@subsection External Icons
The @code{ExternalIcon} class represents arbitrary application defined
icons.  This class is a subclass of @code{Icon} and as such inherits
most of its functionality from it.  This section documents the methods
particular to this class.  See the @code{Icon} class for additional
functionality.






@deffn {Load} Load::ExternalIcon
@sp 2
@example
@group
icn = mLoad(ExternalIcon, id);

unsigned id;    /*  icon id      */
object   icn;   /*  icon object  */
@end group
@end example
This method is used to load a programmer defined application specific
icon.

@code{id} is a programmer defined unsigned integer which identifies
the icon.  This identifier is normally a macro and defined through the
resource editor.  

The value returned is an object representing the icon loaded, or
@code{NULL} if the icon was not found.
@example
@group
@exdent Example:

object  icn;

icn = mLoad(ExternalIcon, MY_ICON);
@end group
@end example
@sp 1
See also:  @code{LoadIcon::Window, SetIcon::Application}
@end deffn











@deffn {LoadStr} LoadStr::ExternalIcon
@sp 2
@example
@group
icn = mLoadStr(ExternalIcon, id);

char    *id;    /*  icon id      */
object   icn;   /*  icon object  */
@end group
@end example
This method is used to load a programmer defined application specific
icon by name.

@code{id} is a programmer defined name which identifies the icon.
This identifier is normally defined through the resource editor.

The value returned is an object representing the icon loaded, or
@code{NULL} if the icon was not found.
@example
@group
@exdent Example:

object  icn;

icn = mLoadStr(ExternalIcon, "myicon");
@end group
@end example
@sp 1
See also:  @code{LoadIcon::Window, SetIcon::Application,}
        @code{Use::Window}
@end deffn





@section Fonts
The @code{Font} class (as well as its subclasses) are used to represent
font objects.  This class is never used by an application.  It is used
to house the functionality common to its subclasses.  Therefore, this
class documents functionality which is common to all of its subclasses.

Note that a more convenient mechanism for using fonts is provided by
@code{LoadFont::Window, LoadSystemFont::Window, LoadFont::Printer},
and @code{LoadSystemFont::Printer}.










@deffn {AveCharWidth} AveCharWidth::Font
@sp 2
@example
@group
r = gAveCharWidth(fnt);

object   fnt;  /*  a font object  */
int      r;    /*  ave char width */
@end group
@end example
This method is used to gain access to the average character width
associated with a particular font.  
@example
@group
@exdent Example:

object  fnt;
int     cw;

cw = gAveCharWidth(fnt);
@end group
@end example
@sp 1
See also:  @code{LineHeight, GetTM}
@end deffn










@deffn {Copy} Copy::Font
@sp 2
@example
@group
c = gCopy(fnt);

object  fnt;    /*  font object          */
object  c;      /*  copy of font object  */
@end group
@end example
This method is used to create a new font object which is a copy of a
given font object.  The value of this is to be able to associate a
font with multiple objects which will automatically dispose of the
font when they are disposed.  If copies weren't used the second object
referencing the font would reference a disposed font object.

Each font object must be disposed (either automatically or manually)
when it is no longer needed.
@example
@group
@exdent Example:

object  f1, f2;

f2 = gCopy(f1);
@end group
@end example
@sp 1
See also:  @code{DeepCopy}
@end deffn











@deffn {DeepCopy} DeepCopy::Font
@sp 2
This method performs the same function as @code{Copy}.  See that
method for details.
@end deffn











@deffn {DeepDispose} DeepDispose::Font
@sp 2
This method performs the same function as @code{Dispose}.  See that
method for details.
@end deffn








@deffn {Dispose} Dispose::Font
@sp 2
@example
@group
r = gDispose(fnt);

object  fnt;   /*  a font object  */
object  r;     /*  NULL           */
@end group
@end example
This method is used to dispose of a font object when it is no longer
needed.  This method is rarely needed due to the fact that when a window
or printer object is disposed it automatically calls this method to
dispose of its associated font object.

The value returned is always @code{NULL} and may be used to null out
the variable which contained the object being disposed in order to
avoid future accidental use.
@example
@group
@exdent Example:

object  fnt;

fnt = gDispose(fnt);
@end group
@end example
@sp 1
See also:  @code{DeepDispose}
@end deffn










@deffn {GetTM} GetTM::Font
@sp 2
@example
@group
r = gGetTM(fnt, tm);

object      fnt;  /*  a font object  */
TEXTMETRIC  *tm;  /*  font metrics   */
object      r;    /*  fnt            */
@end group
@end example
This method is used to gain access to the font metrics associated with
a particular font.
@example
@group
@exdent Example:

object  fnt;
TEXTMETRIC  tm;

gGetTM(fnt, &tm);
@end group
@end example
@sp 1
See also:  @code{LineHeight, AveCharWidth}
@end deffn













@deffn {Handle} Handle::Font
@sp 2
@example
@group
h = gHandle(fnt);

object  fnt;    /*  font object     */
HANDLE  h;      /*  Windows handle  */
@end group
@end example
This method is used to obtain the Windows internal handle associated with
a font object.  

Note that this method may be used with most WDS objects in order to obtain
the internal handle that Windows normally associates with each type of object.
See the appropriate documentation.
@example
@group
@exdent Example:

object  fnt;
HGDIOBJ h;

h = gHandle(fnt);
@end group
@end example
@c @sp 1
@c See also:  @code{}
@end deffn














@deffn {LineHeight} LineHeight::Font
@sp 2
@example
@group
r = gLineHeight(fnt);

object   fnt;  /*  a font object  */
int      r;    /*  line height    */
@end group
@end example
This method is used to gain access to the line height associated with
a particular font.  This number is equal to the height of the largest
character in the font plus a reasonable amount of space to separate
lines containing the font.
@example
@group
@exdent Example:

object  fnt;
int     lh;

lh = gLineHeight(fnt);
@end group
@end example
@sp 1
See also:  @code{AveCharWidth, GetTM}
@end deffn










@subsection System Fonts
The @code{SystemFont} class represents Windows predefined fonts.
This class is a subclass of @code{Font} and as such inherits most
of its functionality from it.  This section documents the methods
particular to this class.  See the @code{Font} class for additional
functionality.








@deffn {Load} Load::SystemFont
@sp 2
@example
@group
fnt = mLoad(SystemFont, id);

unsigned id;    /*  icon id      */
object   fnt;   /*  icon object  */
@end group
@end example
This method is used to create a new object which represents one of the
Windows predefined fonts.  @code{id} should be one of the Windows
defined macros specifying the desired font.  It is defined in the
Windows documentation under the function @code{GetStockObject} and
ends with @code{_FONT}.
@example
@group
@exdent Example:

object  fnt;

fnt = mLoad(SystemFont, ANSI_FIXED_FONT);
@end group
@end example
@sp 1
See also:  @code{LoadSystemFont::Window, LoadSystemFont::Printer,}
@iftex
@hfil @break @hglue .63in 
@end iftex
@code{SetFont::Application, Use::Window, Use::Printer}
@end deffn











@subsection External Fonts
The @code{ExternalFont} class represents arbitrary external fonts.  This
class is a subclass of @code{Font} and as such inherits most of its
functionality from it.  This section documents the methods particular to
this class.  See the @code{Font} class for additional functionality.









@deffn {Indirect} Indirect::ExternalFont
@sp 2
@example
@group
fnt = gIndirect(ExternalFont, lf);

LOGFONT *lf;    /*  logical font  */
object  fnt;    /*  font object   */
@end group
@end example
This method is used to find and load a font which most closely matches
the parameters indicated by @code{lf}.  The structure of @code{lf} is
fully described in the Windows documentation under the structure
@code{LOGFONT}.

The value returned is an object representing the font loaded, or
@code{NULL} if the font was not found.
@example
@group
@exdent Example:

LOGFONT lf;
object  fnt;

/*  initialize lf  */
fnt = gIndirect(ExternalFont, &lf);
@end group
@end example
@sp 1
See also:  @code{LoadFont::Window, LoadFont::Printer, SetFont::Application,}
@iftex
@hfil @break @hglue .63in 
@end iftex
@code{Use::Window, Use::Printer}
@end deffn










@deffn {New} New::ExternalFont
@sp 2
@example
@group
fnt = vNew(ExternalFont, nm, ps);

char    *nm;    /*  font name    */
int     ps;     /*  point size   */
object  fnt;    /*  font object  */
@end group
@end example
This method is used to load an arbitrary font by name at any point size.
@code{nm} is the full name of the font as it appears
when you list the available fonts via the control-panel / fonts Windows
utility, minus the font type in parentheses.  @code{ps} indicates the
desired point size.

The value returned is an object representing the font loaded, or
@code{NULL} if the font was not found.

Note that @code{nm} may also be a Dynace object cast as a (@code{char *}).
@example
@group
@exdent Example:

object  fnt;

fnt = vNew(ExternalFont, "Times New Roman", 12);
@end group
@end example
@sp 1
See also:  @code{LoadFont::Window, LoadFont::Printer, SetFont::Application,}
@iftex
@hfil @break @hglue .63in 
@end iftex
@code{Use::Window, Use::Printer}
@end deffn









@section Brushes
The @code{Brush} class (as well as its subclasses) are used to represent
brush objects.  This class is never used by an application.  It is used
to house the functionality common to its subclasses.  Therefore, this
class documents functionality which is common to all of its subclasses.









@deffn {Color} Color::Brush
@sp 2
@example
@group
c = gColor(bsh);

object    bsh;  /*  brush object   */
COLORREF  c;    /*  brush color    */
@end group
@end example
This method is used to obtain the color associated with a brush.
@example
@group
@exdent Example:

object  bsh;
COLORREF  c;

c = gColor(bsh);
@end group
@end example
@c @sp 1
@c See also:  @code{}
@end deffn










@deffn {Copy} Copy::Brush
@sp 2
@example
@group
c = gCopy(bsh);

object  bsh;    /*  brush object          */
object  c;      /*  copy of brush object  */
@end group
@end example
This method is used to create a new brush object which is a copy of a
given brush object.  The value of this is to be able to associate a
brush with multiple objects which will automatically dispose of the
brush when they are disposed.  If copies weren't used the second object
referencing the brush would reference a disposed brush object.

Each brush object must be disposed (either automatically or manually)
when it is no longer needed.
@example
@group
@exdent Example:

object  b1, b2;

b2 = gCopy(b1);
@end group
@end example
@sp 1
See also:  @code{DeepCopy}
@end deffn









@deffn {DeepCopy} DeepCopy::Brush
@sp 2
This method performs the same function as @code{Copy}.  See that
method for details.
@end deffn





@deffn {DeepDispose} DeepDispose::Brush
@sp 2
This method performs the same function as @code{Dispose}.  See that
method for details.
@end deffn







@deffn {Dispose} Dispose::Brush
@sp 2
@example
@group
r = gDispose(bsh);

object  bsh;   /*  a brush object  */
object  r;     /*  NULL            */
@end group
@end example
This method is used to dispose of a brush object when it is no longer
needed.  This method is rarely needed due to the fact that when a window
or printer object is disposed it automatically calls this method to
dispose of its associated brush object.

The value returned is always @code{NULL} and may be used to null out
the variable which contained the object being disposed in order to
avoid future accidental use.
@example
@group
@exdent Example:

object  bsh;

bsh = gDispose(bsh);
@end group
@end example
@sp 1
See also:  @code{DeepDispose}
@end deffn






@deffn {Handle} Handle::Brush
@sp 2
@example
@group
h = gHandle(bsh);

object  bsh;    /*  brush object    */
HANDLE  h;      /*  Windows handle  */
@end group
@end example
This method is used to obtain the Windows internal handle associated with
a brush object.  

Note that this method may be used with most WDS objects in order to obtain
the internal handle that Windows normally associates with each type of object.
See the appropriate documentation.
@example
@group
@exdent Example:

object  bsh;
HBRUSH  h;

h = gHandle(bsh);
@end group
@end example
@c @sp 1
@c See also:  @code{}
@end deffn








@subsection Stock Brushes
The @code{StockBrush} class represents common Windows defined brushes.  This
class is a subclass of @code{Brush} and as such inherits most of its
functionality from it.  This section documents the methods particular to
this class.  See the @code{Brush} class for additional functionality.






@deffn {New} New::StockBrush
@sp 2
@example
@group
bsh = vNew(StockBrush, id);

unsigned id;    /*  brush id      */
object  bsh;    /*  brush object  */
@end group
@end example
This method is used to load one of the Windows pre-defined brushes.
The available options are macros documented in the Windows documentation
under the function @code{GetStockObject} and end with @code{_BRUSH}.

The value returned is an object representing the brush loaded, or
@code{NULL} if the brush was not found.
@example
@group
@exdent Example:

object  bsh;

bsh = vNew(StockBrush, GRAY_BRUSH);
@end group
@end example
@sp 1
See also:  @code{TextBrush::Window, BackBrush::Window,}
        @code{Use::Window, Use::Printer}
@end deffn









@subsection Solid Brushes
The @code{SolidBrush} class represents arbitrary application defined
brushes.  This class is a subclass of @code{Brush} and as such inherits
most of its functionality from it.  This section documents the methods
particular to this class.  See the @code{Brush} class for additional
functionality.






@deffn {New} New::SolidBrush
@sp 2
@example
@group
bsh = vNew(SolidBrush, red, green, blue);

int     red;
int     green;
int     blue;
object  bsh;    /*  brush object  */
@end group
@end example
This method is used to create a solid colored brush with specific colors.
Each color argument specifies a number between 0 and 255 and indicates
the intensity of the related color.  If all three are 0 you get black,
and all three at 255 is white.

The value returned is an object representing the brush created, or
@code{NULL} if the brush was not created.
@example
@group
@exdent Example:

object  bsh;

bsh = vNew(SolidBrush, 255, 0, 0);
@end group
@end example
@sp 1
See also:  @code{TextBrush::Window, BackBrush::Window,}
        @code{Use::Window, Use::Printer}
@end deffn




@subsection System Brushes
The @code{SystemBrush} class represents brushes which are those colors
which were selected by the user as global to their Windows environment.
The programmer may select, for example, the color the user chose for
background windows.

This class is a subclass of @code{Brush} and as such inherits
most of its functionality from it.  This section documents the methods
particular to this class.  See the @code{Brush} class for additional
functionality.








@deffn {New} New::SystemBrush
@sp 2
@example
@group
bsh = vNew(SystemBrush, id);

unsigned id;    /*  brush id      */
object  bsh;    /*  brush object  */
@end group
@end example
This method is used to load one of the Windows user defined brushes.
The available options are macros documented in the Windows documentation
under the function @code{GetSysColor}.

The value returned is an object representing the brush loaded, or
@code{NULL} if the brush was not found.
@example
@group
@exdent Example:

object  bsh;

bsh = vNew(SystemBrush, COLOR_ACTIVEBORDER);
@end group
@end example
@sp 1
See also:  @code{TextBrush::Window, BackBrush::Window,}
        @code{Use::Window, Use::Printer}
@end deffn










@subsection Hatch Brushes
The @code{HatchBrush} class represents arbitrary application defined
brushes with a specified pattern.  This class is a subclass of
@code{Brush} and as such inherits most of its functionality from it.
This section documents the methods particular to this class.  See the
@code{Brush} class for additional functionality.






@deffn {New} New::HatchBrush
@sp 2
@example
@group
bsh = vNew(HatchBrush, red, green, blue, style);

int     red;
int     green;
int     blue;
int     style;  /*  brush pattern  */
object  bsh;    /*  brush object   */
@end group
@end example
This method is used to create a colored brush with specific colors and
pattern Each color argument specifies a number between 0 and 255 and
indicates the intensity of the related color.  If all three are 0 you
get black, and all three at 255 is white.

The @code{style} parameter indicates the specific pattern for the brush.
This parameter is a macro documented in the Windows documentation under
the function @code{CreateHatchBrush} and begin with @code{HS_}.

The value returned is an object representing the brush created, or
@code{NULL} if the brush was not created.
@example
@group
@exdent Example:

object  bsh;

bsh = vNew(HatchBrush, 255, 10, 10, HS_VERTICAL);
@end group
@end example
@sp 1
See also:  @code{TextBrush::Window, BackBrush::Window,}
        @code{Use::Window, Use::Printer}
@end deffn









@section Pens
The @code{Pen} class (as well as its subclasses) are used to represent
pen objects.  This class is never used by an application.  It is used
to house the functionality common to its subclasses.  Therefore, this
class documents functionality which is common to all of its subclasses.









@deffn {Color} Color::Pen
@sp 2
@example
@group
c = gColor(pn);

object    pn;   /*  pen object   */
COLORREF  c;    /*  pen color    */
@end group
@end example
This method is used to obtain the color associated with a pen.
@example
@group
@exdent Example:

object    pn;
COLORREF  c;

c = gColor(pn);
@end group
@end example
@c @sp 1
@c See also:  @code{}
@end deffn










@deffn {Copy} Copy::Pen
@sp 2
@example
@group
c = gCopy(pn);

object  pn;     /*  pen object          */
object  c;      /*  copy of pen object  */
@end group
@end example
This method is used to create a new pen object which is a copy of a
given pen object.  The value of this is to be able to associate a
pen with multiple objects which will automatically dispose of the
pen when they are disposed.  If copies weren't used the second object
referencing the pen would reference a disposed pen object.

Each pen object must be disposed (either automatically or manually)
when it is no longer needed.
@example
@group
@exdent Example:

object  p1, p2;

p2 = gCopy(p1);
@end group
@end example
@sp 1
See also:  @code{DeepCopy}
@end deffn









@deffn {DeepCopy} DeepCopy::Pen
@sp 2
This method performs the same function as @code{Copy}.  See that
method for details.
@end deffn





@deffn {DeepDispose} DeepDispose::Pen
@sp 2
This method performs the same function as @code{Dispose}.  See that
method for details.
@end deffn







@deffn {Dispose} Dispose::Pen
@sp 2
@example
@group
r = gDispose(pn);

object  pn;   /*  a pen object  */
object  r;    /*  NULL          */
@end group
@end example
This method is used to dispose of a pen object when it is no longer
needed.  This method is rarely needed due to the fact that when a window
or printer object is disposed it automatically calls this method to
dispose of its associated pen object.

The value returned is always @code{NULL} and may be used to null out
the variable which contained the object being disposed in order to
avoid future accidental use.
@example
@group
@exdent Example:

object  pn;

pn = gDispose(pn);
@end group
@end example
@sp 1
See also:  @code{DeepDispose}
@end deffn






@deffn {Handle} Handle::Pen
@sp 2
@example
@group
h = gHandle(pn);

object  pn;     /*  pen object    */
HANDLE  h;      /*  Windows handle  */
@end group
@end example
This method is used to obtain the Windows internal handle associated with
a pen object.  

Note that this method may be used with most WDS objects in order to obtain
the internal handle that Windows normally associates with each type of object.
See the appropriate documentation.
@example
@group
@exdent Example:

object  pn;
HPEN    h;

h = gHandle(pn);
@end group
@end example
@c @sp 1
@c See also:  @code{}
@end deffn









@subsection Stock Pens
The @code{StockPen} class represents common Windows defined pens.  This
class is a subclass of @code{Pen} and as such inherits most of its
functionality from it.  This section documents the methods particular to
this class.  See the @code{Pen} class for additional functionality.






@deffn {New} New::StockPen
@sp 2
@example
@group
pn = vNew(StockPen, id);

unsigned id;    /*  pen id      */
object  pn;     /*  pen object  */
@end group
@end example
This method is used to load one of the Windows pre-defined pens.
The available options are macros documented in the Windows documentation
under the function @code{GetStockObject} and end with @code{_PEN}.

The value returned is an object representing the pen loaded, or
@code{NULL} if the pen was not found.
@example
@group
@exdent Example:

object  pn;

pn = vNew(StockPen, BLACK_PEN);
@end group
@end example
@sp 1
See also:  @code{Use::Printer}
@end deffn







@subsection Custom Pens
The @code{CustomPen} class represents arbitrary application defined pens
with a specified pattern and width.  This class is a subclass of
@code{Pen} and as such inherits most of its functionality from it.  This
section documents the methods particular to this class.  See the
@code{Pen} class for additional functionality.






@deffn {New} New::CustomPen
@sp 2
@example
@group
pn = vNew(CustomPen, red, green, blue, style, width);

int     red;
int     green;
int     blue;
int     style;  /*  pen pattern     */
int     width;  /*  pen line width  */
object  pn;     /*  pen object      */
@end group
@end example
This method is used to create a colored pen with specific colors,
pattern and line width.  Each color argument specifies a number between
0 and 255 and indicates the intensity of the related color.  If all
three are 0 you get black, and all three at 255 is white.

The @code{style} parameter indicates the specific pattern for the pen.
This parameter is a macro documented in the Windows documentation under
the function @code{CreatePen} and begin with @code{PS_}.
The @code{width} parameter specifies the width of the line in logical
units.

The value returned is an object representing the pen created, or
@code{NULL} if the pen was not created.
@example
@group
@exdent Example:

object  pn;

pn = vNew(CustomPen, 255, 10, 10, PS_DASH, 5);
@end group
@end example
@sp 1
See also:  @code{Use::Printer}
@end deffn




@section Help System
The @code{HelpSystem} class is used to support the standard Windows
help system.  In addition to being able to support arbitrary evocation
of help messages, this class is used by other WDS classes in order to
provide complete support for context sensitive help to any level.
Each class provides mechanisms in order to associated context sensitive
help text to windows, dialogs, menus or controls.

Most classes which provide context sensitive help support have a method
called @code{SetTopic} which is used to associate a particular topic
within a given context.

All the methods in this class are class methods.  This means that
the first argument to all the methods will be @code{HelpSystem} and
there is never an instance object to keep track of.

See the appropriate examples for detailed information on how to
create the actual help file.











@deffn {HelpContents} HelpContents::HelpSystem
@sp 2
@example
@group
r = gHelpContents(HelpSystem);

object  r;      /*  HelpSystem  */
@end group
@end example
This method is used to display the contents screen of the help file.
@code{HelpSystem} will be returned if the request is successful and
@code{NULL} otherwise.
@example
@group
@exdent Example:

gHelpContents(HelpSystem);
@end group
@end example
@sp 1
See also:  @code{HelpFile, HelpTopic, HelpInContext}
@end deffn















@deffn {HelpFile} HelpFile::HelpSystem
@sp 2
@example
@group
r = gHelpFile(HelpSystem, hf);

char    *hf;    /*  help file   */
object  r;      /*  HelpSystem  */
@end group
@end example
This method is used to determine the external file to be used for all help
references.  This must be done prior to any of the other help facilities
use.  The help file will not be opened until one of the help display
methods is evoked.  The application must also have a main window prior
to the evocation of the help system.  The help system will automatically
terminate when the user terminates the application.
@example
@group
@exdent Example:

gHelpFile(HelpSystem, "helpfile.hlp");
@end group
@end example
@sp 1
See also:  @code{HelpContents}
@end deffn














@deffn {HelpInContext} HelpInContext::HelpSystem
@sp 2
@example
@group
r = gHelpInContext(HelpSystem);

object  r;      /*  HelpSystem  */
@end group
@end example
This method is used to display the help screen associated with the
current context (set with @code{SetTopic}).  If no context is set the
help contents will be displayed.  @code{HelpSystem} will be returned if
the request is successful and @code{NULL} otherwise.

Note that @code{SetTopic} and @code{HelpInContext}, although available
to an application program, are mainly used internally by WDS in order to
support the context sensitive help facility provided by the @code{Window},
@code{Dialog}, @code{Menu}, and @code{Control} classes.  See the
@code{SetTopic} method associated with those classes.
@example
@group
@exdent Example:

gHelpInContext(HelpSystem);
@end group
@end example
@sp 1
See also:  @code{HelpFile, SetTopic, HelpContents, HelpTopic}
@end deffn







@deffn {HelpTopic} HelpTopic::HelpSystem
@sp 2
@example
@group
r = gHelpTopic(HelpSystem, tpc);

char    *tpc;   /*  help topic  */
object  r;      /*  HelpSystem  */
@end group
@end example
This method is used to display the help screen associated with a given topic.
@iftex
@hfil @break 
@end iftex
@code{HelpSystem} will be returned if the request is successful
and @code{NULL} otherwise.
@example
@group
@exdent Example:

gHelpTopic(HelpSystem, "fileOpen");
@end group
@end example
@sp 1
See also:  @code{HelpFile, HelpContents, HelpInContext}
@end deffn














@deffn {SetTopic} SetTopic::HelpSystem
@sp 2
@example
@group
r = gSetTopic(HelpSystem, tpc);

char    *tpc;   /*  help topic      */
char    *r;     /*  previous topic  */
@end group
@end example
This method is used to set the current help topic context.  It is
used by @code{HelpInContext} to display the help text associated
with the current context.  The previous help topic will be returned.

Note that @code{SetTopic} and @code{HelpInContext}, although available
to an application program, are mainly used internally by WDS in order to
support the context sensitive help facility provided by the @code{Window},
@code{Dialog}, @code{Menu}, and @code{Control} classes.  See the
@code{SetTopic} method associated with those classes.
@example
@group
@exdent Example:

gSetTopic(HelpSystem, "someTopic");
@end group
@end example
@sp 1
See also:  @code{HelpFile, HelpInContext, HelpContents, HelpTopic}
@end deffn










@section Common Dialogs
The @code{CommonDialog} class is only used to group the common dialogs
which are all subclasses of this class.  There is no specific
functionality associated with this class.  See the particular subclasses
for documentation.






@subsection File Selection Dialog
The @code{FileDialog} class provides a standard and convenient method of
querying the user for a file name.  The user is able to browse the disk
or enter a new file name.

The @code{fd} parameter used by all methods in this class refer to
the file dialog object returned by the @code{New} method.














@deffn {AppendFilter} AppendFilter::FileDialog
@sp 2
@example
@group
r = gAppendFilter(fd, ttl, fltr);

object  fd;     /*  file dialog  */
char    *ttl;   /*  title        */
char    *fltr;  /*  file filter  */
object  r;      /*  file dialog  */
@end group
@end example
This method is used to group files according to some file naming
criteria for user selection.  There may be any number of groups.  The
user selects the group they are interested in and are presented with
a list of files which meet the criteria.

@code{ttl} is the name of the group which the user is presented with.
@code{fltr} is the file filter and may consist of several filters,
each separated with a semicolon.

This method may be called any number of times to create several groups.
@example
@group
@exdent Example:

object  fd;

gAppendFilter(fd, "Document Files", "*.txt;*.doc");
gAppendFilter(fd, "Source Files", "*.c;*.h");
@end group
@end example
@sp 1
See also:  @code{SetFile, InitialDir, DefExt}
@end deffn














@deffn {DefExt} DefExt::FileDialog
@sp 2
@example
@group
r = gDefExt(fd, ext);

object  fd;     /*  file dialog     */
char    *ext;   /*  file extension  */
object  r;      /*  file dialog     */
@end group
@end example
This method is used to set the default file extension which will be
displayed for the user.  It should not contain a period.
@example
@group
@exdent Example:

object  fd;

gDefExt(fd, "txt");
@end group
@end example
@sp 1
See also:  @code{SetFile, AppendFilter}
@end deffn






@deffn {Dispose} Dispose::FileDialog
@sp 2
@example
@group
r = gDispose(fd);

object  fd;   /*  file dialog  */
object  r;    /*  NULL         */
@end group
@end example
This method is used to dispose of a file dialog object.  This method must
be called to dispose of the dialog when it is no longer needed.

The value returned is always @code{NULL} and may be used to null out
the variable which contained the object being disposed in order to
avoid future accidental use.
@example
@group
@exdent Example:

object  fd;

fd = gDispose(fd);
@end group
@end example
@c @sp 1
@c See also:  @code{}
@end deffn











@deffn {GetFile} GetFile::FileDialog
@sp 2
@example
@group
fname = gGetFile(fd);

object  fd;     /*  file dialog  */
char    *fname; /*  file name    */
@end group
@end example
This method is used to obtain the file name the user selected once the
file dialog has been completed.  This value will include the full path,
file name and extension.  If multiple selections are allowed, the path
will be returned, then a space delimited list of selected file names.
@example
@group
@exdent Example:

object  fd;
char    *files;

files = gGetFile(fd);
@end group
@end example
@sp 1
See also:  @code{GetOpenFile, GetSaveFile, SetFile, SetFlags}
@end deffn















@deffn {InitialDir} InitialDir::FileDialog
@sp 2
@example
@group
r = gInitialDir(fd, pth);

object  fd;     /*  file dialog  */
char    *pth;   /*  initial path */
object  r;      /*  file dialog  */
@end group
@end example
This method is used to set the initial path associated with the file dialog.
If no initial path is set, the system will use the current directory of
the application.
@example
@group
@exdent Example:

object  fd;

gInitialDir(fd, "c:\\mypath");
@end group
@end example
@sp 1
See also:  @code{SetFile, AppendFilter, DefExt}
@end deffn







@deffn {New} New::FileDialog
@sp 2
@example
@group
fd = vNew(FileDialog, pwind);

object  pwind;  /*  parent window  */
object  fd;     /*  file dialog    */
@end group
@end example
This method is used to create a new file dialog object.  It will not be
displayed until the appropriate method is called.  The object returned
must be disposed when it no longer needed.

The @code{pwind} parameter refers to either the application's main window
or any child window object.
@example
@group
@exdent Example:

object  fd, pwind;

fd = vNew(FileDialog, pwind);
@end group
@end example
@sp 1
See also:  @code{Dispose, GetOpenFile, GetSaveFile}
@end deffn





@deffn {GetOpenFile} GetOpenFile::FileDialog
@sp 2
@example
@group
r = gGetOpenFile(fd);

object  fd;     /*  file dialog    */
int     r;      /*  return status  */
@end group
@end example
This method is used to actually display the dialog and obtain user input.
The type of file dialog presented will be one in which the user must
select a pre-existing file, presumably to open.

The return value is non-zero if the user makes a valid selection and
zero if the user did not make a selection or if an error occurred.
@example
@group
@exdent Example:

object  fd;
int     r;

r = gGetOpenFile(fd);
@end group
@end example
@sp 1
See also:  @code{GetSaveFile}
@end deffn






@deffn {GetSaveFile} GetSaveFile::FileDialog
@sp 2
@example
@group
r = gGetSaveFile(fd);

object  fd;     /*  file dialog    */
int     r;      /*  return status  */
@end group
@end example
This method is used to actually display the dialog and obtain user input.
The type of file dialog presented will be one in which the user may
select a pre-existing file or type in a new name, presumably to save some
data to.

The return value is non-zero if the user makes a valid selection and
zero if the user did not make a selection or if an error occurred.
@example
@group
@exdent Example:

object  fd;
int     r;

r = gGetSaveFile(fd);
@end group
@end example
@sp 1
See also:  @code{GetOpenFile}
@end deffn








@deffn {SetFile} SetFile::FileDialog
@sp 2
@example
@group
r = gSetFile(fd, fname);

object  fd;     /*  file dialog  */
char    *fname; /*  file name    */
object  r;      /*  file dialog  */
@end group
@end example
This method is used to set the default file name displayed when the
file dialog is displayed.
@example
@group
@exdent Example:

object  fd;

gSetFile(fd, "somefile.txt");
@end group
@end example
@sp 1
See also:  @code{InitialDir, GetFile, SetFlags, DefExt}
@end deffn







@deffn {SetFlags} SetFlags::FileDialog
@sp 2
@example
@group
r = gSetFlags(fd, flgs);

object  fd;     /*  file dialog  */
DWORD   flgs;   /*  flags        */
object  r;      /*  file dialog  */
@end group
@end example
This method is used to set the option flags associated with a file dialog.
The available flags are fully documented in the Windows documentation
under the @code{OPENFILENAME} structure and all begin with @code{OFN_}.
@example
@group
@exdent Example:

object  fd;

gSetFlags(fd, OFN_ALLOWMULTISELECT);
@end group
@end example
@sp 1
See also:  @code{SetFile, DefExt}
@end deffn















@deffn {SetTitle} SetTitle::FileDialog
@sp 2
@example
@group
r = gSetTitle(fd, ttl);

object  fd;     /*  file dialog  */
char    *ttl;   /*  dialog title */
object  r;      /*  file dialog  */
@end group
@end example
This method is used to set the text which appears at the top of the
file dialog.  If no value is set the system will use appropriate defaults.
@example
@group
@exdent Example:

object  fd;

gSetTitle(fd, "Select File");
@end group
@end example
@sp 1
See also:  @code{SetFile, AppendFilter, DefExt}
@end deffn





@subsection Printer Selection and Configuration Dialog
The @code{PrintDialog} class is used to present the user with a standard
and convenient dialog for printer selection and configuration.

This class is seldom needed because the @code{Printer} class has
the @code{QueryPrinter::Printer} method which automatically calls
this class for printer information.  However, this class may be used
independently in order to have better control of user options.


The @code{pd} parameter used by all methods in this class refer to
the print dialog object returned by the @code{New} method.














@deffn {Copies} Copies::PrintDialog
@sp 2
@example
@group
cp = gCopies(fd);

object  fd;     /*  file dialog  */
int     cp;     /*  copies       */
@end group
@end example
This method is used to obtain the number of copies selected by the user
subsequent to calling @code{Perform}.
@example
@group
@exdent Example:

object  fd;
int     cp;

cp = gCopies(fd);
@end group
@end example
@sp 1
See also:  @code{Perform, SetFlags, GetPageRange}
@end deffn









@deffn {Dispose} Dispose::PrintDialog
@sp 2
@example
@group
r = gDispose(pd);

object  pd;     /*  print dialog   */
object  r;      /*  NULL           */
@end group
@end example
This method is used to dispose of a print dialog object.  This must
be done when it is no longer needed.

The value returned is always @code{NULL} and may be used to null out
the variable which contained the object being disposed in order to
avoid future accidental use.
@example
@group
@exdent Example:

object  pd;

pd = gDispose(pd);
@end group
@end example
@sp 1
See also:  @code{New, Perform}
@end deffn











@deffn {GetPageRange} GetPageRange::PrintDialog
@sp 2
@example
@group
r = gGetPageRange(fd, start, end);

object  fd;     /*  file dialog           */
int     *start; /*  starting page number  */
int     *end;   /*  ending page number    */
object  r;      /*  file dialog           */
@end group
@end example
This method is used to obtain the page range selected by the user
subsequent to calling @code{Perform}.
@example
@group
@exdent Example:

object  fd;
int     start, end;

gGetPageRange(fd, &start, &end);
@end group
@end example
@sp 1
See also:  @code{Perform, SetFlags, Copies}
@end deffn











@deffn {Handle} Handle::PrintDialog
@sp 2
@example
@group
hdc = gHandle(pd);

object  pd;     /*  print dialog   */
HANDLE  hdc;    /*  handle to dc   */
@end group
@end example
This method is used to obtain the device context handle associated with
the print dialog.  It will only be valid after performing the
dialog (via @code{Perform}).  If the handle is obtained via this method,
the @code{Dispose} method will not release the handle as it would
normally do.  This is done so the device context can be used for the
printer to be opened.  If the handle as obtained it is the application's
responsibility to release the handle when it is no longer needed.  This
may be done via the @code{DeleteDC} Windows function.

This method is mainly used internally to WDS and is only made available
for convenience.
@example
@group
@exdent Example:

object  pd;
HDC     hdc;

hdc = gHandle(pd);
@end group
@end example
@sp 1
See also:  @code{Perform}
@end deffn









@deffn {New} New::PrintDialog
@sp 2
@example
@group
pd = vNew(PrintDialog, pwind);

object  pwind;  /*  parent window  */
object  pd;     /*  print dialog   */
@end group
@end example
This method is used to create a new print dialog object.  It will not be
displayed until the appropriate method is called.  The object returned
must be disposed when it no longer needed.

The @code{pwind} parameter refers to either the application's main window
or any child window object.
@example
@group
@exdent Example:

object  pd, pwind;

pd = vNew(PrintDialog, pwind);
@end group
@end example
@sp 1
See also:  @code{Perform, Dispose}
@end deffn












@deffn {Perform} Perform::PrintDialog
@sp 2
@example
@group
r = gPerform(pd);

object  pd;     /*  print dialog   */
int     r;      /*  return status  */
@end group
@end example
This method is used to actually present the user with the print dialog.
The return value is non-zero if the user appropriately selects a printer.
Zero will be returned if the user cancels the dialog or an error occurs.
@example
@group
@exdent Example:

object  pd;
int     r;

r = gPerform(pd);
@end group
@end example
@sp 1
See also:  @code{New, Dispose}
@end deffn







@deffn {SetFlags} SetFlags::PrintDialog
@sp 2
@example
@group
r = gSetFlags(pd, flgs);

object  pd;     /*  print dialog  */
DWORD   flgs;   /*  flags         */
object  r;      /*  print dialog  */
@end group
@end example
This method is used to set the option flags associated with a print dialog.
The available flags are fully documented in the Windows documentation
under the @code{PRINTDLG} structure and all begin with @code{PD_}.
@example
@group
@exdent Example:

object  pd;

gSetFlags(pd, PD_ALLPAGES);
@end group
@end example
@c @sp 1
@c See also:  @code{}
@end deffn














@subsection Color Selection Dialog
The @code{ColorDialog} class is used to present the user with a standard
and convenient dialog for color selection and configuration.

The @code{cd} parameter used by all methods in this class refer to
the color dialog object returned by the @code{New} method.









@deffn {Dispose} Dispose::ColorDialog
@sp 2
@example
@group
r = gDispose(cd);

object  cd;     /*  color dialog   */
object  r;      /*  NULL           */
@end group
@end example
This method is used to dispose of a color dialog object.  This must
be done when it is no longer needed.

The value returned is always @code{NULL} and may be used to null out
the variable which contained the object being disposed in order to
avoid future accidental use.
@example
@group
@exdent Example:

object  cd;

cd = gDispose(cd);
@end group
@end example
@sp 1
See also:  @code{New, Perform}
@end deffn









@deffn {GetColor} GetColor::ColorDialog
@sp 2
@example
@group
clr = gGetColor(cd);

object   cd;    /*  color dialog    */
COLORREF clr;   /*  selected color  */
@end group
@end example
This method is used to obtain the users color selection subsequent to
executing @code{Perform}.
@example
@group
@exdent Example:

object  cd;
COLORREF clr;

clr = gGetColor(cd);
@end group
@end example
@sp 1
See also:  @code{Perform, SetColor, GetCustomColorPtr}
@end deffn













@deffn {GetCustomColorPtr} GetCustomColorPtr::ColorDialog
@sp 2
@example
@group
cv = gGetCustomColorPtr(cd);

object   cd;    /*  color dialog    */
COLORREF *cv;   /*  color vector    */
@end group
@end example
This method is used to obtain a pointer to 16 @code{COLORREF}s
selected as custom colors by the user.  This vector should be
inspected subsequent to calling @code{Perform} and prior to
calling @code{Dispose}.  One @code{Dispose} is called, the returned
pointer will no longer be valid.
@example
@group
@exdent Example:

object  cd;
COLORREF *cv;

cv = gGetCustomColorPtr(cd);
@end group
@end example
@sp 1
See also:  @code{Perform, GetColor}
@end deffn




















@deffn {New} New::ColorDialog
@sp 2
@example
@group
cd = vNew(ColorDialog, pwind);

object  pwind;  /*  parent window  */
object  cd;     /*  color dialog   */
@end group
@end example
This method is used to create a new color dialog object.  It will not be
displayed until the appropriate method is called.  The object returned
must be disposed when it no longer needed.

The @code{pwind} parameter refers to either the application's main window
or any child window object.
@example
@group
@exdent Example:

object  cd, pwind;

cd = vNew(ColorDialog, pwind);
@end group
@end example
@sp 1
See also:  @code{Perform, Dispose}
@end deffn









@deffn {Perform} Perform::ColorDialog
@sp 2
@example
@group
r = gPerform(cd);

object  cd;     /*  color dialog   */
int     r;      /*  return status  */
@end group
@end example
This method is used to actually present the user with the color
selection dialog.  The return value is non-zero if the user appropriately
selects a color.  Zero will be returned if the user canceled the dialog
or an error occurred.
@example
@group
@exdent Example:

object  cd;
int     r;

r = gPerform(cd);
@end group
@end example
@sp 1
See also:  @code{New, Dispose}
@end deffn









@deffn {SetColor} SetColor::ColorDialog
@sp 2
@example
@group
r = gSetColor(cd, clr);

object   cd;    /*  color dialog    */
COLORREF clr;   /*  selected color  */
object   r;     /*  color dialog    */
@end group
@end example
This method is used to set the initial color associated with the
color selection dialog.  The Windows @code{RGB} macro may be used
to obtain a valid @code{COLORREF}.
@example
@group
@exdent Example:

object  cd;
COLORREF clr;

gSetColor(cd, RGB(255, 0 ,0));
@end group
@end example
@sp 1
See also:  @code{Perform, GetColor}
@end deffn














@deffn {SetFlags} SetFlags::ColorDialog
@sp 2
@example
@group
r = gSetFlags(cd, flgs);

object  cd;     /*  color dialog  */
DWORD   flgs;   /*  flags         */
object  r;      /*  color dialog  */
@end group
@end example
This method is used to set the option flags associated with a color dialog.
The available flags are fully documented in the Windows documentation
under the @code{CHOOSECOLOR} structure and all begin with @code{CC_}.
@example
@group
@exdent Example:

object  cd;

gSetFlags(cd, CC_FULLOPEN);
@end group
@end example
@c @sp 1
@c See also:  @code{}
@end deffn











@subsection Font Selection Dialog
The @code{FontDialog} class is used to present the user with a standard
and convenient dialog for font selection.

The @code{fd} parameter used by all methods in this class refer to
the font dialog object returned by the @code{New} method.








@deffn {Dispose} Dispose::FontDialog
@sp 2
@example
@group
r = gDispose(fd);

object  fd;     /*  font dialog   */
object  r;      /*  NULL          */
@end group
@end example
This method is used to dispose of a font dialog object.  This must
be done when it is no longer needed.

The value returned is always @code{NULL} and may be used to null out
the variable which contained the object being disposed in order to
avoid future accidental use.
@example
@group
@exdent Example:

object  fd;

fd = gDispose(fd);
@end group
@end example
@sp 1
See also:  @code{New, Perform}
@end deffn











@deffn {Font} Font::FontDialog
@sp 2
@example
@group
fnt = gFont(fd);

object  fd;     /*  font dialog  */
object  font;   /*  font object  */
@end group
@end example
This method is used to create a font object representing the font the
user selected.  This font object will be an instance of the
@code{ExternalFont} class and must be disposed (either directly or
indirectly) when it is no longer needed.  This method should only
be called subsequent to @code{Perform}.  @code{NULL} will be returned
if no font was selected or the font object could not be created.
@example
@group
@exdent Example:

object  fd, fnt;

fnt = gFont(fd);
@end group
@end example
@sp 1
See also:  @code{Perform}
@end deffn











@deffn {GetColor} GetColor::FontDialog
@sp 2
@example
@group
clr = gGetColor(fd);

object   fd;    /*  font dialog  */
COLORREF clr;   /*  color        */
@end group
@end example
This method is used to get the user selected color once @code{Perform}
has been called.  This will only work if the @code{CF_EFFECTS} option
is set.
@example
@group
@exdent Example:

object  fd;
COLORREF clr;

clr = gGetColor(fd);
@end group
@end example
@sp 1
See also:  @code{SetFlags, SetColor, Perform}
@end deffn














@deffn {New} New::FontDialog
@sp 2
@example
@group
fd = vNew(FontDialog, pwind);

object  pwind;  /*  parent window  */
object  fd;     /*  font dialog    */
@end group
@end example
This method is used to create a new font dialog object.  It will not be
displayed until the appropriate method is called.  The object returned
must be disposed when it no longer needed.

The @code{pwind} parameter refers to either the application's main window
or any child window object.
@example
@group
@exdent Example:

object  fd, pwind;

fd = vNew(FontDialog, pwind);
@end group
@end example
@sp 1
See also:  @code{Perform, Dispose}
@end deffn









@deffn {Perform} Perform::FontDialog
@sp 2
@example
@group
r = gPerform(fd);

object  fd;     /*  font dialog    */
int     r;      /*  return status  */
@end group
@end example
This method is used to actually present the user with the font
selection dialog.  The return value is non-zero if the user appropriately
selects a font.  Zero will be returned if the user canceled the dialog
or an error occurred.
@example
@group
@exdent Example:

object  fd;
int     r;

r = gPerform(fd);
@end group
@end example
@sp 1
See also:  @code{New, Dispose}
@end deffn









@deffn {SetColor} SetColor::FontDialog
@sp 2
@example
@group
clr = gSetColor(fd);

object   fd;    /*  font dialog  */
COLORREF clr;   /*  color        */
@end group
@end example
This method is used to set the initial color of the user color selection
portion of the font dialog.  This will only work if the
@code{CF_EFFECTS} option is set.  The Windows supplied @code{RGB} macro
may be used to create the @code{COLORREF}.
@example
@group
@exdent Example:

object  fd;

gSetColor(fd, RGB(255, 0, 0));
@end group
@end example
@sp 1
See also:  @code{SetFlags, GetColor, Perform}
@end deffn











@deffn {SetFlags} SetFlags::FontDialog
@sp 2
@example
@group
r = gSetFlags(fd, flgs);

object  fd;     /*  font dialog  */
DWORD   flgs;   /*  flags        */
object  r;      /*  font dialog  */
@end group
@end example
This method is used to set the option flags associated with a font dialog.
The available flags are fully documented in the Windows documentation
under the @code{CHOOSEFONT} structure and all begin with @code{CF_}.
@example
@group
@exdent Example:

object  fd;

gSetFlags(fd, CF_ANSIONLY);
@end group
@end example
@c @sp 1
@c See also:  @code{}
@end deffn






@section Dynamic Link Libraries
The @code{DynamicLibrary} class is used to support the dynamic link
library (DLL) facility of Windows.  This class provides a convenient
mechanism to load, free and use DLLs.

Throughout this section @code{dl} will refer to the object which was
returned by the @code{LoadLibrary} method and represents a DLL.








@deffn {DeepDispose} DeepDispose::DynamicLibrary
@sp 2
This method performs the same function as @code{Dispose}.  See that
method for details.
@end deffn







@deffn {Dispose} Dispose::DynamicLibrary
@sp 2
@example
@group
r = gDispose(dl);

object  dl;   /*  a DLL object  */
object  r;    /*  NULL          */
@end group
@end example
This method is used to release and dispose of a DLL object when it is no
longer needed.  All DLL objects are automatically released by WDS when
the application terminates.

The value returned is always @code{NULL} and may be used to null out
the variable which contained the object being disposed in order to
avoid future accidental use.
@example
@group
@exdent Example:

object  dl;

dl = gDispose(dl);
@end group
@end example
@sp 1
See also:  @code{FreeAll, DeepDispose}
@end deffn









@deffn {FindStr} FindStr::DynamicLibrary
@sp 2
@example
@group
dl = gFindStr(DynamicLibrary, fname);

char    *fname; /*  DLL file name  */
object  dl;     /*  DLL object     */
@end group
@end example
This class method is used to obtain the DLL object (previously loaded via
@iftex
@break 
@end iftex
@code{LoadLibrary}) via its associated file name.  This won't work
after the DLL object is disposed.
@example
@group
@exdent Example:

object  dl;

dl = gFindStr(DynamicLibrary, "mydll.dll");
@end group
@end example
@sp 1
See also:  @code{LoadLibrary}
@end deffn










@deffn {FreeAll} FreeAll::DynamicLibrary
@sp 2
@example
@group
gFreeAll(DynamicLibrary);

@end group
@end example
This class method is used to release and dispose of all DLL objects.
It is automatically called by WDS when the application terminates.
@example
@group
@exdent Example:

gFreeAll(DynamicLibrary);
@end group
@end example
@sp 1
See also:  @code{Dispose, DeepDispose}
@end deffn











@deffn {GetProcAddress} GetProcAddress::DynamicLibrary
@sp 2
@example
@group
fun = gGetProcAddress(dl, fname);

object  dl;     /*  DLL object           */
char    *fname; /*  function name        */
FARPROC fun;    /*  pointer to function  */
@end group
@end example
This method is used to obtain a pointer to a function within a
DLL.  The DLL function may then be evoked with the pointer.
@example
@group
@exdent Example:

FARPROC fun;
object  dl;

fun = gGetProcAddress(dl, "myfunc");
@end group
@end example
@sp 1
See also:  @code{FindStr, LoadLibrary}
@end deffn










@deffn {LoadLibrary} LoadLibrary::DynamicLibrary
@sp 2
@example
@group
dl = gLoadLibrary(DynamicLibrary, file);

char    *file;  /*  DLL file    */
object  dl;     /*  DLL object  */
@end group
@end example
This method is used to open a DLL file and create a WDS object which
represents the loaded library.  If the named file does not contain a
path, the normal Windows search path will be searched.  If the
DLL cannot be found or loaded, @code{NULL} will be returned.
@example
@group
@exdent Example:

object  dl;

dl = gLoadLibrary(DynamicLibrary, "mydll.dll");
@end group
@end example
@sp 1
See also:  @code{GetProcAddress, Dispose, FindStr}
@end deffn








